
\setupexternalfigures[directory={img,src/img,src/doc}]


% \useregime[il1]
% \enableregime[il1]
% \useencoding[ffr]

\mainlanguage[fr]

\setuplayout
  [height=middle,
   width=middle,
   backspace=2cm,
   topspace=1cm]

\setuppagenumbering
  [alternative=doublesided]

\setupwhitespace
  [big]

\preloadtypescripts

\definetypeface[mainfacenormal]  [ss][sans] [iwona]       [default]
    \definetypeface[mainfacenormal]  [rm][serif][palatino]    [default]
    \definetypeface[mainfacenormal]  [tt][mono] [modern]      [default][rscale=1.1]
    \definetypeface[mainfacenormal]  [mm][math] [iwona]       [default][encoding=default]

    \definetypeface[mainfacemedium]  [ss][sans] [iwona-medium][default]
    \definetypeface[mainfacenormal]  [rm][serif][palatino]    [default]
    \definetypeface[mainfacemedium]  [tt][mono] [modern]      [default][rscale=1.1]
    \definetypeface[mainfacemedium]  [mm][math] [iwona-medium][default][encoding=default]

    \definetypeface[mainfacenarrowtt][tt][mono] [modern-cond] [default][rscale=1.1]

\setupbodyfont[mainfacenormal,11pt]

\definehead
  [remark]
  [subsubsubject]

\setuphead [chapter]      [style=\mainfacemedium\bfd,color=blue,page=yes]
\setuphead [section]      [style=\mainfacemedium\bfc,color=blue]
\setuphead [subsection]   [style=\mainfacemedium\bfb,color=blue]
\setuphead [subsubsection][style=\mainfacemedium\bfa,color=blue]

\setupheadertexts
  []

\setupcolors
  [state=start]

\setuptyping
  [color=blue,
   style=\mainfacenarrowtt]

\setuptype
  [color=blue,
   style=\mainfacenarrowtt]

\definecolor[blue] [b=.5]
\definecolor[red]  [b=.5]
\definecolor[green][g=.5]
\definecolor[gray][s=.2,t=.5,a=1]
\definecolor[darkred]    [r=.625]



\setupinteraction
    [state=start,
    style=normal,
    menu=on,
    color=darkred,
    contrastcolor=gray,
    style=\ss\tfx\bf\setupspacing,
    title=Informatique ,
    subtitle=Trucs \& astuces,
    author=o.Turlier,
    keyword={AFPA,TEB,CRP,DAO,Autocad,bâtiment,métré,plans,DQE}]	%,
%     openaction=ResetForm]






%----------------------------------------------------------------------------- page titre
\startuseMPgraphic{TitlePage}{darkness}
  StartPage ;

  numeric factor   ; factor   := 1/3 ;
  numeric multiple ; multiple := PaperHeight/PaperWidth ; % 1.6 ;
  numeric stages   ; stages   := multiple/16 ; % .1 ;
  numeric darkness ; darkness := \MPvar{darkness} ;
  def Scaled(expr s, m) =
    if m = 1 :
      scaled (2*s*PaperWidth)
    else :
      xscaled (2*s*PaperWidth) yscaled (2*s*PaperHeight)
    fi
  enddef ;

  fill Page withcolor (factor*white) ;

  fill fullcircle scaled (multiple*PaperWidth) shifted llcorner Page withcolor (factor*red) ;
  fill fullcircle scaled (multiple*PaperWidth) shifted ulcorner Page withcolor (factor*green) ;
  fill fullcircle scaled (multiple*PaperWidth) shifted urcorner Page withcolor (factor*blue) ;
  fill fullcircle scaled (multiple*PaperWidth) shifted lrcorner Page withcolor (factor*yellow) ;

  for i = llcorner Page, ulcorner Page, urcorner Page, lrcorner Page :
    for j = 0 step stages until (10*stages-eps) : % or .8
      fill fullcircle Scaled(j,1) shifted i withcolor transparent(1,\MPvar{darkness}*(1-j),white) ;
    endfor ;
  endfor ;

  draw Page withpen pencircle scaled .1PaperWidth withcolor transparent(1,.5,.5white) ;

  StopPage
\stopuseMPgraphic


%-------------------------------------------------------------------------- no page
\startuseMPgraphic{lualogo}
    color   luaplanetcolor ; luaplanetcolor := .5blue  ;
    color   luaholecolor   ; luaholecolor   :=   white ;
    numeric luaextraangle  ; luaextraangle  := 0 ;

    vardef lualogo = image (
        % Graphic design by A. Nakonechnyj. Copyright (c) 1998, All rights reserved.

        save luaorbitcolor, d, r, p ; color luaorbitcolor ; numeric d, r, p ;

        luaorbitcolor := .5luaholecolor ; d := sqrt(2)/4 ; r := 1/4 ; p := r/8 ;

        fill fullcircle scaled 1 withcolor luaplanetcolor ;
        draw fullcircle rotated 40.5 scaled (1+r) dashed evenly scaled p withpen pencircle scaled (p/2) withcolor luaorbitcolor ;
        fill fullcircle scaled r shifted (d+1/8,d+1/8) rotated luaextraangle withcolor luaplanetcolor ;
        fill fullcircle scaled r shifted (d-1/8,d-1/8) withcolor luaholecolor   ;
    )  enddef ;

\stopuseMPgraphic

\startuseMPgraphic{luanumber}
    \includeMPgraphic{lualogo}
    luaextraangle := \realfolio*360/\lastpage ;
%     luaextraangle := (\realfolio-2)*360/(\lastpage-2) ;
    picture p ; p := lualogo ;
    setbounds p to boundingbox fullcircle ;
    draw p ysized 1cm ;
\stopuseMPgraphic

\definelayer
  [page]
  [width=\paperwidth,
   height=\paperheight]

\setupbackgrounds
  [leftpage]
  [background=page]

\setupbackgrounds
  [rightpage]
  [background=page]

\startsetups pagenumber:right
  \setlayerframed
    [page]
    [preset=rightbottom,offset=1cm]
    [frame=off,height=1cm,offset=overlay]
    {\useMPgraphic{luanumber}}
  \setlayerframed
    [page]
    [preset=rightbottom,offset=1cm,x=1.5cm]
    [frame=off,height=1cm,width=1cm,offset=overlay]
    {\pagenumber}
  \setlayerframed
    [page]
    [preset=rightbottom,offset=1cm,x=2.5cm]
    [frame=off,height=1cm,offset=overlay]
    {Tutoriel : \getmarking[chapter]}
\stopsetups

\startsetups pagenumber:left
  \setlayerframed
    [page]
    [preset=leftbottom,offset=1cm,x=2.5cm]
    [frame=off,height=1cm,offset=overlay]
    {Tutoriel : \getmarking[chapter]}
  \setlayerframed
    [page]
    [preset=leftbottom,offset=1cm,x=1.5cm]
    [frame=off,height=1cm,width=1cm,offset=overlay]
    {\pagenumber}
  \setlayerframed
    [page]
    [preset=leftbottom,offset=1cm]
    [frame=off,height=1cm,offset=overlay]
    {\useMPgraphic{luanumber}}
\stopsetups


\useURL
    [ftp:o.turlier:AT1]
    [{http://o.turlier.free.fr/dl/cours-0506/AT1/}]

\useURL
    [free]
    [{http://free.fr/}]

\useURL
    [ffilezilla]
    [{http://filezilla.sourceforge.net/}]


%@@@@@@@@@@@@@@@@@@@@@@@@@@@@@@@@@@@@@@@

\starttext

% title page

\definelayer
  [TitlePage]
  [width=\paperwidth,
   height=\paperheight]

\setlayer
  [TitlePage]
  {\useMPgraphic{TitlePage}{darkness=1}}

\setlayerframed
  [TitlePage]
  [preset=rightbottom,
   hoffset=.1\paperwidth,
   voffset=.1\paperwidth]
  [align=left,
   width=\hsize,
   frame=off,
   foregroundcolor=gray]
  {\definedfont[SerifBold sa 10]Tutoriel\endgraf
   \definedfont[SerifBold sa 2.48]\darkred \unknown//FTP//\unknown\kern.25\bodyfontsize}

\startTEXpage
  \tightlayer[TitlePage]
\stopTEXpage




\setupbackgrounds
  [leftpage]
  [setups=pagenumber:left]

\setupbackgrounds
  [rightpage]
  [setups=pagenumber:right]

%===============================================================
\title{Sommaire}

\placecontent[criterium=text]

\page

%===============================================================
\title{Introduction}

La formation TEB implique l'usage intensif de l'ordinateur.

La création de documents électroniques implique bien souvent leur
échange au travers d'Internet.

Nous allons explorer quelques moyens de le faire, avec l'objectif
de pouvoir utiliser ces techniques dans n'importe quel
environnement de travail, gr\^ace aux logiciels libres/gratuits
disponibles aujourd'hui.

Ce document se veut autant une introduction à tous ces concepts
d'échanges de fichiers à distance, qu'un guide pas à pas pour
mettre en oeuvre ces techniques.










%===============================================================
\chapter{Principes généraux du FTP}

Il suffit de maîtriser la technique du transfert de fichier car le principe du FTP est simple : accès à distance au disque dur d'un serveur. Les notions de répertoire, fichiers (exécutables, compressés, txt ... sont indispensables.)

Les données sont classées par grands thèmes arborescents souvent par système d'exploitation (DOS, W3.1, Win 95, OS, Mac...)



Architecture client||serveur

%---------------------------------------------------------------------------------------------------------
\section{Utilisation de base du compte FTP}

A travers l'usage de clients FTP \quote{graphiques}, et d'un client en mode \quote{console}, on
retrouve la manipulation de fichiers (déplacer, renommer, déplacer) de l'explorateur.

Les clients graphiques comportent 2 fenêtres principales :
\startitemize[packed]
    \item l'une pour les fichiers locaux, l'autre
    \item pour les fichiers distants
\stopitemize



%----------------------------------------------------------------------------------------------------------
\subsection{Client FTP en ligne de commande}

Windows XP possède un logiciel FTP en ligne de commande (fen\^etre ou l'on rentre des commandes en les écrivant).

Il y en a qui vont dire \quote{\it Mais quel est l'interêt ? Moi j'ai CuteFTP ou FTPexpert et ils sont trés simples d'emploi ; je pige pas pourquoi vous voulez vous embêter à utiliser un client FTP en ligne de commande ?}

Il y a cela deux raison :
-  sur un vrai serveur vous n'aurez pas d'interface graphique, donc au revoir CuteFTP et les autres clickodromes du même genre...
-  vous ne pouvez pas forcement installer un client (PC de l'école, pas le temps de le télecharger)

Le client en ligne de commande est fourni de base avec la plupart des systèmes d'exploitation (Windows inclu), sauf si l'admin par sécurité l'a coupé volontairement.






Avantages \& Limites :
    A : Vous n'avez pas à installer quoi que ce soit
    I : vous devez conna\^itre les commandes de manipulation de fichiers (upload, download, déplacement dans l'arborescence des fichiers, etc.)
            et écrire \type{sans-faire-de_FAUTES-sinon-ca-marche_po.dwg}


Mode d'emploi :
\startitemize
    \item Ouvrez votre console (\bold{Menu déroulant} : Cliquez sur \type{Démarrer} puis \type{Exécuter...} ; \bold{Raccourci clavier} : appui simultané des touches \type{Windows+R}) et écrivez \type{cmd},
    \item ensuite faites \type{ftp ftpperso.free.fr}, on va
    \item vous demander un nom d'utilisateur et le mot de passe. ça y est
    \item vous êtes connectés, reste plus qu'à passer les commandes de transfert.
\stopitemize

\placefigure
    [here,force,nonumber]    %[-8*line,none]  %[here,force]
    [fig:ftp-client-txt]
    {Connexion au serveur FTP de FREE en ligne de commande}
    {\externalfigure  [ftp-connect-client-txt-win.png][width=.8\textwidth]} %[scale=1000]}


Si vous désirez quitter le client FTP tapez simplement \type{bye} ou \type{quit}.

\subject{Navigation sur le FTP}


Dans la pratique, une fois qu'on est connecté, pour naviger, il
faut lister le repertoire courant \type{ls} pour savoir où aller :
decendre \type{cd} ou remonter \type{cd..}. Si vous \^etes perdus,
un petit \type{pwd} vous renverra le nom du répertoire dans lequel
vous \^etes.


\subject{Récupération de fichiers}

\type {get "nom-du-fichier"}

le problème consiste à déterminer le bon chemin, et avoir les autorisations nécessaires pour y écrire

\subject{Envoi de fichiers}

\type{put "nom-du-fichier-qui-se-trouve-sur-mon-PC"}


\subject{Résumé des commandes FTP}

\starttabulate[|lTB|p|]
\HL
\NC Commande \NC  Description \NC\FR
\HL
\NC help \NC	donne la liste des commandes disponibles; \type{ help <commande>} donne le détails sur la commande \NC\NR
\NC open \NC permet de se connecter à un serveur ftp : \type{ open ftpperso.free.fr} \NC\NR
\NC user \NC si la machine ne vous demande pas de taper immediatement un nom d'utilisateur (login) vous pouvez taper la commande : user login puis votre mot de passe ; ou user login motdepasse directement \NC\NR
\NC get \NC	pour récupérer un fichier : \type{get source [destination] }\NC\NR
\NC put \NC 	pour envoyer un fichier vers le serveur ftp : \type{put source [destination]} \NC\NR
\NC ls \NC 	pour lister tous les fichiers du répertoire courant sur le serveur \NC\NR
\NC pwd \NC	Affiche l'emplacement actuel  \NC\NR
\NC cd \NC 	pour de répertoire sur le serveur \NC\NR
\NC cd nom-du-rep \NC	Va dans le répertoire nom-du-rep  \NC\NR
\NC cd .. ou cdup \NC	Remonte d'un répertoire  \NC\NR
\NC !cmd \NC 	pour exécuter une commande sur votre machine en local \NC\NR
\NC lcd \NC 	pour changer de répertoire en local sur votre machine ( equivalent de : !cd ) \NC\NR
\NC binary \NC 	pour passer en mode binaire, indispensable avant d'envoyer ( put) ou de récupérer (get) des images par exemple, vous devez avoir la réponse : 200 Type set to I. \NC\NR
\NC ascii \NC 	pour passer en mode ascii,vous devez avoir la réponse : 200 Type set to A. \NC\NR
\NC prompt \NC 	Pour passer du mode manuel au mode automatique et inversement ( très pratique pour mget et mput ) \NC\NR
\NC mget \NC 	permet de récupérer plusieurs fichiers à la fois. exemples : \type{mget *.htm} , ou \type{mget toto.htm toto1.htm} \NC\NR
\NC mput \NC 	permet de d'envoyer plusieurs fichiers à la fois. exemples : \type{mput *.htm} , ou \type{mput toto.htm toto1.htm} \NC\NR
\NC close \NC 	fermer la connection \NC\LR
\HL
\stoptabulate

\subject{Conclusion}

C'est possible, mais très pénible.



%----------------------------------------------------------------------------------------------------------
\subsection{Client FTP en mode graphique}








%---------------------------------------------------------------------------------------------------------
\section{Utilisation avancée du compte FTP}

la partie la plus intéressante consiste à créer des répertoires protégés sur le site WEB, dont la
consultation ne peut s'effectuer qu'après avoir rempli un tuple login/mot de passe.

Imaginez un répertoire distant, accessible seulement à certains, sur lequel tous ceux qui auront reçu
leur login/mot depasse par email pourront visualiser les plans, dqe et descriptifs, le tout pour 0 \euro \unknown

\subject{Principe}

L'image \about[fig:ftp_rep-prot_AT1]  ci||dessous vous montre l'arborescence d'un site page perso hébergé par FREE. Le répertoire {\tt /dl/cours-0506/{\bf AT1 }} est protégé par mot de passe.

\placefigure
    [here,force,nonumber]    %[-8*line,none]  %[here,force]
    [fig:ftp_rep-prot_AT1]
    {Répertoire distant à protéger}
    {\externalfigure  [ftp-repertoir-prot_show-rep-a-prot.png][width=.3\textwidth]} %[scale=1000]}

\page
Essayez de vous y connecter en cliquant sur ce lien : \from[ftp:o.turlier:AT1]

Les  2 fen\^etres suivantes s'afficheront (Firefox ou Internet Explorer) vous demandant la m\^eme chose : votre {\bf login} et {\bf mot de passe} !

\placefigure
    [here,force,nonumber]  %here,force,
    [fig:ftp_rep-prot_cnxn-fen]
    {C'est le moment de se souvenir de son login {\bf et}  de son mot de passe !}
{\startcombination[2*1]
    {\externalfigure[ftp-repertoir-prot_fenetre-log-mdp_firefox.png][width=.6\textwidth,height=.3\textheight]}{Fen\^etre de connexion du navigateur Firefox}
    {\externalfigure[ftp-repertoir-prot_fenetre-log-mdp_explorer.png][width=.6\textwidth,height=.3\textheight]}{Fen\^etre de connexion du navigateur Internet Explorer}
\stopcombination}


Il vous faudra remplier 2 cases de texte  (login/pass) : essayez
\startitemize[packed]
    \item login \inframed[style=type,corner=round]{teb}
    \item mot de passe  \inframed[style=type,corner=round]{05.06}
\stopitemize

Normalement vous aurez ce résultat, sinon, \unknown

\placefigure
    [here,force,nonumber]  %here,force,
    [fig:ftp_rep-prot_cnxn_reussi-echk]
    {Résultats : ça marche, ou ça merde !}
{\startcombination[2*1]
    {\externalfigure[ftp-repertoir-prot_reussit-cnxn_firefox.png][width=.6\textwidth,height=.3\textheight]}{Réussite : affichage du contenu du répertoire}
    {\externalfigure[ftp-repertoir-prot_echec-cnxn_firefox.png][width=.6\textwidth,height=.3\textheight]}{\'Echec : erreur 401}
\stopcombination}


\subject{Réalisation}

\subsubject{Il faut :}
\startitemize[packed,n]
    \item créer un répertoire caché (nom commençent par un point, i.e : \type{.autorisations}) contenant  :
    \startitemize[packed]
        \item un fichier de mot de passe \type{.htpasswd} et
        \item un fichier interdisant l'accès à ce répertoire à tous les utlisateurs \type{.htaccess}
    \stopitemize
    \item créer un fichier autorisant uniquement l'accès aux utlisateurs listés sur le fichier de mot de passe \type{.htaccess}
    \item le placer dans le répertoire que l'on veut protéger
\stopitemize

\subsubject{Outils nécessaires :}
\startitemize[packed]
    \item éditeur de texte : le \quote{Bloc-notes}\footnote{Le Bloc-note est parfois désigné par son appelation anglaise : Notepad} suffira (démarrer>tous les programmes>accessoires>Bloc-notes)
    \item client FTP : FILEZILLA ou FIREFTP
\stopitemize


\subsubject{Création des protections}

\startitemize[A]
    \head édition du fichier de mots de passe

    à l'aide d'un éditeur de texte, remplir (sans espaces ni lignes vides, sauf commentaires précedés d'un \# ), ligne par ligne, des couples login:mot-de-passe, du type :

    {\externalfigure[htpasswd.png][width=.8\textwidth]}

    et enregistrer ce fichier sous le nom \type{.htpasswd}\footnote[fotnt:renomage]{Avec  le bloc note, il est impossible d'enregistrer un fichier texte portant le nom complet \type{.htpasswd} : on enregistre d'abord le
    fichier en \type{htaccess.txt}, puis on le téléverse sur le site FREE par ftp avec Filezilla, et on le renomme avec Filezilla, en \type{.htpasswd}}, sur votre disque dur (en \quote{local})

    Les utilisateurs suivants seront autorisés : teb,  projet\_immeuble-mrs, projet\_immeuble-lges, DCE\_les-glycines, \unknown\ , en rentrant leur mot de passe !


    \head édition du fichier de protection d'accès au répertoire (caché) contenant le fichier de mot de passe

    à l'aide du Bloc-notes, écrire ceci :

    {\externalfigure[htaccess_htpasswd.png][width=.8\textwidth]}

    et enregistrer ce fichier sous le nom \type{.htaccess}, en \quote{local}


    \head édition d'un second fichier de protection d'accès, destiné au répertoire à protéger

    à l'aide du Bloc-notes, écrire ceci :

    {\externalfigure[htaccess_rep-a-prot.png][width=.8\textwidth]}

    et enregistrer ce fichier sous le nom \type{.htacces}\footnote{Attention : les fichiers de protection on toujours l'appelation \type{.htaccess}}, en \quote{local}, mais dans un endroit différent que le premier \type{.htaccess}, sinon vous \quote{écraseriez} le précedent !

    Le chemin absolu vers le fichier .htpasswd pointe vers : /.autorisations/.htpasswd : il faudra créer le répertoire correspondant (en \quote {distant}, par FTP)
\stopitemize




\subsubject{Mise en place par FTP}

\startitemize[A]
    \head ouvrir FILEZILLA, et connectez-vous sur votre compte FREE

    l'hébergeur FREE propose un nom de serveur de type \type{ftpperso}, à la suite duquel vous rentrez vos informations de login et mot de passe (reçu par courrier)

    {\externalfigure[ftp-put-htpasswd-htaccess_01.png][width=\textwidth]}

    en \quote{local} (panneau de gauche, vers le bas) : on voit les fichiers \type{.htpasswd} et \type{.htaccess} nouvellement crées

    en \quote{distant} (panneau de droite) : on voit le répertoire caché (un point devant le nom)


    \head création du répertoire caché

    clic||droit dans le panneau de droite et sélection \type{Créer un répertoire}

    {\externalfigure[ftp-put-htpasswd-htaccess_02.png][width=\textwidth]}

    le nommer : \inframed[corner=round,style=\tt]{.autorisations} {\bf ne pas oublier le point devant le nom!}

    double||clic dessus pour l'ouvrir

    \head déplacement des fichiers vers ce répertoire (local vers distant = \quote{upload})

    dans le panneau de gauche, clic||droit sur le fichier à \quote{uploader} et sélection \type{Charger sur le serveur}, ou

    \quote{cliquer||glisser} du panneau gauche vers le panneau droit

    faire de m\^eme pour le \type{.htaccess}



    \head renommage des fichiers

    Ces fichiers sont reconnus si ils s'appellent {\bf uniquement} \type{.htaccess} , c.a.d sans l'extension \type{*.txt} rajouté automatiquement par le Bloc||notes.

    clic||droit sur le fichier, sélection \type{renommer} et enlever l'extension de fin (à faire pour tous les \type{.htaccess} et le \type{.htpasswd})


    \head \quote{upload} de la protection d'accès vers le répertoire à protéger (/dl/cours-0506/{\bf AT1 })

    déplacer d'un panneau à l'autre le \type{.htaccess} pour protéger le répertoire dito.


    {\externalfigure[ftp-put-htpasswd-htaccess_03.png][width=\textwidth]}

\stopitemize


Vous \^etes pr\^ets à acceuillir vous utilisateurs d\^uement authentifiés, et rejeter tous les autres curieux!
%===============================================================
\chapter{Utilisation avancée}

%---------------------------------------------------------------------------------------------------------
\section{Accès restreint à un répertoire}

%.............................................................................................................................
\subject{Activation de l'espace WEB + FTP 10 GO}

%.............................................................................................................................
\subject{Iinstallation des logiciels de CMS/BLOG}


%.............................................................................................................................
\subject{Avantages et Limites }

limite bande passante à 50kB/s

logiciels pas toujours en dernière version

configuration limitée (pas d'accès à certains fichiers de conf)

solution:
    hébergement pro : qq offres pas cher
    serveur perso sur ligne adsl ou dedibox

%---------------------------------------------------------------------------------------------------------
\section{Connexion sécurisée}




%===============================================================
\chapter{Exemples}

%---------------------------------------------------------------------------------------------------------
\section{Répertoire partagé avec accès restreint à 5 utilisateurs}



%---------------------------------------------------------------------------------------------------------
\section{Réperoire partagé avec accès restreint à un groupe}

%.............................................................................................................................
\subject{}



%.............................................................................................................................
\subject{}

%.............................................................................................................................
\subject{}




% %===============================================================
% \chapter{Plus loin}
% 
% L'échange de nombreux fichiers entre de multiples utlisateurs peut
% générer nombre de problèmes liés aux versions des fichiers :
% travaille-t-on sur la dernière version ?
% 
% Autocad peut répondre à cette question lorsqu'on utilise le système des XREF, car il détecte un changement au chargement de celles||ci lors de l'ouverture d'un dessin qui en contient.
% 
% \section{Contr\^ole de version}
% 
% avertissemnt : non prévu pour fichiers binaires, mais les métadonnées  restent intéressantes lors de chaque "commit"
% 
% 
% \subsection{Principes}
% 
% \subsection{Application c\^oté serveur}
% 
% subversion
% 
% \subsection{Application c\^oté client}
% 
% win : tortoise svn
% 
% %---------------------------------------------------------------------------------------------------------
% \section{Interface Web}
% 
% Trac
% 
% 
% 
% 
% %---------------------------------------------------------------------------------------------------------
% \section{Utilisation quotidienne de subversion}


%---------------------------------------------------------------------------------------------------------
\section{Exemples}

\subject{Répertoire accès web restreint à un groupe, modifications restreintes à quelques utilisateurs de ce groupe}



%===================================================================
\startbackmatter

\chapter{Annexes}

%---------------------------------------------------------------------------------------------------------
\section{Création compte FREE 10 GO}


Créons un compte gratuit chez le fournisseur
FREE , (email+site web et ftp).

%.........................................................................................................................................
\subsection{Création du compte}

Les captures d'écran suivantes vous indiquent les étapes à suivre. Notez l'URL en haut pour vous rendre à la m\^eme page.

\placefigure
    [here,force,nonumber]  %here,force,
    [fig:capt-creat-cont-free]
    {Cliquez sur les aperçus en réduction pour afficher les images à leur format d'origine. Un nouveau clic sur les grandes
    images vous ramènera ici au point de départ}
{\startcombination[3*3]
    {\gotobox{\externalfigure[abo-free-gratuit_001.png][width=.2\textwidth,height=.2\textheight]}[fig:plan001]}{\'Etape 1}
    {\gotobox{\externalfigure[abo-free-gratuit_002.png][width=.2\textwidth,height=.2\textheight]}[fig:plan002]}{\'Etape 2}
    {\gotobox{\externalfigure[abo-free-gratuit_003.png][width=.2\textwidth,height=.2\textheight]}[fig:plan003]}{\'Etape 3}
    {\gotobox{\externalfigure[abo-free-gratuit_003-b.png][width=.2\textwidth,height=.2\textheight]}[fig:plan004]}{3 bis}
    {\gotobox{\externalfigure[abo-free-gratuit_003-c.png][width=.2\textwidth,height=.2\textheight]}[fig:plan005]}{3 ter}
    {\gotobox{\externalfigure[abo-free-gratuit_003-d.png][width=.2\textwidth,height=.2\textheight]}[fig:plan006]}{3 quart}
    {\gotobox{\externalfigure[abo-free-gratuit_004.png][width=.2\textwidth,height=.2\textheight]}[fig:plan007]}{\'Etape 4}
    {\gotobox{\externalfigure[abo-free-gratuit_005.png][width=.2\textwidth,height=.2\textheight]}[fig:plan008]}{\'Etape 5}
    {\gotobox{\externalfigure[abo-free-gratuit_006.png][width=.2\textwidth,height=.2\textheight]}[fig:plan009]}{\'Etape 6}
\stopcombination}


% Vous pouvez aussi ouvrir un les fichiers originaux en cliquant sur les liens ci||dessous :
% \startbuffer[plans]
% \starttabulate[|c|c|]
% \NC {\inframed[corner=round,framecolor=darkred,rulethickness=3pt,strut=yes]{\goto{\tfb Fichier *dwg}[program(AT5_synth_imm03_v2.dwg)]}}
% \NC {\inframed[corner=round,framecolor=darkred,rulethickness=3pt,strut=yes]{\goto{\tfb Publication *dwf}[program(AT5-synth-imm03.dwf)]}}    \NC\NR
% \stoptabulate
% \stopbuffer

% %\startalign[middle]
% %\midaligned{
% $$
% \cbox{
% \getbuffer[plans]
% }
% $$
% %}
% %\stopalign

\startbuffer[img-grd-format-creation-cpte-free-large]
\placefigure
    [here,force,nonumber]    %[-8*line,none]  %[here,force]
    [fig:plan001]
    {Page d'accueil : cliquez sur {\bf\darkred inscrivez||vous} pour poursuivre}
    {\gotobox
    {\externalfigure  [abo-free-gratuit_001.png][width=\textwidth]} %[scale=1000]}
    [fig:capt-creat-cont-free]}

\placefigure
    [here,force,nonumber]    %[-8*line,none]  %[here,force]
    [fig:plan002]
    {Remplir les renseignements demandés. Les astérisques indiquent qu'il faut obligatoirement répondre à la question}
    {\gotobox
    {\externalfigure  [abo-free-gratuit_002.png][width=\textwidth]} %[scale=1000]}
    [fig:capt-creat-cont-free]}

\placefigure
    [here,force,nonumber]    %[-8*line,none]  %[here,force]
    [fig:plan003]
    {Choix du login : essai 1 . Attention, il faut trouver un nom de login qui ne soit pas trop \quote{farfelu}, car c'est celui que vous donnerez
    à vos clients, associés ou collègues}
    {\gotobox
    {\externalfigure  [abo-free-gratuit_003.png][width=\textwidth]} %[scale=1000]}
    [fig:capt-creat-cont-free]}

\placefigure
    [here,force,nonumber]    %[-8*line,none]  %[here,force]
    [fig:plan004]
    {Choix du login : essai 2. L'essai est infructueux. Si c'est pour une agence, cela pourrait \^etre \type{agence.3a} par exemple}
    {\gotobox
    {\externalfigure  [abo-free-gratuit_003-b.png][width=\textwidth]} %[scale=1000]}
    [fig:capt-creat-cont-free]}

\placefigure
    [here,force,nonumber]    %[-8*line,none]  %[here,force]
    [fig:plan005]
    {Choix du login : essai 3. Encore infructueux : décidemment, il y a beacoup de personnes qui ont déjà choisi ce login !}
    {\gotobox
    {\externalfigure [abo-free-gratuit_003-c.png][width=\textwidth]} %[scale=1000]}
    [fig:capt-creat-cont-free]}

\placefigure
    [here,force,nonumber]    %[-8*line,none]  %[here,force]
    [fig:plan006]
    {login : le graal existe||t||il ?}
    {\gotobox
    {\externalfigure  [abo-free-gratuit_003-d.png][width=\textwidth]} %[scale=1000]}
    [fig:capt-creat-cont-free]}

\placefigure
    [here,force,nonumber]    %[-8*line,none]  %[here,force]
    [fig:plan007]
   {Ca y est, le choix du login \type{olivier.turlier}, tel que
   proposé au départ, est accepté. Si vous voulez absolument un
   kit de connexion (CDROM contenant des pilotes pour différents
   modem analogiques, ainsi qu'un logiciel de configuration pour
   une connexion par modem, et une suite logicielle pour
   travailler sur l'espace perso : édition web, client ftp, etc. (version
   dépassées \unknown)), cliquez sur le bouton idoine, sinon faites comme sur la capture d'écran}
   {\gotobox
    {\externalfigure  [abo-free-gratuit_004.png][width=\textwidth]} %[scale=1000]}
    [fig:capt-creat-cont-free]}

\placefigure
    [here,force,nonumber]    %[-8*line,none]  %[here,force]
    [fig:plan008]
    {Confirmation des informations de connexion}
    {\gotobox
    {\externalfigure  [abo-free-gratuit_005.png][width=\textwidth]} %[scale=1000]}
    [fig:capt-creat-cont-free]}

\placefigure
    [here,force,nonumber]    %[-8*line,none]  %[here,force]
    [fig:plan009]
    {Réussite de l'inscription : attente du courrier}
    {\gotobox
    {\externalfigure  [abo-free-gratuit_006.png][width=\textwidth]} %[scale=1000]}
    [fig:capt-creat-cont-free]}

\page

\stopbuffer

\getbuffer[img-grd-format-creation-cpte-free-large]
\page

%.......................................................................................................................................
\subsection{Configuration du compte FREE}

Le compte que l'on vient de créer n'existe pas encore. Il faut  :
\startitemize[packed]
    \item activer les \quote{pages perso}\footnote{Terminologie FREE}
    \item activer l'espace disque à 10 GO
    \item activer la base de donnée MYSQL
%     \item activer les services web ((blog : dotclear, ...)
\stopitemize

Pour cela :

\startnarrower
\itx
accédez au site FREE : \from[free]

Cliquez sur accès gratuit

Entrez votre identifiant (ex: site web = tebd.free.fr \rightarrow\ identifiant = tebd )

Le formulaire de la page suivante vous demande de rentrer aussi
votre mot de passe (reçu par courrier\footnote{J'espère que vous
avez indiqué une adresse correcte : cette étape est à bien
vérifier lors de la création du compte : voir l'image \about[fig:plan008]})
\stopnarrower

La capture d'écran ci-dessous récapitule en image ce qu'il faut faire .

\placefigure
    [here,force,nonumber]    %[-8*line,none]  %[here,force]
    [fig:interfac-gest-cpte-free-small]
    {Interface de gestion du compte FREE : activations nécessaires \crlf {\tfxx(cliquez sur l'image pour afficher la version grand format --- et vice||versa)}}
    {\gotobox
    {\externalfigure  [interface-gestion-cpte-free_01.png][width=.5\textwidth]} %[scale=1000]}
    [fig:interfac-gest-cpte-free-large]}

\startbuffer[interfac-gest-cpte-free-large]
\placefigure
    [here,force,nonumber] %[here,force] %[here,force,nonumber]
    [fig:interfac-gest-cpte-free-large]
    {Interface de gestion du compte FREE : activations nécessaires}
    {\gotobox
    {\externalfigure  [interface-gestion-cpte-free_01.png][width=\textwidth]} %[scale=1000]}
    [fig:interfac-gest-cpte-free-small]}
\page
\stopbuffer



\getbuffer[interfac-gest-cpte-free-large]
\page



%---------------------------------------------------------------------------------------------------------
\section{FILEZILLA}


Description :


%.............................................................................................................................
\subject{Usage} :
    Téléchargement :  \from[filezilla]
    Installation
    configuration : création d'un compte (accès au compte ftp de free)
        entrez vos informations de connexion : adresse du serveur, nom d'utilisateur et mot de passe; cliquez sur Connexion rapide

\placefigure
    [here,force,nonumber]    %[-8*line,none]  %[here,force]
    [fig:ftp-filezilla-small]
    {écran Filezilla \crlf {\tfxx (cliquer sur l'image pour afficher la version grand format --- et vice||versa)}}
    {\gotobox
    {\externalfigure  [ftp-connect-client-filezilla.png][width=.5\textwidth]} %[scale=1000]}
    [fig:ftp-filezilla-large]}

\startbuffer[fig-flezilla-large]
\placefigure
    [-8*line,nonumber] %[here,force,nonumber]    %[-8*line,none]  %[here,force]
    [fig:ftp-filezilla-large]
    {écran FILEZILLA, connexion établie avec un site FTP distant}
    {\gotobox
    \rotate{\externalfigure  [ftp-connect-client-filezilla.png][height=\textwidth]} %[scale=1000]}
    [fig:ftp-filezilla-small]}
\page
\stopbuffer

\getbuffer[fig-flezilla-large]

D'un coté, à gauche, vous naviguez sur votre disque dur, à droite,
vous avez l'espace disque du serveur web. Vous n'avez plus qu'à
faire glisser les fichiers pour qu'ils soient transférés
automatiquement.

Il existe ensuite de nombreuses possibilités de configuration, de
favoris (pour mémoriser vos serveurs), qui rendent Filezilla une
application ftp incontournable.

% %---------------------------------------------------------------------------------------------------------
% \section{PUTTY}
% 
% 
% %---------------------------------------------------------------------------------------------------------
% \section{Subversion + trac sur ubuntu dédibox}
% 
% \useURL
%     [subversion]
%     [{http://free.fr/}]
% 
% \useURL
%     [trac]
%     [{http://free.fr/}]
% 
% \useURL
%     [install-svn+trac]
%     [{http://free.fr/}]





\stopbackmatter


















\stoptext
