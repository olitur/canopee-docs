% Generated by Sphinx.
\def\sphinxdocclass{report}
\newif\ifsphinxKeepOldNames \sphinxKeepOldNamestrue
\documentclass[a4paper,12pt,french]{sphinxmanual}
\usepackage{iftex}

\ifPDFTeX
  \usepackage[utf8]{inputenc}
\fi
\ifdefined\DeclareUnicodeCharacter
  \DeclareUnicodeCharacter{00A0}{\nobreakspace}
\fi
\usepackage{cmap}
\usepackage[T1]{fontenc}
\usepackage{amsmath,amssymb,amstext}
\usepackage{babel}
\usepackage{mathpazo}
\usepackage[Sonny]{fncychap}
\usepackage{longtable}
\usepackage{sphinx}
\usepackage{multirow}
\usepackage{eqparbox}


\addto\captionsfrench{\renewcommand{\figurename}{Fig.\@ }}
\addto\captionsfrench{\renewcommand{\tablename}{Tableau }}
\SetupFloatingEnvironment{literal-block}{name=Code source }

\addto\extrasfrench{\def\pageautorefname{page}}

\setcounter{tocdepth}{1}
\usepackage{flaskstyle}

\title{canopee-docs Documentation}
\date{11-10-2016}
\release{1.0.1 alpha}
\author{Olivier TURLIER}
\newcommand{\sphinxlogo}{}
\renewcommand{\releasename}{Version}
\makeindex

\makeatletter
\def\PYG@reset{\let\PYG@it=\relax \let\PYG@bf=\relax%
    \let\PYG@ul=\relax \let\PYG@tc=\relax%
    \let\PYG@bc=\relax \let\PYG@ff=\relax}
\def\PYG@tok#1{\csname PYG@tok@#1\endcsname}
\def\PYG@toks#1+{\ifx\relax#1\empty\else%
    \PYG@tok{#1}\expandafter\PYG@toks\fi}
\def\PYG@do#1{\PYG@bc{\PYG@tc{\PYG@ul{%
    \PYG@it{\PYG@bf{\PYG@ff{#1}}}}}}}
\def\PYG#1#2{\PYG@reset\PYG@toks#1+\relax+\PYG@do{#2}}

\expandafter\def\csname PYG@tok@gd\endcsname{\def\PYG@tc##1{\textcolor[rgb]{0.63,0.00,0.00}{##1}}}
\expandafter\def\csname PYG@tok@gu\endcsname{\let\PYG@bf=\textbf\def\PYG@tc##1{\textcolor[rgb]{0.50,0.00,0.50}{##1}}}
\expandafter\def\csname PYG@tok@gt\endcsname{\def\PYG@tc##1{\textcolor[rgb]{0.00,0.27,0.87}{##1}}}
\expandafter\def\csname PYG@tok@gs\endcsname{\let\PYG@bf=\textbf}
\expandafter\def\csname PYG@tok@gr\endcsname{\def\PYG@tc##1{\textcolor[rgb]{1.00,0.00,0.00}{##1}}}
\expandafter\def\csname PYG@tok@cm\endcsname{\let\PYG@it=\textit\def\PYG@tc##1{\textcolor[rgb]{0.25,0.50,0.56}{##1}}}
\expandafter\def\csname PYG@tok@vg\endcsname{\def\PYG@tc##1{\textcolor[rgb]{0.73,0.38,0.84}{##1}}}
\expandafter\def\csname PYG@tok@vi\endcsname{\def\PYG@tc##1{\textcolor[rgb]{0.73,0.38,0.84}{##1}}}
\expandafter\def\csname PYG@tok@mh\endcsname{\def\PYG@tc##1{\textcolor[rgb]{0.13,0.50,0.31}{##1}}}
\expandafter\def\csname PYG@tok@cs\endcsname{\def\PYG@tc##1{\textcolor[rgb]{0.25,0.50,0.56}{##1}}\def\PYG@bc##1{\setlength{\fboxsep}{0pt}\colorbox[rgb]{1.00,0.94,0.94}{\strut ##1}}}
\expandafter\def\csname PYG@tok@ge\endcsname{\let\PYG@it=\textit}
\expandafter\def\csname PYG@tok@vc\endcsname{\def\PYG@tc##1{\textcolor[rgb]{0.73,0.38,0.84}{##1}}}
\expandafter\def\csname PYG@tok@il\endcsname{\def\PYG@tc##1{\textcolor[rgb]{0.13,0.50,0.31}{##1}}}
\expandafter\def\csname PYG@tok@go\endcsname{\def\PYG@tc##1{\textcolor[rgb]{0.20,0.20,0.20}{##1}}}
\expandafter\def\csname PYG@tok@cp\endcsname{\def\PYG@tc##1{\textcolor[rgb]{0.00,0.44,0.13}{##1}}}
\expandafter\def\csname PYG@tok@gi\endcsname{\def\PYG@tc##1{\textcolor[rgb]{0.00,0.63,0.00}{##1}}}
\expandafter\def\csname PYG@tok@gh\endcsname{\let\PYG@bf=\textbf\def\PYG@tc##1{\textcolor[rgb]{0.00,0.00,0.50}{##1}}}
\expandafter\def\csname PYG@tok@ni\endcsname{\let\PYG@bf=\textbf\def\PYG@tc##1{\textcolor[rgb]{0.84,0.33,0.22}{##1}}}
\expandafter\def\csname PYG@tok@nl\endcsname{\let\PYG@bf=\textbf\def\PYG@tc##1{\textcolor[rgb]{0.00,0.13,0.44}{##1}}}
\expandafter\def\csname PYG@tok@nn\endcsname{\let\PYG@bf=\textbf\def\PYG@tc##1{\textcolor[rgb]{0.05,0.52,0.71}{##1}}}
\expandafter\def\csname PYG@tok@no\endcsname{\def\PYG@tc##1{\textcolor[rgb]{0.38,0.68,0.84}{##1}}}
\expandafter\def\csname PYG@tok@na\endcsname{\def\PYG@tc##1{\textcolor[rgb]{0.25,0.44,0.63}{##1}}}
\expandafter\def\csname PYG@tok@nb\endcsname{\def\PYG@tc##1{\textcolor[rgb]{0.00,0.44,0.13}{##1}}}
\expandafter\def\csname PYG@tok@nc\endcsname{\let\PYG@bf=\textbf\def\PYG@tc##1{\textcolor[rgb]{0.05,0.52,0.71}{##1}}}
\expandafter\def\csname PYG@tok@nd\endcsname{\let\PYG@bf=\textbf\def\PYG@tc##1{\textcolor[rgb]{0.33,0.33,0.33}{##1}}}
\expandafter\def\csname PYG@tok@ne\endcsname{\def\PYG@tc##1{\textcolor[rgb]{0.00,0.44,0.13}{##1}}}
\expandafter\def\csname PYG@tok@nf\endcsname{\def\PYG@tc##1{\textcolor[rgb]{0.02,0.16,0.49}{##1}}}
\expandafter\def\csname PYG@tok@si\endcsname{\let\PYG@it=\textit\def\PYG@tc##1{\textcolor[rgb]{0.44,0.63,0.82}{##1}}}
\expandafter\def\csname PYG@tok@s2\endcsname{\def\PYG@tc##1{\textcolor[rgb]{0.25,0.44,0.63}{##1}}}
\expandafter\def\csname PYG@tok@nt\endcsname{\let\PYG@bf=\textbf\def\PYG@tc##1{\textcolor[rgb]{0.02,0.16,0.45}{##1}}}
\expandafter\def\csname PYG@tok@nv\endcsname{\def\PYG@tc##1{\textcolor[rgb]{0.73,0.38,0.84}{##1}}}
\expandafter\def\csname PYG@tok@s1\endcsname{\def\PYG@tc##1{\textcolor[rgb]{0.25,0.44,0.63}{##1}}}
\expandafter\def\csname PYG@tok@ch\endcsname{\let\PYG@it=\textit\def\PYG@tc##1{\textcolor[rgb]{0.25,0.50,0.56}{##1}}}
\expandafter\def\csname PYG@tok@m\endcsname{\def\PYG@tc##1{\textcolor[rgb]{0.13,0.50,0.31}{##1}}}
\expandafter\def\csname PYG@tok@gp\endcsname{\let\PYG@bf=\textbf\def\PYG@tc##1{\textcolor[rgb]{0.78,0.36,0.04}{##1}}}
\expandafter\def\csname PYG@tok@sh\endcsname{\def\PYG@tc##1{\textcolor[rgb]{0.25,0.44,0.63}{##1}}}
\expandafter\def\csname PYG@tok@ow\endcsname{\let\PYG@bf=\textbf\def\PYG@tc##1{\textcolor[rgb]{0.00,0.44,0.13}{##1}}}
\expandafter\def\csname PYG@tok@sx\endcsname{\def\PYG@tc##1{\textcolor[rgb]{0.78,0.36,0.04}{##1}}}
\expandafter\def\csname PYG@tok@bp\endcsname{\def\PYG@tc##1{\textcolor[rgb]{0.00,0.44,0.13}{##1}}}
\expandafter\def\csname PYG@tok@c1\endcsname{\let\PYG@it=\textit\def\PYG@tc##1{\textcolor[rgb]{0.25,0.50,0.56}{##1}}}
\expandafter\def\csname PYG@tok@o\endcsname{\def\PYG@tc##1{\textcolor[rgb]{0.40,0.40,0.40}{##1}}}
\expandafter\def\csname PYG@tok@kc\endcsname{\let\PYG@bf=\textbf\def\PYG@tc##1{\textcolor[rgb]{0.00,0.44,0.13}{##1}}}
\expandafter\def\csname PYG@tok@c\endcsname{\let\PYG@it=\textit\def\PYG@tc##1{\textcolor[rgb]{0.25,0.50,0.56}{##1}}}
\expandafter\def\csname PYG@tok@mf\endcsname{\def\PYG@tc##1{\textcolor[rgb]{0.13,0.50,0.31}{##1}}}
\expandafter\def\csname PYG@tok@err\endcsname{\def\PYG@bc##1{\setlength{\fboxsep}{0pt}\fcolorbox[rgb]{1.00,0.00,0.00}{1,1,1}{\strut ##1}}}
\expandafter\def\csname PYG@tok@mb\endcsname{\def\PYG@tc##1{\textcolor[rgb]{0.13,0.50,0.31}{##1}}}
\expandafter\def\csname PYG@tok@ss\endcsname{\def\PYG@tc##1{\textcolor[rgb]{0.32,0.47,0.09}{##1}}}
\expandafter\def\csname PYG@tok@sr\endcsname{\def\PYG@tc##1{\textcolor[rgb]{0.14,0.33,0.53}{##1}}}
\expandafter\def\csname PYG@tok@mo\endcsname{\def\PYG@tc##1{\textcolor[rgb]{0.13,0.50,0.31}{##1}}}
\expandafter\def\csname PYG@tok@kd\endcsname{\let\PYG@bf=\textbf\def\PYG@tc##1{\textcolor[rgb]{0.00,0.44,0.13}{##1}}}
\expandafter\def\csname PYG@tok@mi\endcsname{\def\PYG@tc##1{\textcolor[rgb]{0.13,0.50,0.31}{##1}}}
\expandafter\def\csname PYG@tok@kn\endcsname{\let\PYG@bf=\textbf\def\PYG@tc##1{\textcolor[rgb]{0.00,0.44,0.13}{##1}}}
\expandafter\def\csname PYG@tok@cpf\endcsname{\let\PYG@it=\textit\def\PYG@tc##1{\textcolor[rgb]{0.25,0.50,0.56}{##1}}}
\expandafter\def\csname PYG@tok@kr\endcsname{\let\PYG@bf=\textbf\def\PYG@tc##1{\textcolor[rgb]{0.00,0.44,0.13}{##1}}}
\expandafter\def\csname PYG@tok@s\endcsname{\def\PYG@tc##1{\textcolor[rgb]{0.25,0.44,0.63}{##1}}}
\expandafter\def\csname PYG@tok@kp\endcsname{\def\PYG@tc##1{\textcolor[rgb]{0.00,0.44,0.13}{##1}}}
\expandafter\def\csname PYG@tok@w\endcsname{\def\PYG@tc##1{\textcolor[rgb]{0.73,0.73,0.73}{##1}}}
\expandafter\def\csname PYG@tok@kt\endcsname{\def\PYG@tc##1{\textcolor[rgb]{0.56,0.13,0.00}{##1}}}
\expandafter\def\csname PYG@tok@sc\endcsname{\def\PYG@tc##1{\textcolor[rgb]{0.25,0.44,0.63}{##1}}}
\expandafter\def\csname PYG@tok@sb\endcsname{\def\PYG@tc##1{\textcolor[rgb]{0.25,0.44,0.63}{##1}}}
\expandafter\def\csname PYG@tok@k\endcsname{\let\PYG@bf=\textbf\def\PYG@tc##1{\textcolor[rgb]{0.00,0.44,0.13}{##1}}}
\expandafter\def\csname PYG@tok@se\endcsname{\let\PYG@bf=\textbf\def\PYG@tc##1{\textcolor[rgb]{0.25,0.44,0.63}{##1}}}
\expandafter\def\csname PYG@tok@sd\endcsname{\let\PYG@it=\textit\def\PYG@tc##1{\textcolor[rgb]{0.25,0.44,0.63}{##1}}}

\def\PYGZbs{\char`\\}
\def\PYGZus{\char`\_}
\def\PYGZob{\char`\{}
\def\PYGZcb{\char`\}}
\def\PYGZca{\char`\^}
\def\PYGZam{\char`\&}
\def\PYGZlt{\char`\<}
\def\PYGZgt{\char`\>}
\def\PYGZsh{\char`\#}
\def\PYGZpc{\char`\%}
\def\PYGZdl{\char`\$}
\def\PYGZhy{\char`\-}
\def\PYGZsq{\char`\'}
\def\PYGZdq{\char`\"}
\def\PYGZti{\char`\~}
% for compatibility with earlier versions
\def\PYGZat{@}
\def\PYGZlb{[}
\def\PYGZrb{]}
\makeatother

\renewcommand\PYGZsq{\textquotesingle}

\begin{document}

\maketitle
\tableofcontents
\phantomsection\label{index::doc}


Ce coin de Web regroupe toutes les documentations qui peuvent être utiles dans les missions quotidiennes d'un technicien d'études bâtiment.

\begin{notice}{note}{Note:}
La documentation entière est aussi disponible sous forme de document \sphinxcode{PDF}
\end{notice}

\noindent\begin{tabular}{|p{0.475\linewidth}|p{0.475\linewidth}|}
\hline

{\hyperref[init_su+acad/index:index\string-init\string-su\string-acad]{\sphinxcrossref{\DUrole{std,std-ref}{Initiation Sketchup et AutoCAD}}}}

Les premiers pas ... et plus encore ...
&
{\hyperref[psd/index:index\string-ptshp]{\sphinxcrossref{\DUrole{std,std-ref}{Photoshop et autres}}}}

Quelques conseils pour enrichir vos dessins
\\
\hline
{\hyperref[su/index:index\string-su]{\sphinxcrossref{\DUrole{std,std-ref}{Sketchup}}}}

Esquisses 3D en 3 clics
&
{\hyperref[ftpwebmail/index:index\string-ftpwebmail]{\sphinxcrossref{\DUrole{std,std-ref}{FTP//WEB//Mail}}}}

Le Web à usage professionnel
\\
\hline
{\hyperref[acad/index:index\string-acad]{\sphinxcrossref{\DUrole{std,std-ref}{AutoCAD}}}}

Aborder le maître en toute sérénité ...
&
{\hyperref[ftpwebmail/index:index\string-ftpwebmail]{\sphinxcrossref{\DUrole{std,std-ref}{Bureautique}}}}

Écrits professionnels
\\
\hline\end{tabular}


Vous y trouverez toutes sortes de documents, sous forme de :
\begin{description}
\item[{\emph{Guides :}}] \leavevmode
En réponse à une problématique concrète, une liste de tâches pas à pas est indiquée. On reste dans le domaine pratique, par exemple : imprimer à l'échelle sur AutoCAD, exporter vers Google-earth depuis Sketchup, etc. Ces documents sont courts.

\item[{\emph{Tutoriels :}}] \leavevmode
Au travers d'un exemple complet à réaliser, les concepts de base et la pratique associée seront traités à chaque étape de la réalisation guidée. C'est le cas de ``initiation Sketchup + AutoCAD''. Ces documents peuvent faire appel à certains guides, ils sont par essence assez longs.

\item[{\emph{FAQ :}}] \leavevmode
Les Foires Aux Questions regroupent les questions les plus fréquemment posées. Vous retrouverez ces documents pour chaque logiciel concerné.

\item[{\emph{Glossaire :}}] \leavevmode
Liste de termes les plus fréquemment employés. Placé côte à côte des FAQ ...

\end{description}

Contents:


\chapter{Initiation Sketchup et Autocad}
\label{init_su+acad/index:initiation-sketchup-et-autocad}\label{init_su+acad/index:index-init-su-acad}\label{init_su+acad/index::doc}\label{init_su+acad/index:documentation-canopee}

\section{Introduction}
\label{init_su+acad/intro:introduction}\label{init_su+acad/intro::doc}
\begin{notice}{note}{Note:}
Cette initiation est composée de plusieurs documents : celui-ci, mais aussi d'autres, tels que {\hyperref[su/config\string-su::doc]{\sphinxcrossref{\DUrole{doc}{Configuration de Sketchup}}}}, ou {\hyperref[acad/config_acad::doc]{\sphinxcrossref{\DUrole{doc}{Configuration d'AutoCAD}}}} . Suivre cette initiation consiste donc à ``naviguer'' entre différents documents qui ne sont \emph{pas tous} écrits dans le contexte d'une initiation.
\end{notice}


\subsection{Principe}
\label{init_su+acad/intro:principe}
Pour aborder l'utilisation de ces logiciels complémentaires, nous allons
réaliser un tâche \emph{concrète :}
\begin{itemize}
\item {} 
esquisse d'une maison \emph{simple} sur \textbf{Sketchup}

\item {} 
export sur \textbf{AutoCAD}

\item {} 
impression au format pdf et sur papier

\end{itemize}

En effectuant ces dessins, nous utiliserons les commandes principales
d'édition, modification, export, etc. Nous serons donc \emph{sensibilisés} à
l'environnement du(des) logiciel(s)


\subsection{Objectif}
\label{init_su+acad/intro:objectif}
Acquérir des compétences initiales sur les logiciels  \emph{Sketchup} et \emph{AutoCAD}.

Par le biais d'une mise en situation immédiate, au travers de l'application d'une méthode pas-à-pas, \textbf{apprendre en faisant bien, la première fois}.

Acquérir une méthode de travail \emph{reproductible}, en maîtrisant le maximum de paramètres (AutoCAD, de par son caractère généraliste, est compliqué à configurer correctement pour le dessin d'architecture), visant l'obtention d'une \emph{qualité régulière}.


\subsection{Le projet}
\label{init_su+acad/intro:le-projet}
La recherche du modèle ``idéal'' à représenter est basée sur des critères multiples :
\begin{itemize}
\item {} 
\textbf{simplicité} : pour apprendre, c'est bien. On modélisera une maison individuelle basique

\item {} 
\textbf{cohérence} : c'est là où la conception architecturale devient raisonnable... (i.e : dimensions entières (cm), ergonomie des pièces, etc.)

\item {} \begin{description}
\item[{\textbf{technicité}}] \leavevmode{[}on en profite toujours pour apprendre quelque chose. Nous nous intéresserons aux réponses apportées à:{]}\begin{itemize}
\item {} 
\emph{la lutte contre le réchauffement climatique} : construction à basse/très basse consommation énergétique

\item {} 
\emph{le développement durable} : emplois de matériaux locaux, ayant nécessité peu d'énergie primaire de fabrication, facilement recyclables, etc.

\item {} 
\emph{la santé} : utilisation de matériaux ne provoquant pas de maladies à court ou long terme, de façon connue ou supposée, de la mise en oeuvre pendant le chantier au recyclage, en passant bien sûr par l'utilisation au quotidien.

\end{itemize}

\end{description}

\item {} 
\textbf{intéressant} : un modèle simple d'accord, mais si je pouvais ajouter mon ``grain de sel'' pour d'améliorer?

\end{itemize}

Il n'a pas été facile de choisir ... tellement l'offre est vaste en matière de constructions modernes, efficaces, belles.

Je me suis tourné vers un modèle assez évolué, proposé par un consortium d'architectes engagés dans une démarche commune promouvant l'habitat de type \href{http://fr.ekopedia.org/PassivHaus}{Passiv Haus} , de la région du \href{http://fr.wikipedia.org/wiki/Voralberg}{Voralberg} , fer de lance du développement ``durable'' en Autriche (quand on pense que 80\% des chaudières sont alimentées par du bois, être à la pointe signifie avoir ... 20 ans d'avance par rapport à la France dans le même domaine!)


\sphinxstrong{Voir aussi:}

\begin{description}
\item[{\url{http://www.fixhaus.at/Berchtold\%20Typ2.pdf}}] \leavevmode
la notice du projet, à télécharger : voir {\hyperref[init_su+acad/demarrage:demarrage\string-init\string-su\string-acad]{\sphinxcrossref{\DUrole{std,std-ref}{Démarrage}}}}

\item[{\url{http://www.berchtoldholzbau.com/pages/d\_pages/seiten/fix\_02.htm}}] \leavevmode
Si vous parler l'Autrichien couramment

\item[{\url{http://translate.google.fr/translate?js=n\&prev=\_t\&hl=fr\&ie=UTF-8\&u=http\%3A\%2F\%2Fwww.berchtoldholzbau.com\%2Fpages\%2Fd\_pages\%2Fseiten\%2Ffix\_02.htm\&sl=de\&tl=fr\&history\_state0=}}] \leavevmode
Version traduite automatiquement par Google ... si vous désirez progressez en Autrichien (le site est composé de beaucoup d'images fixes, non pas de texte)

\item[{\url{http://www.dailymotion.com/video/x24or4\_vorarlberg-une-provocation-construc}}] \leavevmode
voilà comment tout à commencé ...au pays du développement*désirable*

\end{description}




\subsubsection{Aperçu en images}
\label{init_su+acad/intro:apercu-en-images}\begin{quote}

et un de plus!
\end{quote}
\begin{figure}[htbp]
\centering
\capstart

\noindent\sphinxincludegraphics{{fixhaus_berchtold-typ2_pers_01}.png}
\caption{Modèle ``Berchtold type 2'' original}\label{init_su+acad/intro:fig-pers-porkeno}\label{init_su+acad/intro:id1}\end{figure}
\begin{figure}[htbp]
\centering
\capstart

\noindent\sphinxincludegraphics{{fixbercholdtype2}.jpg}
\caption{Variante plus ouverte}\label{init_su+acad/intro:id2}\end{figure}
\begin{figure}[htbp]
\centering
\capstart

\noindent\sphinxincludegraphics{{fixhaus-berchtold2_fac_nord}.png}
\caption{Façade Nord}\label{init_su+acad/intro:id3}\end{figure}
\begin{figure}[htbp]
\centering
\capstart

\noindent\sphinxincludegraphics{{fixhaus-berchtold2_fac_sud}.png}
\caption{Façade Sud}\label{init_su+acad/intro:id4}\end{figure}
\begin{figure}[htbp]
\centering
\capstart

\noindent\sphinxincludegraphics{{fixhaus-berchtold2_fac_ouest}.png}
\caption{Façade Ouest}\label{init_su+acad/intro:id5}\end{figure}
\begin{figure}[htbp]
\centering
\capstart

\noindent\sphinxincludegraphics{{fixhaus-berchtold2_fac_est}.png}
\caption{Façade Est}\label{init_su+acad/intro:id6}\end{figure}
\begin{figure}[htbp]
\centering
\capstart

\noindent\sphinxincludegraphics{{fixhaus-berchtold2_plan_r+0}.png}
\caption{Plan Rez de chaussée}\label{init_su+acad/intro:id7}\end{figure}
\begin{figure}[htbp]
\centering
\capstart

\noindent\sphinxincludegraphics{{fixhaus-berchtold2_plan_r+1}.png}
\caption{Plan Étage}\label{init_su+acad/intro:id8}\end{figure}


\subsubsection{Analyse du modèle}
\label{init_su+acad/intro:analyse-du-modele}\begin{description}
\item[{Type de construction}] \leavevmode
habitation individuelle isolée,
formes simples, sur 2 niveaux.

\item[{Principes ``PassivHaus''}] \leavevmode
isolation thermique renforcée (pas ou peu de ponts thermiques, isolation par l'extérieur, fenêtres triple-vitrage, etc.),
prise en compte globale des phénomènes thermodynamiques en reagard des migrations de vapeur d'eau
grande étanchéité à l'air (jonctions sol/murs, murs/murs, murs/planchers, murs/toit, murs/menuiseries, etc.),
forme massive peu sensible aux fluctuations thermiques
apport solaire passif : maximum de vitrages au Sud et minimum d'ouvertures au Nord
VMC ``double-flux'' (récupération calories sur air vicié extrait)
géothermie : puit canadien (récupération calories/frigories sur air ``souterrain''), réseau fluide caloporteur (horizontal = 2 X surface construite! ou vertical, sur nappe phréatique)
régulation de type ``logique floue'', axée sur la détection de présence (i.e. la VMC se déclenche si il y a du monde, etc.)

\item[{Modularité}] \leavevmode
la conception architecturale inclut la fabrication qui est en général industrielle : il n'y a plus de contrainte météorologique, la qualité d'exécution est augmentée car mieux contrôlable et le coût total est diminué.
Le chantier n'est plus qu'un lieu d'assemblage de panneaux formant des murs.

\item[{Matériaux}] \leavevmode
\emph{bois :} structure, revêtements, etc.

\end{description}


\subsubsection{Travail réel sur ce projet}
\label{init_su+acad/intro:travail-reel-sur-ce-projet}
Il ne reste plus qu'à activer le {\hyperref[init_su+acad/demarrage:demarrage\string-init\string-su\string-acad]{\sphinxcrossref{\DUrole{std,std-ref}{Démarrage}}}} du projet.


\section{Démarrage}
\label{init_su+acad/demarrage:demarrage-init-su-acad}\label{init_su+acad/demarrage::doc}\label{init_su+acad/demarrage:demarrage}
Récupérons les notices en *.pdf sur le site Web pour nous servir de base de travail et à dessinons notre maison sur Sketchup (c'est là que les choses se compliquent!)


\subsection{Récupération des fichiers du projet}
\label{init_su+acad/demarrage:recuperation-des-fichiers-du-projet}
Il faut récupérer les *.pdf du projet, et les dessiner sur sketchup.

Rendez-vous sur : \url{http://www.canopee.org/fichiers/teb-d/aides/acad/init\_su+acad/dessins/pdf/} et récupérez (\sphinxmenuselection{clic-droit \(\rightarrow\) enregistrer sous}) les plans et façades (nom commençant par \sphinxtitleref{fixhaus\_*} ) ainsi que la notice générale du modèle ``Berchtold Type 2'' sur \url{http://www.fixhaus.at/Berchtold\%20Typ2.pdf}

\begin{notice}{note}{Note:}
Vous êtes perdus ? Cliquez sur \textbf{Parent Directory} pour remonter dans l'arborescence des répertoires de cet espace de stockage
\end{notice}

\begin{notice}{warning}{Avertissement:}
L'emplacement des fichiers nécessaires à l'exécution de ce tutoriel est situé sur un serveur distant. Il vaut mieux les télécharger sur votre poste, en local, pour éviter les désagréments consécutifs à une coupure de réseau ...
\end{notice}


\subsection{Organisation préalable des documents}
\label{init_su+acad/demarrage:organisation-prealable-des-documents}
Vous aller créer plusieurs fichiers, certains feront appel à d'autres, etc. Il convient d'adopter une \emph{arborescence type} (dossiers \& fichiers), une \emph{convention de nommage} (fichiers) et de ne plus la changer! Par la suite, vous pourrez appliquer une autre logique, mais suivez celle qui suit.


\subsubsection{Principe}
\label{init_su+acad/demarrage:principe}
Il faut \textbf{organiser} les fichiers au sein de plusieurs dossiers, en appliquant
une stratégie prévoyant l'évolution nécessaire du projet. C'est une habitude à
prendre dès le début d'un nouveau projet, pour en faciliter la gestion.
Typiquement, on doit retrouver, au sein d'un unique répertoire \textbf{Projets} :
\begin{enumerate}
\item {} \begin{description}
\item[{\textbf{Projet 1} (nom du}] \leavevmode{[}projet, client+projet, date+projet, etc.){]}\begin{itemize}
\item {} \begin{description}
\item[{\emph{dessins}}] \leavevmode\begin{itemize}
\item {} 
dwg (tous les dessins au format \sphinxcode{\textbackslash{}*.dwg}  d'AutoCAD) \emph{édition}

\item {} 
img (tous les fichiers ``images'' : \sphinxcode{\textbackslash{}*.jpg}, \sphinxcode{\textbackslash{}*.png}, \sphinxcode{\textbackslash{}*.bmp}, \sphinxcode{\textbackslash{}*.pdf}, etc.)

\item {} 
pdf (tous les fichiers \sphinxcode{\textbackslash{}*.pdf} issus d'un logiciel de dessin (AutoCAD, Sketchup, Archicad, Vectorworks, etc.) \emph{visualisation}

\item {} 
skp (tous les dessins au format \sphinxcode{\textbackslash{}*.skp} de Sketchup)

\item {} 
pln (tous les dessins au format \sphinxcode{\textbackslash{}*.pln} d'Archicad)

\item {} 
\emph{toute autre appellation indiquant clairement le type de fichier ...}

\end{itemize}

\end{description}

\item {} \begin{description}
\item[{\emph{écrits}}] \leavevmode\begin{itemize}
\item {} 
corresp (toutes les lettres, fax, etc. Si abondante, il vaut mieux prévoir un dossier par destinataire ...)

\item {} 
dqe (devis quantitatif estimatif)

\item {} 
plannings

\item {} 
etc.

\end{itemize}

\end{description}

\end{itemize}

\end{description}

\item {} \begin{description}
\item[{\textbf{Projet 2} (projet de plus faible importance)}] \leavevmode\begin{itemize}
\item {} \begin{description}
\item[{\emph{dessins}}] \leavevmode\begin{itemize}
\item {} 
dwg

\item {} 
pdf

\end{itemize}

\end{description}

\item {} \begin{description}
\item[{\emph{ecrits}}] \leavevmode\begin{itemize}
\item {} 
corresp

\end{itemize}

\end{description}

\end{itemize}

\end{description}

\end{enumerate}

\begin{notice}{note}{Note:}
L'organisation est une notion personnelle. L'expérience, la nature du projet ou plus simplement la structure dans laquelle vous travaill(er)ez vous indiquera la marche à suivre ...
\end{notice}


\subsubsection{Organisation des fichiers pour ce tutoriel}
\label{init_su+acad/demarrage:organisation-des-fichiers-pour-ce-tutoriel}
Voici une liste prévoyant les principales étapes :
\begin{enumerate}
\item {} 
Il faut télécharger des *.pdf d'Internet, les transformer dans un format d'image utilisable dans Sketchup

\item {} 
Faire un dessin dans Sketchup

\item {} 
Exporter chaque vue (plans, coupes, façades, toiture) en *.dwg

\item {} 
Créer un fichier AutoCAD regroupant les ``exports''

\item {} 
imprimer ce dessin en .pdf

\item {} 
prévoir des modifications, envois par mail

\end{enumerate}

Le modèle employé, originaire d'Autriche, porte une appellation ``barbare'' :
changeons-la en une appellation plus facile à retenir, qui reprend aussi les
notions d'espoir lié à ce type de construction (vers un monde meilleur ...). On
appellera le projet \textbf{Porkeno}, allez savoir pourquoi ...
\phantomsection\label{init_su+acad/demarrage:arborescence-projet}
Nous aurons donc

\begin{Verbatim}[commandchars=\\\{\}]
\textbar{}
{}`\PYGZhy{}\PYGZhy{} porkeno\PYGZus{}init\PYGZus{}su+acad
   \textbar{}\PYGZhy{}\PYGZhy{} 00\PYGZus{}description.txt
   \textbar{}\PYGZhy{}\PYGZhy{} dessins
   \textbar{}   \textbar{}\PYGZhy{}\PYGZhy{} dwg
   \textbar{}   \textbar{}   \textbar{}\PYGZhy{}\PYGZhy{} porkeno\PYGZus{}plan\PYGZhy{}coupe\PYGZhy{}facad\PYGZus{}o.turlier\PYGZus{}25mai09\PYGZus{}12h00.dwg
   \textbar{}   \textbar{}   {}`\PYGZhy{}\PYGZhy{} xref
   \textbar{}   \textbar{}       \textbar{}\PYGZhy{}\PYGZhy{} porkeno\PYGZus{}plan\PYGZus{}R+0\PYGZus{}export\PYGZhy{}2d\PYGZhy{}su\PYGZus{}o.turlier\PYGZus{}25mai.dwg
   \textbar{}   \textbar{}       \textbar{}\PYGZhy{}\PYGZhy{} porkeno\PYGZus{}plan\PYGZus{}R+1\PYGZus{}export\PYGZhy{}2d\PYGZhy{}su\PYGZus{}o.turlier\PYGZus{}25mai.dwg
   \textbar{}   \textbar{}       \textbar{}\PYGZhy{}\PYGZhy{} porkeno\PYGZus{}plan\PYGZus{}toiture\PYGZus{}export\PYGZhy{}2d\PYGZhy{}su\PYGZus{}o.turlier\PYGZus{}25mai.dwg
   \textbar{}   \textbar{}       \textbar{}\PYGZhy{}\PYGZhy{} porkeno\PYGZus{}coupe\PYGZus{}AA\PYGZus{}export\PYGZhy{}2d\PYGZhy{}su\PYGZus{}o.turlier\PYGZus{}25mai.dwg
   \textbar{}   \textbar{}       \textbar{}\PYGZhy{}\PYGZhy{} porkeno\PYGZus{}coupe\PYGZus{}BB\PYGZus{}export\PYGZhy{}2d\PYGZhy{}su\PYGZus{}o.turlier\PYGZus{}25mai.dwg
   \textbar{}   \textbar{}       \textbar{}\PYGZhy{}\PYGZhy{} porkeno\PYGZus{}facad\PYGZus{}Sud\PYGZus{}export\PYGZhy{}2d\PYGZhy{}su\PYGZus{}o.turlier\PYGZus{}25mai.dwg
   \textbar{}   \textbar{}       \textbar{}\PYGZhy{}\PYGZhy{} porkeno\PYGZus{}facad\PYGZus{}Ouest\PYGZus{}export\PYGZhy{}2d\PYGZhy{}su\PYGZus{}o.turlier\PYGZus{}25mai.dwg
   \textbar{}   \textbar{}       \textbar{}\PYGZhy{}\PYGZhy{} porkeno\PYGZus{}facad\PYGZus{}Est\PYGZus{}export\PYGZhy{}2d\PYGZhy{}su\PYGZus{}o.turlier\PYGZus{}25mai.dwg
   \textbar{}   \textbar{}\PYGZhy{}\PYGZhy{} img
   \textbar{}   \textbar{}   \textbar{}\PYGZhy{}\PYGZhy{} porkeno\PYGZus{}fac\PYGZus{}est.jpg
   \textbar{}   \textbar{}   \textbar{}\PYGZhy{}\PYGZhy{} porkeno\PYGZus{}fac\PYGZus{}nord.jpg
   \textbar{}   \textbar{}   \textbar{}\PYGZhy{}\PYGZhy{} porkeno\PYGZus{}fac\PYGZus{}ouest.jpg
   \textbar{}   \textbar{}   \textbar{}\PYGZhy{}\PYGZhy{} fixhaus\PYGZhy{}berchtold2\PYGZus{}fac\PYGZus{}sud.jpg
   \textbar{}   \textbar{}   \textbar{}\PYGZhy{}\PYGZhy{} fixhaus\PYGZhy{}berchtold2\PYGZus{}perspective.pdf
   \textbar{}   \textbar{}   \textbar{}\PYGZhy{}\PYGZhy{} fixhaus\PYGZhy{}berchtold2\PYGZus{}plan\PYGZus{}r+0.jpg
   \textbar{}   \textbar{}   {}`\PYGZhy{}\PYGZhy{} fixhaus\PYGZhy{}berchtold2\PYGZus{}plan\PYGZus{}r+1.jpg
   \textbar{}   \textbar{}\PYGZhy{}\PYGZhy{} pdf
   \textbar{}   \textbar{}   \textbar{}\PYGZhy{}\PYGZhy{} fixhaus\PYGZhy{}berchtold2\PYGZus{}fac\PYGZus{}est.pdf
   \textbar{}   \textbar{}   \textbar{}\PYGZhy{}\PYGZhy{} fixhaus\PYGZhy{}berchtold2\PYGZus{}fac\PYGZus{}nord.pdf
   \textbar{}   \textbar{}   \textbar{}\PYGZhy{}\PYGZhy{} fixhaus\PYGZhy{}berchtold2\PYGZus{}fac\PYGZus{}ouest.pdf
   \textbar{}   \textbar{}   \textbar{}\PYGZhy{}\PYGZhy{} fixhaus\PYGZhy{}berchtold2\PYGZus{}fac\PYGZus{}ouest\PYGZus{}.pdf
   \textbar{}   \textbar{}   \textbar{}\PYGZhy{}\PYGZhy{} fixhaus\PYGZhy{}berchtold2\PYGZus{}fac\PYGZus{}sud.pdf
   \textbar{}   \textbar{}   \textbar{}\PYGZhy{}\PYGZhy{} fixhaus\PYGZhy{}berchtold2\PYGZus{}plan\PYGZus{}r+0.pdf
   \textbar{}   \textbar{}   \textbar{}\PYGZhy{}\PYGZhy{} fixhaus\PYGZhy{}berchtold2\PYGZus{}plan\PYGZus{}r+1.pdf
   \textbar{}   \textbar{}   \textbar{}\PYGZhy{}\PYGZhy{} porkeno\PYGZus{}plan\PYGZus{}r+0\PYGZus{}A3H\PYGZus{}1\PYGZhy{}50e\PYGZus{}o.turlier\PYGZus{}14mai09\PYGZus{}12h00.pdf
   \textbar{}   \textbar{}   {}`\PYGZhy{}\PYGZhy{} porkeno\PYGZus{}plan\PYGZus{}r+0\PYGZus{}A4H\PYGZus{}1\PYGZhy{}100e\PYGZus{}o.turlier\PYGZus{}14mai09\PYGZus{}12h00.pdf
   \textbar{}   {}`\PYGZhy{}\PYGZhy{} skp
   \textbar{}       \textbar{}\PYGZhy{}\PYGZhy{} porkeno\PYGZus{}o.turlier\PYGZus{}25mai09\PYGZus{}11h00.skp
   {}`\PYGZhy{}\PYGZhy{} ecrits
      \textbar{}\PYGZhy{}\PYGZhy{} porkeno\PYGZus{}dqe\PYGZus{}o.turlier\PYGZus{}19mai09.ods
      {}`\PYGZhy{}\PYGZhy{} porkeno\PYGZus{}lettre\PYGZhy{}accompagnement\PYGZhy{}envoi\PYGZhy{}plans\PYGZhy{}a3\PYGZus{}o.t\PYGZus{}19mai09.odt
\end{Verbatim}

\begin{notice}{note}{Note:}
On notera l'existence d'un sous-répertoire dessins\textgreater{}dwg\textgreater{}xref .

Ce répertoire regroupe les exports 2D Sketchup --\textgreater{} dwg. Il est important de ne pas le déplacer par la suite.

Vous pouvez renommer tous les fichiers de \sphinxcode{fixhaus-berchtold2\_..} à \sphinxcode{porkeno\_...} pour une meilleure ``consistance'' de votre projet.

N'oubliez pas d'inclure les auteurs originaux dans le fichier \sphinxcode{00\_description.txt}
\end{notice}


\subsection{Sketchup}
\label{init_su+acad/demarrage:sketchup}\begin{description}
\item[{Ce logiciel constitue un excellent point d'entrée dans le monde de la DAO.}] \leavevmode\begin{itemize}
\item {} 
Il est très \emph{intuitif} : vous arriverez à produire un dessin en une après-midi.

\item {} 
c'est un logiciel \emph{d'esquisse} (conception globale): cela correspond parfaitement au phasage du déroulement d'un projet de dessin architectural

\item {} 
les possibilités d'export vers d'autres logiciels sont grandes (2D : formats d'image, de dessin(*.dwg); 3D : *.dwg, *.dae (3d importable dans Photoshop notamment), etc.)

\item {} \begin{description}
\item[{Ouverture :}] \leavevmode\begin{itemize}
\item {} 
développé dans un langage orienté ``objet'', facile à employer : \href{http://sketchup.google.com/intl/fr/download/rubyscripts.html}{Ruby},  avec un développement logiciel ouvert à tout le monde \href{http://code.google.com/intl/fr/apis/sketchup/docs/index.html}{API} depuis 2006 : Il y a plusieurs centaines de ``plugins'' (gratuits ou payants) qui améliorent grandement la version de base (ex: voir {\hyperref[su/config\string-su:config\string-su\string-02]{\sphinxcrossref{\DUrole{std,std-ref}{Configuration avancée}}}}).

\item {} 
prix très faible : \textasciitilde{} 600 \texteuro{} pour la version Pro (export vers le *.dwg) et gratuit pour la version grand public.

\end{itemize}

\end{description}

\end{itemize}

\end{description}


\subsubsection{Configuration initiale}
\label{init_su+acad/demarrage:configuration-initiale}
Suive ce guide : {\hyperref[su/config\string-su::doc]{\sphinxcrossref{\DUrole{doc}{Configuration de Sketchup}}}}


\subsubsection{Importation du ``fond de plan''}
\label{init_su+acad/demarrage:init-su-acad-import-img-su}\label{init_su+acad/demarrage:importation-du-fond-de-plan}
Une fois les documents *.pdf du projet sur votre disque dur, suivez : {\hyperref[su/import\string-ssqu::doc]{\sphinxcrossref{\DUrole{doc}{Importation ``fond de plan'' dans Sketchup}}}} pour avancer à l'étape suivante.


\section{Modélisation initiale dans Sketchup}
\label{init_su+acad/su1:modelisation-initiale-dans-su}\label{init_su+acad/su1::doc}\label{init_su+acad/su1:su1}
\begin{notice}{note}{Note:}
Suivez les {\hyperref[su/intro\string-su:notions\string-essentielles\string-su]{\sphinxcrossref{\DUrole{std,std-ref}{Notions Essentielles}}}} avant de vous lancer dans le dessin à tout va!
\end{notice}


\subsection{Murs extérieurs}
\label{init_su+acad/su1:murs-exterieurs}\begin{enumerate}
\item {} 
Faisons un rectangle de 12,63 m X 6,63 m, puis ``évidons-le'' en son centre pour ne laisser qu'une bande de 36 cm (=largeur des murs) :
\begin{itemize}
\item {} 
\textbf{R} (ou clic sur \sphinxincludegraphics{{ic_mesure}.png}) : activation outil \emph{Rectangle}

\item {} 
premier point : clic sur coin gauche supérieur des murs (à l'extérieur)

\item {} 
deuxième point : rentrez les dimensions suivantes dans la ZCV : \sphinxcode{12,63;6,63}

\item {} 
\textbf{Spc} (= barre d'espace ou clic sur \sphinxincludegraphics{{ic_selection}.png}) : activation outil \emph{sélection}

\item {} 
double-clic au centre du rectangle (la surface est sélectionnée ainsi que les bords)

\item {} 
\textbf{F} (ou clic sur \sphinxincludegraphics{{ic_decalage}.png})

\item {} 
sélectionner un bord

\item {} 
\textbf{Tab} (= aller dans la ZCV, ou clic dans la \sphinxincludegraphics{{ZCV}.png} avec le pointeur), et rentrer \sphinxcode{36cm} (attention à ne pas oublier d'inscrire l'unité \sphinxcode{cm} juste après le 36, sinon, ça fera des murs de 36 m!)

\item {} 
\textbf{Spc} , clic au centre du rectangle et suppression de la zone centrale.

\end{itemize}

\end{enumerate}

\begin{notice}{note}{Note:}
On peut indiquer une unité de dimension différente de celle spécifiée dans le modèle de dessin. Il suffit de l'écrire juste après le nombre dans la ZCV
\end{notice}
\begin{enumerate}
\item {} 
Création d'un \textbf{groupe :} cette action permettra d'isoler l'objet du reste du dessin (utile pour le déplacer sans rester ``collé'' aux reste de la géométrie)
\begin{itemize}
\item {} 
double-clic dans la bande restante (la surface \emph{et} tous les bords sont sélectionnés (pointillés + bleu ``gras'')

\item {} 
\sphinxmenuselection{clic-droit \(\rightarrow\) Créer groupe}

\item {} 
édition du groupe : double clic (affichage d'une boîte englobante en pointillé et estompe des autres géométries)

\end{itemize}

\item {} 
``Élévation'' des murs : extrusion avec l'outil ``Push-Pull''
\begin{itemize}
\item {} 
\textbf{P} (ou clic sur \sphinxincludegraphics{{ic_pushpull}.png})

\item {} 
\textbf{Tab} (on est dans la ZCV...) : rentrez \sphinxcode{6} (pour 6 m de hauteur ``estimés'' pour les 2 niveaux)

\end{itemize}

\item {} 
Hauteur exacte des murs extérieurs
\begin{itemize}
\item {} 
\emph{Problème :} sans information plus complète, il est impossible de connaître la hauteur exacte de ces murs!

\item {} \begin{description}
\item[{\emph{Solution :} (identique à {\hyperref[su/import\string-ssqu:import\string-ssqu]{\sphinxcrossref{\DUrole{std,std-ref}{Importation ``fond de plan'' dans Sketchup}}}})}] \leavevmode\begin{itemize}
\item {} 
récupérons une image de la façade Sud \href{http://www.canopee.org/fichiers/teb-d/aides/acad/init\_su+acad/dessins/pdf/fixhaus\_Berchtold-Typ2\_fac-sud.pdf}{ici}

\item {} 
enregistrons-là dans \sphinxcode{../porkeno/dessins/pdf/}

\item {} 
transformons-la en *.png

\item {} 
importons-là dans Sketchup, en la collant contre le mur ``Sud'' (premier coin en bas à gauche, deuxième à l'intersection coin supérieur mur/ bord droit image)

\item {} 
redimensionnons-là (outil mesure --\textgreater{} 1er \& 2ème clic --\textgreater{} nouvelle valeur dans la ZCV, en utilisant la seule dimension connue : \sphinxcode{12,36})

\item {} 
déplaçons l'image pour que les origines coïncident, en utilisant l'outil ``déplacer'' (raccourci : \textbf{M}, ou clic sur \sphinxincludegraphics{{ic_move}.png})

\item {} 
ajustons la hauteur du mur en se basant sur l'image

\item {} 
ce processus est long, pour arriver à l'estimation de départ : \textasciitilde{} 3 m de hauteur!

\end{itemize}

\end{description}

\end{itemize}

\end{enumerate}

\begin{notice}{note}{Note:}
Dimensions du modèle : \textbf{stop aux approximations!} : si il faut recommencer l'étape précédente pour chaque façade, le cloisonnement intérieur, etc. ça risque de devenir fastidieux!

Nous proposons donc de télécharger un pdf mentionnant toutes les cotes principales : \href{http://www.canopee.org/fichiers/teb-d/aides/acad/init\_su+acad/dessins/pdf/porkeno\_plans-simples\_o.turlier\_03juin09\_09h00.pdf}{porkeno\_plan-simple.pdf}
\end{notice}


\subsection{Cloisons}
\label{init_su+acad/su1:cloisons}\label{init_su+acad/su1:porkeno-plans-simples}
En nous basant sur les dimensions du document \href{http://www.canopee.org/init\_su+acad/fichiers/porkeno\_plans-simples\_o.turlier\_03juin09\_09h00.pdf}{porkeno\_plan-simple.pdf}, nous pouvons aisément tracer les cloisons.

Au préalable (c'est une bonne habitude à prendre sur tout logiciel comportant des calques ...), nous allons créer un calque contenant les cloisons.
\begin{description}
\item[{lignes de construction :}] \leavevmode
En nous aidant des dimensions présentes dans le *.pdf, créons des \emph{lignes de construction} (outil mesure \textbf{T}), réprésentant l'implantation au sol des cloisons.

\item[{ligne :}] \leavevmode
Avec l'outil ligne \textbf{L}, tracons le contour au sol des cloisons (sans tenir compte des ouvertures). Sketchup reconnaît un contour, il crée une surface.

\item[{groupe :}] \leavevmode
double/triple cliquons sur cette surface --\textgreater{} clic-droit --\textgreater{} créer un groupe.

\item[{extrusion :}] \leavevmode
activons l'édition de ce nouveau groupe en double-cliquant dessus.
outil ``pousser/tirer'' \textbf{P} : rentrez la valeur de \sphinxcode{2,50} (hauteur d'étage - épaisseur plancher)

\end{description}

\begin{notice}{note}{Note:}
\textbf{dimensions du projet} : au fur et à mesure de l'avancement de l'esquisse, nous prenons des ``distances'' par rapport au modèle originel. De toute façon, on ne dispose que de très peu                 d'informations sur celui-ci ...
\begin{description}
\item[{Afin d'optimiser les dimensions pour arriver à des cotes ``entières'', nous sommes arrivés à :}] \leavevmode\begin{itemize}
\item {} 
dimensions hors-tout : façade = \textbf{12,62 m} (au lieu de 12,63m), pignon = \textbf{6,62 m}

\item {} 
hauteur d'étage : \textbf{2,80 m}

\item {} \begin{description}
\item[{épaisseur de plancher}] \leavevmode{[}\textbf{30 cm}{]}\begin{itemize}
\item {} 
carrelage + colle : 1cm

\item {} 
chappe sèche ``Fermacell'' en 2 couches : 2 cm

\item {} 
isolant thermo-acoustique laine de bois dense : 6 cm

\item {} 
panneau bois contre-collé (type KLH,Leno,MHM,etc. : une épaisseur de 95mm est suffisante, mais ...) : 20 cm

\item {} 
fermacell en sous face : 1 cm (pièces humides)

\end{itemize}

\end{description}

\item {} \begin{description}
\item[{épaisseur murs extérieurs}] \leavevmode{[}\textbf{36 cm}{]}\begin{itemize}
\item {} 
crépi minéral (type monocouche) ou bardage terre cuite ou bardage bois : 1,5 cm

\item {} 
panneau de fibres bois aggloméré au ciment, type fibralith qualité ``extérieur'', heraklith, etc. (pour accroche enduit monocouche et augmentation inertie thermique pour confort d'été)  : ep 3,5 cm

\item {} 
isolant fibre de bois rigide, dens. env. 160 kg/m3, type ``steico therm'' (inertie \& isolation thermique ``confort d'été'') : 6 cm

\item {} 
isolant fibres de bois semi-rigide, dens. env. 50 kg/m3, type ``steico flex'' (isolation thermique) : 10 cm

\item {} 
mur porteur panneaux bois contre-collé (raboté 1 face intérieure) : 14 cm

\item {} 
(éventuellement, en pièces humides notamment) : panneau Fermacell :  + 1 cm

\end{itemize}

\end{description}

\item {} \begin{description}
\item[{épaisseur des cloisons}] \leavevmode{[}\textbf{7,1 cm}{]}\begin{itemize}
\item {} 
panneaux bois contre-collé raboté 2 faces de 71 mm

\item {} 
prévoir habillage plaques de fermacell + faïence en pièces humides : + 1 cm

\end{itemize}

\end{description}

\item {} \begin{description}
\item[{épaisseur de la toiture rampante}] \leavevmode{[}\textbf{50 cm}{]}\begin{itemize}
\item {} 
tuiles + liteaunage : 4 cm

\item {} \begin{description}
\item[{isolant fibre de bois (pose en ``sarking'')}] \leavevmode{[}30 cm{]}\begin{itemize}
\item {} 
isolant/pare-pluie fibre de bois rigide, type ``steico universal'' : 6 cm

\item {} 
isolant fibre de bois semi-rigide, dens. env. 50 kg/m3, type ``steico roof'' : 24 cm

\end{itemize}

\end{description}

\item {} 
panneau bois contre-collé : 16 cm

\item {} 
(éventuellement, selon les restriction imposées par la règlementation incendie)  1 plaque de fermacell : 1 cm

\end{itemize}

\end{description}

\end{itemize}

\end{description}
\end{notice}

Les cloisons sont donc extrudées à une hauteur de 2,50 m, de façon uniforme, \textbf{sans tenir compte des ouvertures}.
\begin{description}
\item[{placement des portes :}] \leavevmode
Affichez la fenêtre ``composants'' : vous trouverez quantité de portes (Sketchup 6 : il faut installer les composants ``Architecture au préalable ...)
Positionnez-les en vous aidant avec des lignes de construction que vous aurez tracé à partir des arêtes des cloisons.
Attention au sens d'ouverture! Si le composant s'ouvre dans le mauvais sens, vous pouvez tenter de le faire tourner avec l'outil déplacement, sinon, essayez l'outil échelle, et au lieu d'étirer l'objet, réduisez sa largeur, puis continuez encore, jusqu'à ``inverser'' celui-ci : vou venez de faire un ``miroir''

\end{description}


\subsection{Plancher}
\label{init_su+acad/su1:plancher}
Sans rentrer dans les détails (différentes couches ...), mettez une ``plaque'' de 30 cm au-dessus des cloisons


\subsection{Export vers Autocad}
\label{init_su+acad/su1:export-vers-autocad}\begin{description}
\item[{Votre modèle 3D est fini? Par \emph{fini}, on entend que les éléments ci-dessous soient présents :}] \leavevmode\begin{itemize}
\item {} 
Plancher

\item {} 
murs extérieurs

\item {} 
cloisons

\item {} 
équipements : sanitaires, cuisine ...

\item {} 
mobilier

\end{itemize}

\end{description}

Les calques doivent organiser l'information selon 2 critères (simultanément) : objet + lieu. Ainsi, on retrouvera un calque pour les murs extérieurs au R+0, mais aussi un calque pour les mêmes murs, mais au R+1. Finalement, la liste des calques doit ressembler à ceci, selon que l'on veuille activer l'affichage de tel ou tel niveau :
\begin{figure}[htbp]
\centering

\noindent\sphinxincludegraphics{{calques-r+0}.png}
\end{figure}
\begin{figure}[htbp]
\centering

\noindent\sphinxincludegraphics{{calques-r+1}.png}
\end{figure}

Pour visualiser l'habitation niveau par niveau, on aurait pu faire plus simple : un calque ``R+0'' regroupant tous les aobjets du niveau éponyme, et un calque ``R+1'' ... L'important est que l'on puisse afficher soit un niveau , soit l'autre.

Pour exporter vers AutoCAD, allons voir une technique Sketchup bien pratique {\hyperref[su/export\string-dwg:export\string-dwg]{\sphinxcrossref{\DUrole{std,std-ref}{Export vers le dwg}}}}


\section{Édition dans AutoCAD}
\label{init_su+acad/acad1:edition-dans-autocad}\label{init_su+acad/acad1::doc}
On vient d'exporter des ``vues'' ou ``coupes'' (revoir la {\hyperref[su/export\string-dwg:export\string-dwg]{\sphinxcrossref{\DUrole{std,std-ref}{méthode}}}} ) 2D du modèle 3D ``Porkeno'' (plans R+0 \& R+1, façades Nord, Sud, Est \& Ouest) au format dwg.

Ces fichiers sont placés dans le dossier \textbf{xrefs} qui est inclus dans le dossier \emph{dwg} du projet. (revoir {\hyperref[init_su+acad/demarrage:arborescence\string-projet]{\sphinxcrossref{\DUrole{std,std-ref}{l'arborescence}}}}  type)


\subsection{Réglage de l'unité d'insertion en mètres}
\label{init_su+acad/acad1:reglage-de-l-unite-d-insertion-en-metres}
Nous avons dessiné en mètres dans Sketchup, mais ce facétieux logiciel exporte des \sphinxcode{*.dwg} avec un format d'insertion en \emph{pouces} !

Utilisons la commande \textbf{\texttt{insunits}} en l'écrivant dans la ``ligne de commande'' d'AutoCAD


\subsubsection{Fichiers exportés par Sketchup}
\label{init_su+acad/acad1:fichiers-exportes-par-su}
Dans le dossier \sphinxcode{xrefs}, ouvrons un fichier exporté par Sketchup, par exemple \sphinxcode{porkeno\_plan-R+0\_export2D-sketchup.dwg} en double-cliquant sur son nom dans l'explorateur windows.

Un fois AutoCAD lancé (le chargement du ruban est accéléré par rapport à AutoCAD 2009, mais ça prend encore du temps, surtout avec des ordinateurs dotés d'une carte graphique ``asthmatique''!), ouvrons les autres fichiers, depuis autocad, par la commande \sphinxmenuselection{fichier \(\rightarrow\) ouvrir} ou le raccourci \sphinxcode{ctrl+o}

Autocad a maintenant 7 fichiers ouverts (plans R+0, R+1 et toiture, façades Nord, Sud, Est et Ouest). On peut passer de l'un à l'autre par le menu \sphinxmenuselection{fenêtres \(\rightarrow\)} (voir la page configuration d'AutoCAD pour afficher ce menu) oupar le raccourci \sphinxcode{ctrl + tab}

Appliquons le réglage sus-mentionné en écrivant la variable suivante \textbf{\texttt{insunits}} et en réglant sa valeur (20 possiblités !) à \textbf{6}. Cette commande est le raccourci correspondant au menu \sphinxmenuselection{A \(\rightarrow\) utilitaires de dessin \(\rightarrow\) unités} qui provoque l'affichage de la fenêtre \emph{Unités} (affichable directement avec la commande ``texte'' éponyme), sur laquelle il faut indiquer \emph{mètre} en ce qui concerne les unités d'insertion ... Avouez que la ligne de commande est franchement plus rapide dans certains cas!

Il efectuer ce réglage avec \textbf{\texttt{insunits : 6}} à tous les fichiers \sphinxcode{*.dwg}  créés par Sketchup, les enregistrer et les fermer.


\subsubsection{Nouveau fichier ``Hôte'' des exports de Sketchup}
\label{init_su+acad/acad1:nouveau-fichier-hote-des-exports-de-su}
Toujours dans AutoCAD, ouvrons un nouveau fichier par la commande \sphinxmenuselection{fichier \(\rightarrow\) nouveau} ou le raccourci \sphinxcode{ctrl + n}.En configuration par défaut (c.a.d que AutoCAD ouvre le gabarit \sphinxcode{acadiso.dwt}), il faut changer \emph{aussi} l'unité d'insertion qui est par défaut en mm (insunits : 4). Cette étape ne s'applique pas si vous avez configuré AutoCAD selon :ref{}`config-acad{}` !

Réglons l'unité d'insertion en mètres, avec la commande \textbf{\texttt{insunits : 6}}

Enregistrons ce fichier dans le dossier \sphinxcode{dwg} sous le nom \sphinxcode{porkeno\_plan+coupe+facades\_\textless{}auteur\textgreater{}\_\textless{}date\textgreater{}.dwg}. Réglez les variables ``auteur'' et ``date'' à votre convenance. Le changement de nom de ce fichier n'aura pas d'influence sur l'insertion en \emph{xref} des fichiers exportés par Sketchup


\subsection{Import en xref}
\label{init_su+acad/acad1:import-en-xref}
Le fichier \sphinxcode{porkeno\_plan+coupe+facades\_\textless{}auteur\textgreater{}\_\textless{}date\textgreater{}.dwg}  est vide.

Nous allons y ``copier'' les fichiers créés par Sketchup de façon \emph{bidirectionnelle} : si l'on modifie le fichier ``invité'' , le fichier ``hôte'' le sera aussi.

Cette façon d'insérer un(des) (ou plusieurs dans notre cas ...) dessin(s) ``invités'' dans un autre ``hôte'' permet de reporter des modifications \emph{automatiquement}. Il est ainsi facile de modifier l'esquisse originale dans Sketchup, de ré-exporter les vues axono dans les mêmes fichiers \sphinxcode{dwg} (en gardant \textbf{absolument} le même nom, donc en écrasant les anciennes versions) : le fichier ``hôte'' en sera ``automagiquement'' modifié!


\subsubsection{Insertion de la première xref}
\label{init_su+acad/acad1:insertion-de-la-premiere-xref}\begin{enumerate}
\item {} 
clic sur insertion --\textgreater{} réference externe

\item {} 
choisissez, dans le sous-dossier \sphinxcode{xref},  le fichier \sphinxcode{porkeno\_plan-R+0\_export2D-sketchup.dwg}

\item {} 
type de chemin \textbf{relatif} (possible puisque le nouveau dessin ``hôte'' est \textbf{enregistré}) : cela permet d'enregistrer l'arborescence complète du projet sur un autre emplacement (2ème ordinateur, clé usb, etc.) sans perdre les xrefs (le chemin relatif enregistre uniquement les informations de positionnement de la xref, relativement au dessin ``hôte'', comme \sphinxcode{../porkeno\_plan+coupe+facade\_o.turlier\_14juin-15h20.dwg}, et pas jusquà la racine complète, i.e. \sphinxcode{D:/Mesdoc/olivier/projets/porkeno/dessins/dwg/porkeno\_plan+coupe+facade\_o.turlier\_14juin-15h20.dwg} )

\item {} 
La seule ``coche'' à laisser est ``spécifiez le point d'insertion à l'écran''

\end{enumerate}


\subsubsection{Insertion des suivantes}
\label{init_su+acad/acad1:insertion-des-suivantes}\begin{enumerate}
\item {} 
une fois le R+0 inséré, faites de même pour le R+1 (à positionner au dessus, à la verticale),

\item {} 
puis la toiture,

\item {} 
et les façades, à disposer sur les cotés des plans correspondants (vue des projections à l'américaine).

\end{enumerate}


\subsubsection{Copie des xrefs}
\label{init_su+acad/acad1:copie-des-xrefs}
\noindent\sphinxincludegraphics{{orientation_xref}.png}
\begin{enumerate}
\item {} 
pour plus de clarté dans la présentation, n'hésitez pas à faire des copies des xrefs avec la commande \textbf{\texttt{copie}}, pour disposer les ouvertures en regard : en traçant des lignes de brouillon des ouvertures de la façade Est, on doit ``tomber'' exactement sur les ouvertures en plan (rez de chaussée ou autre niveau ...). De même, il est préférable d'aligner les xrefs en vertical comme ne horizontal, grâce à des lignes de repère ``brouillon''.

\item {} 
il est bon, toujours dans les copies, de disposer aussi les façades de façon alignée, sur la même ligne horizontale, idéalement le niveau +- 0,00 archi.

\end{enumerate}


\subsection{Création des calques nécessaires}
\label{init_su+acad/acad1:creation-des-calques-necessaires}
En partant du principe que vous utilisez un AutoCAD ``standard'', qui n'a pas utilisé un fichier de gabarit ``personnalisé'', il faut au moins créer quelques calques permettant de travailler de façon ordonnée :

\noindent\begin{tabulary}{\linewidth}{|L|L|L|L|}
\hline

\textbf{Calque}
&
\textbf{couleur}
&
\textbf{epaisseur ligne}
&
\textbf{impression}
\\
\hline
\textbf{brouillon}
&
magenta
&
default
&
\textbf{non}
\\
\hline
\textbf{trait\_fin}
&
jaune
&
0,25
&
oui
\\
\hline
\textbf{xrefs}
&
gris
&
defaut
&
oui
\\
\hline\end{tabulary}


Sélectionnez toutes les ``xrefs'' (fichiers \sphinxcode{dwg} créés par Sketchup) en cliquant une fois dessus et sélectionnez le calque ``xrefs'' dans la liste déroulante.


\subsection{Traçage des lignes de construction}
\label{init_su+acad/acad1:tracage-des-lignes-de-construction}
Activez le mode ``ortho'' (trait uniquement horizontaux ou verticaux) en appuyant sur la touche \sphinxcode{F8} (vous pouvez lire dans la ligne de commande : \sphinxcode{\textless{}ortho actif\textgreater{}})

Activez le calque ``brouillon'' et commencez à tracer un trait vertical sur l'arête extérieure gauche du plan R+0 importé en xref (mur extérieur).

Décaler ce trait vers la droite d'une valeur de 36 cm en rentrant la valeur \textbf{\texttt{0.36}}.

Faites de même pour toutes les arêtes déterminant un mur, une cloison, etc., en vous aidant du plan fourni à cette page {\hyperref[init_su+acad/su1:porkeno\string-plans\string-simples]{\sphinxcrossref{\DUrole{std,std-ref}{Cloisons}}}}, ou pour plus de facilité :  \sphinxcode{là}

L'important est de ne pas s'accrocher sur la géométrie existante des ``xrefs'' créées par Sketchup et importées dans ce dessin, mais de se servir des cotes indiquées par le plan \sphinxcode{*.pdf}. Les xrefs servent uniquement de \emph{fond de plan} : on ne peut se fier à Sketchup pour créer une géométrie exacte !

Une fois que les arêtes verticales sont tracées par décalage, faites de même pour les horizontales.


\subsection{Traçage des contours du plan}
\label{init_su+acad/acad1:tracage-des-contours-du-plan}

\subsubsection{Traçage initial ``au kilomètre'' (édition)}
\label{init_su+acad/acad1:tracage-initial-au-kilometre-edition}
\begin{notice}{note}{Note:}
Première approche d'une méthode de traçage ``rigoureuse'' (dimensions exactes): ne l'appliquez que sur le plan R+0. Les étapes suivantes seront plus ``réfléchies'', i.e : les arêtes seront dessinées à leur bonne épaisseur ``à l'avancement''
\end{notice}
\begin{enumerate}
\item {} 
Activez le calque \sphinxcode{trait\_fin}

\item {} 
Estompez la visibilité du calque \sphinxcode{xrefs} (60\% maximum)

\item {} 
Dessinez des lignes allant d'une intersection de ligne de construction à l'autre, en vous aidant avec les xrefs.

\end{enumerate}

\begin{notice}{note}{Note:}
Nouveauté AutoCAD 2010 : l'inportation d'un calque sous-jacent en \sphinxcode{*.pdf} est possible, et \textbf{recommandée}. Si le pdf a été bien dessiné, les traits de celui-ci sont ``accrochables'' : on peut dessiner une géométrie en se calant directement sur le \sphinxcode{*.pdf} sous-jacent.

Hélas, si vous désirez exporter votre dessin dans une version antérieure à AutoCAD 2010, vous ne pourrez plus apercevoir le \sphinxcode{*.pdf}, car c'est une fonctionalité \emph{réservée} à la version d'AutoCAD 2010
\end{notice}
\begin{enumerate}
\item {} 
Lorsque vous avez tracé tous les contours, vous pouvez masquer(= ``geler'') le calque ``brouillon''

\end{enumerate}

On s'aperçoit que les épaisseurs de lignes ne correspondent pas aux conventions architecturales : les arêtes coupées sont en trait fin, alors qu'elle devraient être en trait fort


\subsubsection{Changement épaisseur des lignes (modification)}
\label{init_su+acad/acad1:changement-epaisseur-des-lignes-modification}
Les ligne de traits fin correspondant aux arêtes vues doivent être interrompues aux intersections avec les arêtes coupées qui doivent être en traits forts.

Il y a 2 solutions :
\begin{enumerate}
\item {} 
l'une - hélas - trop souvent utilisée (elle ``pollue'' le dessin par la superposition de lignes) consiste à \textbf{redessiner} en traits forts sur les traits fins (pour nettoyer les lignes dupliquées, heureusement qu'il y a un outil dans ``Express Tools'' (en anglais uniquement!) : \sphinxmenuselection{Express Tools \(\rightarrow\) find duplicates})

\item {} 
l'autre plus élégante, consiste à couper les lignes aux intersections où elles doivent changer d'épaisseur, en utisant l'outil \textbf{\texttt{couper au point}}. C'est efficace, (sélectionnez la ligne à couper, puis l'intersection où elle doit être coupée), mais long! Il faut répeter cette opération à chaque point (pas d'option ``multiple'' à ma connaissance), et en plus, je ne sais pas pourquoi, AutoCAD se met souvent à couper non plus au point, mais aussi un petit segment : il faut alors rallonger la ligne coupée! Une fois coupée, il faut changer le calque de la la ligne en la sélectionnant, puis en cliquant sur le calque désiré (\sphinxcode{trait\_forts} dans notre cas)

\end{enumerate}

\begin{notice}{warning}{Avertissement:}
Les propriétés des entités de dessin dépendent des calques auxquels elles appartiennent! i.e. une ligne est épaisse, de couleur rouge, etc. parce que le calque sur lequel elle est tracée a comme propriété : couleur=rouge, épaiseur ligne=0.70, etc.
\begin{quote}


\end{quote}

Dans la fenêtre propriétés, on doit toujours voir ``Du Calque/Du Calque/Du Calque'' pour les paramètres couleur/épaisseur/style de ligne

Par pitié, n'ayez pas l'idés saugrenue de modifier les propriétés d'un ligne de façon ``individuelle'' , de façon différente de son calque! Vos modifications ultérieures risquent de devenir un cauchemar!
\end{notice}

Pour rallonger un ensemble de lignes, \emph{de façon rigoureuse...}, il faut utiliser l'outil \textbf{\texttt{prolonger}}. Celui-ci ``demande'' (lisez, lisez \& re-lisez la ligne de commande ...) de sélectionnez une ligne faisant office de repère de prolongation ou tout : choisissez cette option en validant (le \sphinxcode{clic-droit} est votre ami le plus rapide pour la validation ...). Donc, en résumé, \sphinxmenuselection{prolonger \(\rightarrow\) clic-droit \(\rightarrow\) selection des lignes à prolonger}

\begin{notice}{note}{Note:}
Les méthodes de sélection sont nombreuses dans AutoCAD et il est utile de les connaître.

De nombreuses commandes, généralement du groupe \emph{modifications} (supprimer, déplacer, rotation, copie, etc.) demandent de sélectionner des objets, lignes, etc. et l'on peut :
* sélectionner les entités (objets, lignes) une à une, en cliquant dessus (ajout automatique)
* sélectionner un groupe d'entités:
\begin{itemize}
\item {} 
en traçant une \textbf{fenêtre} de capture (forme rectangulaire) \emph{partielle} (pas besoin d'inclure les entités en entier) : 1er clic coin droit haut, puis ``glisser'' vers le coin bas gauche

\item {} 
en traçant un \textbf{chemin} de capture : lorsque qu'une commande vous ``demande'' de sélectionner, cliquez sur \textbf{\texttt{T}}, et tracez un ``trajet'' de sélection (en cas de traçage au sein d'une géométrie complexe, il peut être utile de désactiver l'accrochage objet en cliquant sur la touche \sphinxcode{F3}, et attention, l'option \textbf{\texttt{trajet}} ne marche que pour la portion visible à l'écran : vous ne pouvez pas tracer un trait d'un bord à l'autre, et le continuer ne déplaçant les objets)

\item {} 
en traçant un \textbf{polygone} de capture (forme angulaire quelquonque) : à l'invite de sélection, tapez \textbf{\texttt{CP}} et tracez un polygone incluant (partiellement) les entités à sélectionner. Attention, la limitation d'effet de la commande à la portion visible de l'écran s'applique aussi ...

\end{itemize}
\end{notice}

Vous comprenez qu'il est fastidieux de modifier après coup un grand nombre de lignes : il est préférable de répérer à l'avance quelle épaissuer doit avoir le trait à tracer, et de semettre dans le calque idoine. Cela s'appelle de la préparation ... La fable de Jean de la Fontaine \href{http://www.leplaisirdapprendre.com/Le-lievre-et-la-tortue.html}{le lièvre et la tortue} nous en rappelle les vertus...

La suite est simple : une fois les contours tracés (à leur bonne épaisseur, que l'on peut visualiser en cliquant sur le bouton du bas \textbf{\texttt{el}}), on passe à la


\subsection{Cotation}
\label{init_su+acad/acad1:cotation}
Cela nécessite de configurer au préalable (si ce n'est pas déjà fait) :
\begin{itemize}
\item {} 
les styles de texte (on créera des styles ayant des propriétés ``annotatives'')

\item {} 
les styles de cote (appelant les styles de texte définis auparavant)

\item {} 
les échelles / échelles d'annotation (création pour le dessin en mètre et pour le dessin en centimètre et suppression des échelles impériales)

\end{itemize}

puis à la


\subsection{Mise en page}
\label{init_su+acad/acad1:mise-en-page}
dans les présentations (= espace papier) : cela implique de créer/configurer au préalable (si ce n'est pas déjà fait) :
\begin{itemize}
\item {} 
un cartouche

\item {} 
une imprimante ``système'' qui exporte le \sphinxcode{*.dwg} en \sphinxcode{*.pdf} : celle fournie par AutoCAD 2010 \textbf{DWG To PDF} est excellente!

\item {} 
une mise en page nommée, appelant l'imprimante ci-dessus, dans des formats de papier compatible avec notre matériel : du A3 paysage par exemple

\end{itemize}

enfin à l'


\subsection{Impression en pdf}
\label{init_su+acad/acad1:impression-en-pdf}
et à


\subsection{Imprimer du pdf sur du papier}
\label{init_su+acad/acad1:imprimer-du-pdf-sur-du-papier}
Voilà!

Il faut donc se tourner vers la page {\hyperref[acad/config_acad:config\string-acad]{\sphinxcrossref{\DUrole{std,std-ref}{Configuration d'AutoCAD}}}} pour arriver à mener à terminer ces étapes dans un délai raisonnable, mais aussi pour avoir un rendu similaire ...


\section{Modification avec Sketchup}
\label{init_su+acad/su2:modification-avec-sketchup}\label{init_su+acad/su2::doc}

\section{Suivi des modifications dans AutoCAD}
\label{init_su+acad/acad2::doc}\label{init_su+acad/acad2:suivi-des-modifications-dans-autocad}

\chapter{Sketchup}
\label{su/index:sketchup}\label{su/index::doc}\label{su/index:index-su}

\section{Présentation de Sketchup}
\label{su/intro-su:presentation-de-sketchup}\label{su/intro-su::doc}

\subsection{Notions Essentielles}
\label{su/intro-su:notions-essentielles}\label{su/intro-su:notions-essentielles-su}

\subsubsection{Navigation}
\label{su/intro-su:navigation}
Un clic tenu enfoncé sur la molette change le point de vue.

Un appui simultané sur la touche majuscule et le clic maintenu enfoncé de la molette permet de faire une translation de la vue.


\subsubsection{Repères de dessin}
\label{su/intro-su:reperes-de-dessin}\begin{enumerate}
\item {} 
repères ``machine''

\end{enumerate}

Sketchup Vous propose toujours d'aligner votre dessins selon l'un des trois axes (x,y,z), en affichant des pointillés (rouge,vert,bleu)
\begin{enumerate}
\setcounter{enumi}{1}
\item {} 
repère utilisateur : \emph{trait de construction}

\end{enumerate}

Il est essentiel d'utiliser les traits de construction pour obtenir des tracés précis.

Utiliser l'outil mètre pour tracer des traits parallèles aux axes rouge, vert et bleu. Pour tracer un trait à une distance précise, entrez un chiffre au clavier dans la ZCV.
\begin{description}
\item[{Procédure :}] \leavevmode
avec l'outil mètre sélectionné (raccourci : \textbf{T}),
cliquez sur un trait existant (ou une arête),
maintenez le clic enfoncé et entrez une valeur numérique

\end{description}
\begin{enumerate}
\setcounter{enumi}{2}
\item {} 
repère utilisateur : \emph{lignes d'inférence}

\end{enumerate}

Les lignes d'inférences permettent de trouver un point d'accrochage sur une géométrie distante
\begin{description}
\item[{Procédure :}] \leavevmode
avec un outil de dessin (ligne, cercle, rectangle, etc.) ou de modification (déplacement, rotation, échelle, décalage, etc.),
appuyez sur la touche {[}\textbf{Maj}{]}
vous apercevrez des lignes pointillées vous indiquant l'accroche possible vers une extrémité voisine

\end{description}


\subsubsection{Groupes}
\label{su/intro-su:groupes}
Avec Sketchup, vous \emph{devez} créer des groupes, \emph{dès que vous avez terminé le dessin d'une forme simple} (=surface).
Cliquez 3 fois sur un objet pour le sélectionner (surface + bords)
Faites un clic droit et choisissez : créer un groupe

Cette opération est préalable à l'extrusion (voir ci-dessous). Elle permet d'isoler chaque entité de dessin par rapport aux autres. Ceci fait, vous pourrez déplacer une table sans entraîner le tapis, tout simplement parce qu'ils sont dans des groupes séparés!

Pour pouvoir éditer un groupe, il faut double-cliquer dessus : on aperçoit une boîte englobante en pointillés noir, et les autres géométries/objets apparaissent ``estompés'' : cela signifie que l'on édite uniquement le groupe sélectionné.


\subsubsection{Pousser / tirer}
\label{su/intro-su:pousser-tirer}
Raccourci : \textbf{P}

Quand une face est tracée, vous pouvez l'extruder avec l'outil Pousser/tirer. Pour une plus grande précision, entrez une valeur numérique au clavier.


\subsection{Les bases élémentaires}
\label{su/intro-su:les-bases-elementaires}
Travailler dans l'espace peut se résumer à un certain nombre d'opérations simples qui, combinées entre elles, vous permettent de représenter à peu près n'importe quoi, de la cabane de jardin à la structure de l'Empire State building...
\begin{description}
\item[{Naviguer :}] \leavevmode
Travailler dans un espace en 3D requiert tout d'abord d'être en mesure de pouvoir s'y déplacer.

Dans Sketchup, cette fonction se fait principalement avec la molette de la souris, qui permet de zoomer, ce qui est déjà un bon début, mais aussi d' ``orbiter '' (terme officiel ...) autour du modèle. La combinaison de ces deux fonctions permet de faire à peu près ce que l'on veut, mais il est parfois nécessaire d'utiliser également la fonction ``Look around '' (l'icône avec deux yeux de cocker triste) pour tourner (virtuellement...) sur soi-même (fort pratique à l'intérieur d'un espace clos).

\item[{Créer :}] \leavevmode
La mise en place d'une 3D va d'abord passer par le dessin d'éléments de base. L'outil crayon permet de créer des segments, définis par un point d'origine et un point d'arrivée, ou au moins une direction et une distance.

Lorsque trois lignes, au moins, forment une surface fermée, une face apparaît. En utilisant l'outil ``pousser-tirer '' sur celle-ci, on génère une volumétrie.

Cette simple opération est la base de tout ce que l'on peut faire avec ce programme. Il suffit ensuite de combiner les différentes fonctions pour créer n'importe quelle géométrie. Par exemple, redessiner une surface fermée sur une boîte et l'extruder au travers de ce dernier donnera naissance à une ouverture, tout simplement.

La création du dessin d'origine s'enrichit ensuite de l'usage d'outils annexes, dont les applications sont assez facile à cerner : courbe, cercles, rectangle, polygones...

\item[{Sélectionner :}] \leavevmode
Une fois que l'on a créé quelque chose dans l'espace, il devient nécessaire de pouvoir s'en saisir, afin de l'exploiter un minimum. Dans Sketchup, cliquer sur un élément (arrête, face...) avec l'outil ``flèche '' le sélectionne, logiquement. Ensuite, cette sélection s'étend en fonction du nombre de clics : double clic, l'élément et son entourage direct sont sélectionnés ; triple clic, tous les éléments liés sont sélectionnés.

Le système repose aussi sur le dessin à l'écran d'une zone de sélection, prenant en compte une direction : de gauche à droite, seuls les éléments entièrement inclus seront sélectionnés ; de droite à gauche, tout ce qui est touché sera sélectionné.

\item[{Modifier :}] \leavevmode
Le tout n'est pas d'avoir un ensemble de géométrie, si l'on ne peut rien en faire... Il existe donc plusieurs outils permettant de manipuler les différents éléments 3D.

\end{description}


\section{Configuration de Sketchup}
\label{su/config-su:config-su}\label{su/config-su::doc}\label{su/config-su:configuration-de-sketchup}

\subsection{Configuration Initiale}
\label{su/config-su:configuration-initiale}

\subsubsection{Interface graphique}
\label{su/config-su:interface-graphique}
Après l'installation, Sketchup nous offre une interface assez pauvre :
\begin{figure}[htbp]
\centering
\capstart

\noindent\sphinxincludegraphics{{su_001}.png}
\caption{\textbf{Fig} Sketchup en configuration initiale}\label{su/config-su:id1}\end{figure}

Un petit coup de \sphinxmenuselection{Affichage \(\rightarrow\) barre d'outils}, qui doit ressembler à ça :
\begin{figure}[htbp]
\centering
\capstart

\noindent\sphinxincludegraphics{{su_002}.png}
\caption{\textbf{Fig} Cochez toutes les barres d'outils}\label{su/config-su:id2}\end{figure}

Nous proposera tous les outils basiques nécessaires :
\begin{figure}[htbp]
\centering
\capstart

\noindent\sphinxincludegraphics{{su_003}.png}
\caption{\textbf{Fig} Sketchup configuré correctement}\label{su/config-su:id3}\end{figure}


\subsubsection{Spécifications des unités de travail, etc.}
\label{su/config-su:specifications-des-unites-de-travail-etc}
Par le menu \sphinxmenuselection{Fenêtre \(\rightarrow\) information du modèle}
\begin{itemize}
\item {} 
\textbf{Emplacement géographique :}

\end{itemize}
\begin{figure}[htbp]
\centering
\capstart

\noindent\sphinxincludegraphics{{su_004}.png}
\caption{\textbf{Fig} emplacement géographique}\label{su/config-su:id4}\end{figure}
\begin{itemize}
\item {} 
\textbf{Unités :} \emph{de longueur :} format décimal : \textbf{mètres}, précision : 0,00,  \emph{angulaire :} précision : 0,00

\item {} 
\textbf{Cotation :} aligner texte sur ligne de cotation, \emph{au-dessus}

\end{itemize}

\begin{notice}{note}{Note:}
Pour un dessin d'architecture concernant un projet neuf, il est préférable de travailler en \textbf{mètres}.
Certains préfèrent le \textbf{cm}, ils ont raison tant que l'on reste dans des dimensions petites (maison individuelles). Dessiner un immeuble mettra fin à leur préférences!

Le dessin en \textbf{cm} est enviseageable pour les dessins de mise au net de relevé de bâtiment existants.
\end{notice}


\subsection{Configuration avancée}
\label{su/config-su:configuration-avancee}\label{su/config-su:config-su-02}

\subsubsection{Plugins}
\label{su/config-su:plugins}
Il y a plein de plugins pour Sketchup qui permettent un travail plus aisé.

Nous allons Configurer SU pour utiliser les (excellentes) barres d'outils SCF, téléchargeables sur ce site. Pour cela, il faudra remplacer les barres d'outils originelles


\sphinxstrong{Voir aussi:}

\begin{description}
\item[{\url{http://www.special.eclipse.co.uk/\_ruby.html}}] \leavevmode
SCF toolbars : powerbar, selector, drawtools, standard tools et archiland toolbar

\item[{\url{http://www.casttv.com/video/xqa81g1/scf-toolbar-sketchy-bevel-video}}] \leavevmode
Vidéo : \emph{Sketchy bevel} (une application de la barre d'outil SCF ``powerbar'')

\item[{\url{http://www.casttv.com/video/5lrv85/scf-toolbar-protrude-video}}] \leavevmode
Vidéo : \emph{protusion}

\end{description}



Ensuite, nous allons installer des plugins ``individuels'' :
\begin{quote}
\begin{description}
\item[{\url{http://www.sketchucation.com/forums/scf/download/file.php?id=25306}}] \leavevmode
Shape bender

\item[{\url{http://www.crai.archi.fr/RubyLibraryDepot/Ruby/fr\_RUBY\_Library\_Depot.html}}] \leavevmode
Tous les plugins sur un seul site, en Français!

\end{description}
\end{quote}


\section{Installation de plugins}
\label{su/install-plugin-su:installation-de-plugins}\label{su/install-plugin-su::doc}\label{su/install-plugin-su:install-plugin-su}
Sketchup est, à la base, un programme simple, voir simpliste diront certains (les pauvres...).

Néanmoins cette simplicité n'est qu'apparente, car le logiciel offre la possibilité (à quelques initiés) de créer de nouveaux outils, de mettre en place de nouvelles fonctions.

Un certains nombre de ``geeks'' et de chercheurs farfelus passent donc leurs journées (et sans doute une partie de leurs nuits...) à coder pour le bien de l'humanité, et il en ressort ce que l'on appelle des \emph{scripts Ruby}. Ces petits bouts de programme, en langage ruby, ne pesant quasiment rien (quelques Ko tout au plus) s'ajoutent dans le dossier plugin de Sketchup et donnent accès à de nouvelles fonctions au sein du programme.

La grande majorité de ceux ci, ainsi que beaucoup d'autres peuvent également être téléchargés sur le site qui leur est dédié par laboratoire du CRAI, Nancy (merci  a eux !), ou sur sketchucation, dans le forum prévu a cet effet.

Tous ces scripts se retrouvent globalement sous
* l'onglet \textbf{Plugin} de Sketchup,
* mais certains ont des barres d'outils attitrées (Affichage --\textgreater{} Barres d'outils --\textgreater{}...),
* d'autres une place dans le menu contextuel du clic droit,
* d'autres encore sont ailleurs (dans ...Outils ou dans ...Dessin par exemple).

Enfin , il est parfois nécessaire de les activer via le menu ...Fenêtre --\textgreater{} ...Préferences --\textgreater{} ...Extensions (c'est le cas pour la Sandbox par exemple)

\begin{notice}{note}{Note:}
Sketchup, logiciel américain, est agrémenté par des plugins venant des 4 coins de la planète : le langage universel reste \textbf{l'anglais}, et la majorité des plugins ne sont pas traduits en français ... c'est la rançon de la gratuité ...
\end{notice}


\subsection{Installation}
\label{su/install-plugin-su:installation}

\subsubsection{Méthode traditionnelle}
\label{su/install-plugin-su:methode-traditionnelle}
L'installation d'un script Ruby est est simple : il faut copier le fichier voulu (extension \sphinxcode{*.rb}) dans le fichier \sphinxcode{plugin} de Sketchup (logiquement situé dans :file:{\color{red}\bfseries{}{}`}C:Program filesGoogleGoogle Sketchup(6 ou 7)Plugins).

\begin{notice}{warning}{Avertissement:}
Certains scripts ruby ont des \textbf{dépendances} , i.e : leur exécution nécéssite l'installation d'autres scripts ``additionnels'' : lisez bien les recomandations des auteurs quand à l'installations des plugins/scripts

Version de Sketchup : certains scripts sont spécifiques à une version particulère de Sketchup
\end{notice}

Certains plugins sont très complets, ils comportent même une barre d'outil dédiée. Leur format est souvent sous forme de \sphinxcode{*.zip}, il faut à ce moment là \textbf{décompresser} le contenu entier du zip à l'intérieur du dossier plugin. On y retrouve alors un script ruby et un dossier contenant d'autres scripts, ainsi que des images (icônes de barre d'outils)


\subsubsection{Méthode ``installateur''}
\label{su/install-plugin-su:methode-installateur}
Certains plugins, en plus d'être accompagnés de barres d'outils, comportent des fenêtres de dialogue lors de leur exécution. Nous sommes en présence de minis logiciels venant se greffer sur l'interface de Sketchup. Pour en faciliter l'installation (qui requiert souvent d'effectuer plus d'opérations que celles précitées auparavant) les développeurs ont poussé le bouchanon jusquà proposer des installateurs ``autonomes''.

Le processus consiste
* à double-cliquer sur l'installateur,
* puis à  valider le chemin où se trouve Sketchup... c'est tout!

(Meta)Plugins livrés sous ce format : les barres d'outls SCF, 1001 tools, etc.


\subsection{Liste des scripts}
\label{su/install-plugin-su:liste-des-scripts}
\textbf{NON} exhaustive ! voir plus bas quelques liens WWW
\begin{itemize}
\item {} \begin{description}
\item[{Symetrie de la sélection}] \leavevmode\begin{itemize}
\item {} 
Usage : Complète efficacement l'outil un peu étrange dont Sketchup est affublé à l'origine (« retourner le long de » ou « flip along »...).

\item {} 
Méthode : Sélectionner l'objet à « symétriser » ---\textgreater{} clic droit --\textgreater{} « symétrie de la sélection » --\textgreater{} définir le plan de symétrie en traçant deux lignes, librement, dans l'espace --\textgreater{} choisir de conserver ou non l'élément d'origine.

\end{itemize}

\end{description}

\item {} \begin{description}
\item[{Compo spray}] \leavevmode\begin{itemize}
\item {} 
Usage : Permet de poser extrêmement rapidement et efficacement des composants dans un modèle. Indispensable pour peupler une maquette d'arbres, de personnages ou de n'importe quoi d'autre. Comporte des fonctions de rotation et de changement d'échelles paramétriques qui donneront l'impression que chaque petit composant est différent de ses voisins, renforçant ainsi l'illusion d'un dur labeur accompli...

\item {} 
Méthode : barre d'outils (« Draw » --\textgreater{} « Compo spray ») --\textgreater{} dans la fenêtre de dialogue, choisir le composant, le type de projection voulue, les paramètres d'aléa et...enjoy ! (doc. très complète dans le fichier du plugin)

\end{itemize}

\end{description}

\item {} \begin{description}
\item[{Purger}] \leavevmode\begin{itemize}
\item {} 
Usage : Supprime les éléments inutilisés dans le model (textures, composants, styles...) qui ne manquent pas de se stocker stupidement un peu partout dès que vous les utilisez, ne serait ce qu'une fois, et ne disparaissent pas tout seuls par la suite (ça peut alourdir considérablement un fichier).

\item {} 
Méthode : Rien de plus simple,  « Plugins » --\textgreater{} « Purger »...

\end{itemize}

\end{description}

\item {} \begin{description}
\item[{Courbes de bezier}] \leavevmode\begin{itemize}
\item {} 
Usage : Ajoute plusieurs méthodes de dessins de courbes aux outils Sketchup. Permet de dépasser largement les limites évidentes de l'outil arc d'origine de Sketchup, et ainsi de s'épanouir dans le dessin spontané de formes courbes complexes.

\item {} 
Méthode : Sélectionner un des outils de dessin de courbe et improviser...

\end{itemize}

\end{description}

\item {} \begin{description}
\item[{Fenetreur}] \leavevmode\begin{itemize}
\item {} 
Usage:Vous en avez assez de toujours galérer pour créer un ensemble de fenêtres ou un mur rideau un peu régulier ? Pas de soucis, ce plugin est pour vous. A partir d'une surface, vous pouvez créer directement une (ou plusieurs) fenêtre, en paramétrant le nombre d'éléments, la taille ou les matériaux de ceux ci.

\item {} 
Méthode : sélectionner une ou plusieurs surfaces -\textgreater{} clic droit -\textgreater{} windowizer -\textgreater{} définir les paramètres des fenêtres.

\end{itemize}

\end{description}

\item {} \begin{description}
\item[{Grey scale}] \leavevmode\begin{itemize}
\item {} 
Usage : Fonctionne exactement comme un passage en noir et blanc dans Photoshop, mais applique à tout le model, ajoutant ainsi la petite note de style qui manquait à votre 3D...

\end{itemize}

Le phénomène est de plus réversible à tout moment, en revenant simplement dans le plugin.
- Méthode : Plugin --\textgreater{} Grey mode. Petit plus : toute couleur rajoutée après le passage en mode gris est affichée telle quelle, il est donc possible de mettre en évidence facilement des éléments particuliers du projet.

\end{description}

\item {} \begin{description}
\item[{Face de coupe}] \leavevmode\begin{itemize}
\item {} 
Usage : Non, ce n'est pas une vieille insulte des années 80, mais un outil qui crée une face à l'intersection du plan de coupe, « refermant » ainsi visuellement les murs et autres géométries, qui restent sinon désespérement creuses... Attention, cette face ne se déplace pas avec le plan de section... Si vous coupez ensuite ailleurs, il faudra refaire l'opération.

\item {} 
Méthode : Faire une coupe --\textgreater{} sélectionner le plan de coupe --\textgreater{} clic droit --\textgreater{} « add a section cut face » --\textgreater{} choisir sa couleur et son calque.

\end{itemize}

\end{description}

\item {} \begin{description}
\item[{Projection / extension}] \leavevmode\begin{itemize}
\item {} 
Usage : Permet (entre autres) d'extruder une ligne ou un ensemble de lignes. Fonctionne globalement comme l'outil pousser/tirer, mais en moins restrictif. Très utile pour récupérer des données d'Autocad sans perdre des heures à recréer toutes les faces nécessaires à l'extrusion d'un mur. Ce plugin offre également la possibilite d'extruder des faces (une ou  plusieurs en même temps) selon un axe quelconque (et pas systématiquement uniquement leur normale...)

\item {} 
Méthode : Barre d'outil --\textgreater{} sélectionner les lignes /faces à extruder --\textgreater{} définir la direction et la distance voulue.

\end{itemize}

\end{description}

\item {} \begin{description}
\item[{Supprimer les lignes coplanaires}] \leavevmode\begin{itemize}
\item {} 
Usage : Principalement utile si vous devez importer des données créées dans d'autres applications, via le format 3Ds par exemple. En effet, ces opérations d'import/export se traduisent souvent par des phénomènes de triangulation des faces, qui grèvent  sauvagement l'effet visuel, et la manipulation des éléments dans Sketchup.

\item {} 
Méthode : Sélectionner l'objet à « nettoyer » --\textgreater{} clic droit --\textgreater{}  « détruire segments coplanaires ».

\end{itemize}

\end{description}

\item {} \begin{description}
\item[{Sandbox :}] \leavevmode\begin{itemize}
\item {} 
Usage : Set d'outils multi-usage de création de terrains et de bien d'autres choses si on a de l'imagination... Ce n'est pas à proprement parler un script Ruby, puisqu'il est intégré de base dans Sketchup, mais il se comporte comme tel, car il n'est pas affiché d'origine. Il propose deux méthodes distinctes, l'une se basant sur la construction d'une grille déformable, l'autre sur l'exploitation de courbes de niveaux.

\item {} \begin{description}
\item[{Méthode :}] \leavevmode\begin{itemize}
\item {} 
Grille : Tracer une grille aux dimensions voulues --\textgreater{}  sélectionner l'outil de déformation et lui donner une taille -\textgreater{} cliquer sur une zone et définir l'élévation.

\item {} 
Courbes de niveaux : Sélectionner l'ensemble des courbes --\textgreater{} sélectionner l'outil correspondant --\textgreater{} éventuellement, afficher les arrêtes cachées pour pouvoir supprimer la géométrie inutile.

\end{itemize}

\end{description}

\end{itemize}

\end{description}

\end{itemize}

etc.


\sphinxstrong{Voir aussi:}

\begin{description}
\item[{\url{http://www.crai.archi.fr/RubyLibraryDepot/Ruby/fr\_RUBY\_Library\_Depot.html}}] \leavevmode
site Web maintenu par \textbf{Didier BUR} (enseignant/chercheur à l'école d'archi de Nancy). on peut y trouver les contributions des développeurs de scripts Ruby pour SketchUp depuis 2004. Il y a en tout \textbf{494} scripts téléchrgeables gratuitement. Le plugin \emph{fr\_SectionCutFace.rb} est installé sur {\hyperref[su/export\string-dwg:export\string-dwg\string-coupes]{\sphinxcrossref{\DUrole{std,std-ref}{cette page}}}}

\item[{\url{http://www.scriptspot.com/}}] \leavevmode
des scripts

\item[{\url{http://www.plugins.ro/}}] \leavevmode
encore

\item[{\url{http://archsymb.com/ruby/}}] \leavevmode
d'autres

\item[{\url{http://www.formfonts.com/browse/browseScripts.php?function=category}}] \leavevmode
certains payants

\item[{\url{http://www.objectivenetworks.net/indexScripts.php}}] \leavevmode
d'autres aussi

\item[{\url{http://www.smustard.com/}}] \leavevmode
quoiqu'il puisse y avoir les 2

\item[{\url{http://www.pushpullbar.com/}}] \leavevmode
des scripts sur un forum ? ben oui, et franchement pas les moins mauvais, comme ``joint push pull''

\item[{\url{http://members.cox.net/rick.wilson/}}] \leavevmode
windowizer, tu me bluffera toujours ....

\item[{\url{http://www.iesve.com/Software/Free-and-trial-software/}}] \leavevmode
par des pros : analyse de la consommation énergétique globale de votre projet

\item[{\url{http://www.pixdim.com/en/home.html}}] \leavevmode
besoin de mesure 3D d'après des photos : la photogrammétrie c'est possible aussi dans Sketchup

\item[{\url{http://www.tensile-structures.de/help.html}}] \leavevmode
architecture ``tensile'' (toile tendues) avec le plugin ``soap \& bubble skin''

\item[{\url{http://www.northernlightstimberframing.com/su/}}] \leavevmode
un ancien informaticien devenu charpentier : avec la famille de plugins ``TF rubies'' vous pourrez non seulement dessiner, mais aussi extraire les plans de débits de vos pièces de bois! plus besoin de Cadwork! de la folie!

\item[{\url{http://www.drawmetal.com/}}] \leavevmode
vous avez envie de dessiner du fer forgé?

\item[{\url{http://1oo1bit.com/default.html}}] \leavevmode
plein (trop?) en une seule barre

\item[{\url{http://www.special.eclipse.co.uk/sketchup\_training.html} (et cliquez sur Downloads)}] \leavevmode
5 barres d'outils qui regroupent les meilleurs plugins, mais pas tous ....

\end{description}




\section{Importation ``fond de plan'' dans Sketchup}
\label{su/import-ssqu:import-ssqu}\label{su/import-ssqu::doc}\label{su/import-ssqu:importation-fond-de-plan-dans-sketchup}
Nous disposons de *.pdf du projet (renommés de fixhaus\_berchtold2.. en porkeno..). Essayons de l'importer dans Sketchup.


\subsection{Préparation du dessin}
\label{su/import-ssqu:preparation-du-dessin}
La configuration initiale de l'interface se poursuit ici. Chaque modification est enregistrée automatiquement par le logiciel : le prochain démarrage aura mémorisé les réglages de la ``session'' précédente.

Affichons des ``palettes'' (terminologie AutoCAD) directement dans la zone de dessin, essentielles à tout bon travail. Ces palettes sont ``dockable'' : elle peuvent se ``coller'' à un bord de fenêtre, etc. Elles sont aussi repliables : un clic sur la barre de titre (barre supérieure ``épaisse'' de la palette) les plie, un second la déplie.


\subsubsection{Calques}
\label{su/import-ssqu:calques}\label{su/import-ssqu:creation-calques-su-debut}
\begin{notice}{note}{Note:}
Comme pour AutoCAD plus tard, il faut créer des ``couches'' de dessin qui recevront des objets similaires (\emph{nature} : images, lignes épaisses, lignes fines, etc., ou \emph{fonction} menuiserie intérieure, menuiserie extérieure, etc.)
\end{notice}
\begin{enumerate}
\item {} \begin{description}
\item[{Liste des calques à créer \emph{avant} de dessiner quoi que ce soit}] \leavevmode{[}(une appellation explicite est plus facilement exploitable{]}{[}donnez des noms compréhensibles à vos calques!){]}\begin{itemize}
\item {} 
import\_img

\item {} 
murs\_ext

\end{itemize}

\end{description}

\item {} \begin{description}
\item[{Activation du calque}] \leavevmode\begin{itemize}
\item {} 
Un calque est rendu actif lorsqu'on clique dans le cercle situé à gauche de son nom : Activez \sphinxcode{import\_img}

\end{itemize}

\end{description}

\end{enumerate}


\subsubsection{Information sur l'entité}
\label{su/import-ssqu:information-sur-l-entite}
Le menu \sphinxmenuselection{Fenêtres \(\rightarrow\) Information sur l'entité}, vous afficherez la palette qui vous permettra de faire changer de calque les ``entités'' que vous aurez sélectionnées.

``Dockez'' cette palette sous celle des calques et repliez-là, en cliquant sur la barre supérieure.


\subsection{Importation en *.pdf : NOK}
\label{su/import-ssqu:importation-en-pdf-nok}

\subsubsection{Problème : importation des images en *.pdf impossible}
\label{su/import-ssqu:probleme-importation-des-images-en-pdf-impossible}
\sphinxmenuselection{Fichier \(\rightarrow\) Import \(\rightarrow\) type de fichier \(\rightarrow\) tous les formats d'image supportés} :  le pdf n'est pas reconnu!.

Sketchup ne reconnaît pas le format \emph{*.pdf} comme un format d'image \emph{importable}.

Il faut transformer ce pdf en un autre format qui sera utilisable.

Il existe plusieurs alternatives :
\begin{itemize}
\item {} \begin{description}
\item[{ouverture du *.pdf dans un logiciel pouvant en transformer le format vers le png, jpg, etc. :}] \leavevmode\begin{itemize}
\item {} 
Acrobat

\item {} 
Photoshop

\item {} 
Gimp

\end{itemize}

\end{description}

\item {} \begin{description}
\item[{capture d'image :}] \leavevmode\begin{itemize}
\item {} 
soit en utilisant les outils ``windows'' : touche {[}printscreen{]} pour capturer l'écran entier, ou {[}alt{]}+{[}printscreen{]} pour capturer la fenêtre active, puis ouverture d'un logiciel d'imagerie, et nouveau fichier, puis copie du contenu du ``presse-papier'' et enregistrement.

\item {} 
soit en utilisant un logiciel dédié : \href{http://www.faststone.org/FSCaptureDetail.htm}{Fastone Capture} , ou \href{http://www.techsmith.fr/snagit.asp}{Snagit} : l'enregistrement en sera simplifié.

\item {} 
capture avec l'outil \textbf{instantané}, présent dans \emph{Acrobat reader} (à faire apparaître en faisant un clic-droit sur la barre d'outils et sélectionner ``ajouter des outils'').

\end{itemize}

\end{description}

\end{itemize}

\begin{notice}{note}{Note:}
Beaucoup d'entre-nous confondent \emph{Acrobat} et \textbf{Acrobat reader} : ce 2e logiciel n'est que le \emph{visualisateur} de *.pdf , téléchargeable gratuitement, mais aux possibilités limitées ... à la visualisation et l'impression.
\end{notice}

Nous choisirons de ``capturer'' l'image ou la portion d'image avec l'outil instantané d'AcrobatReader, puis de copier cette capture dans Paint, pour obtenir une image que nous enregistrerons au format *.png .


\subsubsection{Capture du pdf avec l'outil instantané dans Acrobat Reader}
\label{su/import-ssqu:capture-du-pdf-avec-l-outil-instantane-dans-acrobat-reader}\label{su/import-ssqu:capture-outil-acroread-instantane}
Sélectionnez la zone à capturer avec cet outil :
\phantomsection\label{su/import-ssqu:fig-outil-acroread-instantane}
\noindent{\hspace*{\fill}\sphinxincludegraphics{{import_ssq_outil-instantane_01}.png}\hspace*{\fill}}

\noindent{\hspace*{\fill}\sphinxincludegraphics{{import_ssq_outil-instantane_02}.png}\hspace*{\fill}}

Ouverture d'un petit logiciel bien pratique  \emph{PAINT} :
\phantomsection\label{su/import-ssqu:fig-paint}
\noindent{\hspace*{\fill}\sphinxincludegraphics{{import_ssq_outil-paint_03}.png}\hspace*{\fill}}

Dans Paint : ``importation'' de la capture réalisée par l'outil ``instantané'' (c'est un copier-coller entre applications différentes) :

\noindent{\hspace*{\fill}\sphinxincludegraphics{{import_ssq_outil-paint_04}.png}\hspace*{\fill}}

Si l'image importée est plus grande que la zone prévue par le logiciel, autorisez celui-ci à agrandir la zone d'image :

\noindent{\hspace*{\fill}\sphinxincludegraphics{{import_ssq_outil-paint_05}.png}\hspace*{\fill}}

Recadrage/Retaillage :

\noindent{\hspace*{\fill}\sphinxincludegraphics{{import_ssq_outil-paint_06}.png}\hspace*{\fill}}


\subsection{Importation en *.png : OK}
\label{su/import-ssqu:importation-en-png-ok}
Il ne reste plus qu'à enregistrer l'image en \sphinxcode{porkeno\_plan-rdc.png}.

Activation du calque \sphinxcode{import\_img}, créé plus haut {\hyperref[su/import\string-ssqu:creation\string-calques\string-su\string-debut]{\sphinxcrossref{\DUrole{std,std-ref}{Calques}}}}
\begin{description}
\item[{\sphinxmenuselection{Fichier \(\rightarrow\) Import \(\rightarrow\) type de fichier \(\rightarrow\) tous les formats d'image supportés} :}] \leavevmode\begin{itemize}
\item {} 
sélectionnez \sphinxcode{porkeno\_plan-rdc.png}

\item {} 
utiliser comme image

\end{itemize}

\end{description}

premier point d'insertion : l'image apparaît ``collée'' au bout du pointeur cliquez sur l'origine (0;0;0)

deuxième point : à estimer selon la taille de l'image : si c'est la capture d'une page au format A4 portrait (c'est le cas ici) entrez \sphinxcode{20cm} dans la zone de controle de valeurs (ZCV)


\subsection{Mise à l'échelle de l'image dans Sketchup}
\label{su/import-ssqu:mise-a-l-echelle-de-l-image-dans-sketchup}
Suivez ce document {\hyperref[su/redimensionnement\string-images\string-outil\string-mesure::doc]{\sphinxcrossref{\DUrole{doc}{Redimensionnement d'une image}}}} pour avancer à l'étape suivante.


\section{Redimensionnement d'une image}
\label{su/redimensionnement-images-outil-mesure:redimensionnement-images-outil-mesure}\label{su/redimensionnement-images-outil-mesure::doc}\label{su/redimensionnement-images-outil-mesure:redimensionnement-d-une-image}
Sketchup propose un outil spécifique ``Mettre à l'échelle'' (raccourci : \textbf{S}) pour redimensionner les objets. On peut donc l'utiliser pour redimensionner une image.


\subsection{Problème  : inadéquation de l'outil ``Échelle''}
\label{su/redimensionnement-images-outil-mesure:probleme-inadequation-de-l-outil-echelle}
Cet outil peut redimensionner de façon \emph{uniforme} (homothétie) {[}Maj{]}+{[}S{]}, à partir du \emph{centre} {[}Ctrl{]}+{[}S{]}, avec un \emph{facteur d'échelle} : ZCV = nombre et avec une \emph{longueur déterminée} : ZCV = nombre avec unités.

Cette dernière possibilité est intéressante, mais difficilement applicable à une image, car l'outil échelle ne s'applique que sur les bords de l'objet ``image''. Si les points de référence sont \emph{à l'intérieur} de l'image, à moins d'un découpage à ces limites, l'outil échelle est inapproprié.


\subsection{Solution : Utilisation de l'outil ``Mesure''}
\label{su/redimensionnement-images-outil-mesure:solution-utilisation-de-l-outil-mesure}
Cet outil (raccourci : \textbf{T}) est utilisé fréquemment dans Sketchup pour établir des lignes de ``construction'' (repères).

Ici, on va détourner son usage en appliquant une méthode de redimensionnement par indication d'une nouvelle distance (comme dans les logiciels de DAO ``industrie'').

Si vous venez de lire {\hyperref[su/import\string-ssqu:import\string-ssqu]{\sphinxcrossref{\DUrole{std,std-ref}{Importation ``fond de plan'' dans Sketchup}}}} vous avez une belle image du plan de Rez de Chaussée du modèle .

Pour la dimensionner, il nous faut connaître les dimensions ``hors-tout'' de cette maison, en se basant sur la \href{http://www.fixhaus.at/Berchtold\%20Typ2.pdf}{notice} du modèle ``Berchtold Type 2'' . On y découvre en page 2, que la forme est un \emph{rectangle de 12,63 m sur 6,63 m}
\begin{enumerate}
\item {} 
\textbf{T} activez l'outil mesure (icône : mètre à ruban)

\item {} 
premier point : cliquez sur le coin supérieur gauche (extérieur du mur). (Au besoin, faites un grossissement d'affichage (zoom ``molette''))

\item {} 
deuxième point : cliquez sur le coin supérieur droit (extérieur du mur). (Au besoin, déplacez l'image sans rotation (=pan) : {[}Maj{]}+{[}Molette{]} et Zoomez)

\item {} 
placez le curseur dans la ZCV {[}Tab{]}, et rentrez la dimension voulue \sphinxcode{12,63} (attention à la virgule : c'est celle du clavier ``texte'')

\item {} 
acceptez le redimensionnement de l'image

\end{enumerate}

\begin{notice}{note}{Note:}
Le redimensionnement d'une image en sous-qu n'est jamais ``exact'', car même si l'on donne une dimension correcte, on n'indique jamais le point de départ de la ligne de ``redimensionnement'' avec précision : on clique un peu au hasard sur ce que l'on croit être le milieu d'un ligne, cela est dû à l'imprécision du fort grossissement d'une image.

Ce fond de plan sert surtout à donner un idée générale de ce qu'il faut redessiner avec Sketchup.
\end{notice}

Redimensionner avec l'outil mesure permet d'utiliser n'importe quelle image possédant une longueur connue.
\begin{description}
\item[{Le seul désavantage}] \leavevmode{[}c'est une opération homothétique. Si l'on désire ``déformer'' une image différemment selon les axes X et Y, ou même en s'accrochant sur plusieurs points ``d'appariement'', il faut se tourner vers :{]}\begin{itemize}
\item {} 
une autre méthode : voir {\hyperref[su/import\string-img\string-com\string-texture:image\string-com\string-txtur]{\sphinxcrossref{\DUrole{std,std-ref}{Importation d'une image en tant que texture}}}}

\item {} 
des logiciels plus spécialisés, comme \href{http://www.autodesk.fr/adsk/servlet/index?siteID=458335\&id=12615796}{AutoCAD Raster Design}

\end{itemize}

\end{description}

Nous sommes prêts à dessiner : suivons le guide {\hyperref[init_su+acad/su1:su1]{\sphinxcrossref{\DUrole{std,std-ref}{Modélisation initiale dans Sketchup}}}}


\subsection{Redimensionnement : liste des solutions possibles}
\label{su/redimensionnement-images-outil-mesure:redimensionnement-liste-des-solutions-possibles}
Nous venons de voir qu'il existe plusieurs façons de redimensionner un image, chacune ayant des avantages et des inconvénients :

\noindent\begin{tabulary}{\linewidth}{|L|L|L|}
\hline

\textbf{Méthode/Outil}
&
\textbf{Avantage}
&
\textbf{Inconvénient}
\\
\hline
\textbf{Échelle {[}S{]}}
&
pratique, précis, redimensionnement de plusieurs images possible
&
s'accroche au bord de l'image, ne permet pas d'utiliser d'autres points de référence
\\
\hline
\textbf{Mesure {[}T{]}}
&
permet de redimensionner à partir de n'importe quelle référence
&
ne permet pas de redimensionner plusieurs images
\\
\hline
\textbf{Texture}
&
redimensionne aisément une image importée en tant que texture
&
impose d'avoir une géométrie au préalable, méthode un peu laborieuse (punaises)
\\
\hline\end{tabulary}



\section{Importation d'une image en tant que texture}
\label{su/import-img-com-texture:image-com-txtur}\label{su/import-img-com-texture:importation-d-une-image-en-tant-que-texture}\label{su/import-img-com-texture::doc}
Il existe plusieurs façons d'importer une image dans Sketchup

\noindent\begin{tabulary}{\linewidth}{|L|L|}
\hline

\textbf{Image}
&
\textbf{Type importation}
\\
\hline
\sphinxcode{image}
&
l'image est importée en tant qu'image. Elle peut être redimensionnée, etc.
\\
\hline
\sphinxcode{texture}
&
l'image est importée en tant que texture. C'est pour la redimensionner selon une géométrie existante
\\
\hline
\sphinxcode{photo-match}
&
l'image est importée en tant que modèle. C'est la géométrie qui va pouvoir être redimensionnée selon cette image.
\\
\hline\end{tabulary}



\subsection{Concept}
\label{su/import-img-com-texture:concept}
Ici, nous traitons du deuxième type d'import.

C'est une fonctionnalité très pratique, notamment pour (re)créer des monuments à partir d'images : il suffit de créer une géométrie (coque vide), et d'y plaquer une texture à partir des images. L'illusion fonctionne bien, c'est d'ailleurs ce qui a motivé \sphinxcode{Google} pour l'achat du logiciel Sketchup détenu alors par la société \sphinxcode{@last} .


\subsection{Étapes}
\label{su/import-img-com-texture:etapes}\begin{enumerate}
\item {} 
importez une image, sélectionnez ``en tant que texture'' (bouton radio en bas à droite)

\item {} 
collez cette image sur la géométrie servant de référence

\item {} 
clic-droit \textgreater{} texture \textgreater{} position : l'image apparaît démultipliée, avec des ``punaises'' de couleur différentes aux 4 coins (bords extérieurs)

\item {} 
décochez (si ce n'est pas déjà fait) : ``punaises bloquées''

\item {} \begin{description}
\item[{déplacement des punaises vers les points de référence de l'image :}] \leavevmode\begin{itemize}
\item {} 
clic (une fois, puis relâchez) sur une punaise : elle change de forme, ainsi que le pointeur (qui hélas! la recouvre)

\item {} 
déplacement de l'ensemble pointeur/punaise vers la référence

\item {} 
clic sur le point de référence choisi dans l'image (un angle de bâtiment, ou autre point caractéristique, ...)

\item {} 
répéter l'opération pour chaque punaise

\end{itemize}

\end{description}

\item {} \begin{description}
\item[{redimensionnement de l'image selon la géométrie sous-jacente}] \leavevmode\begin{itemize}
\item {} 
cliquez-glisser (clic, puis maintient de la touche gauche de la souris enfoncée et glissement vers le point ...) sur chaque punaise : l'image se déforme nettement

\item {} 
relâchez sur le point choisi (angle de la géométrie, etc.)

\item {} 
répétez l'opération pour chaque punaise

\end{itemize}

\end{description}

\item {} 
vous pouvez répéter les étapes précédentes pour chaque face ...

\end{enumerate}


\subsection{Autres méthodes de redimensionnement}
\label{su/import-img-com-texture:autres-methodes-de-redimensionnement}
Selon les besoins, il peut être bon de faire un tour sur ce document : {\hyperref[su/redimensionnement\string-images\string-outil\string-mesure:redimensionnement\string-images\string-outil\string-mesure]{\sphinxcrossref{\DUrole{std,std-ref}{Redimensionnement d'une image}}}}


\section{Déplacement multiple/clonage}
\label{su/clonage:deplcmt-multipl}\label{su/clonage::doc}\label{su/clonage:deplacement-multiple-clonage}
La touche {[}\textbf{CTRL}{]} associée à un outil de modification (déplacement {[}\textbf{M}{]}, rotation {[}\textbf{Q}{]} ) permet de ``cloner'' des objets.

L'équivalent AutoCAD est la commande \emph{Réseau}


\subsection{Principe \& usage}
\label{su/clonage:principe-usage}
Ces objets peuvent être nimporte-quelle entité de dessin : ligne, surface, volume, composant, etc.
\begin{description}
\item[{le clonage est soit :}] \leavevmode\begin{itemize}
\item {} 
effectué avec une distance fixe (petite) et un nombre de copies (par \emph{multiplication})

\item {} 
effectué avec la distance totale et un nombre de copie (par \emph{division})

\end{itemize}

\end{description}


\subsection{Déplacement M + CTRL}
\label{su/clonage:deplacement-m-ctrl}
Réseau ``linéaire'' : copie le long d'un axe (x ou y ou z)
\begin{description}
\item[{Exemple 1 :}] \leavevmode\begin{itemize}
\item {} 
vous dessinez les chevrons (section 6X8) d'une charpente, disposés selon un entraxe de 60 cm.

\item {} 
sélectionnez le chevron

\item {} 
activez l'outil déplacement : \textbf{M}

\item {} 
appuyez sur la touche \textbf{CTRL} (l'icône de l'outil possède un ``plus'' dans le coin droit supérieur)

\item {} 
sélectionnez l'arête du chevron

\item {} 
indiquez comme valeur de déplacement (60 -1/2 ep. chevron = 60-3 = ) \sphinxcode{57 cm} dans la ZCV. Attention : restez ``collé'' sur un axe

\item {} 
vous avez à ``chevronner'' sur une distance de 12 m de façade? indiquez (12/0.6 = ) \sphinxcode{x20} ou \sphinxcode{20x} dans la ZCV. Remarquer le ``x''

\end{itemize}

\item[{Example 2 :}] \leavevmode\begin{itemize}
\item {} 
vous dessinez les poteaux d'une pergola. Cette pergola couvre une terrasse faisant 14,45 m de longueur (c'est une terrasse du Sud de la France ...)

\item {} 
sélectionnez le poteau que vous avez dessiné à une extrémité de la pergola

\item {} 
activez l'outil déplacement : \textbf{M}

\item {} 
appuyez sur la touche \textbf{CTRL} (l'icône de l'outil possède un ``plus'' dans le coin droit supérieur)

\item {} 
cliquez sur l'arête du poteau

\item {} 
indiquez comme valeur de déplacement \sphinxcode{14,45} (Sketchup a été configuré pour le dessin en mètres, 14,45=14,45m). Attention à utiliser la \sphinxcode{,} et non le \sphinxcode{.}

\item {} 
vous voulez 4 poteaux entre les poteaux d'extémité? Indiquez \sphinxcode{4/} ou \sphinxcode{/4}. Enjoy

\end{itemize}

\end{description}


\subsection{Rotation Q + CTRL}
\label{su/clonage:rotation-q-ctrl}
Réseau ``polaire'' : copie autour d'un axe (x ou y ou z)

La combinaison de ces touches permet d'effectuer des copies ``circulaires''.
\begin{description}
\item[{Example 3 :}] \leavevmode\begin{itemize}
\item {} 
vous (tenter de ...) dessiner les marches d'un escalier balancé

\item {} 
sélectionnez la marche

\item {} 
activez l'outil rotation : \textbf{Q}

\item {} 
appuyez sur la touche \textbf{CTRL} (l'icône de l'outil possède un ``plus'' dans le coin droit supérieur)

\item {} 
cliquez sur le coin intérieur supérieur (axe de référence)

\item {} 
cliquez sur l'extrémité extérieure (on vient de définir la ligne de référence)

\item {} 
indiquez comme valeur de déplacement (giron = ) \sphinxcode{22cm} (on doit spécifier l'unité pour toute dimension différente de l'unité par défaut)

\item {} 
indiquez le nombre \sphinxcode{4x} ou \sphinxcode{x4}

\item {} 
il ne vous reste plus qu'a déplacer chaque marche verticalement.

\item {} 
vous pouvez aussi choisir un déplacement faisant tourner \emph{\&} monter les marches ...

\end{itemize}

\end{description}


\section{Escaliers}
\label{su/escalier:escalier-su}\label{su/escalier::doc}\label{su/escalier:escaliers}
Il existe au moins 3 méthodes pour faire un escalier, nous les détaillerons plus bas.

Plusieurs plugins permettent de réaliser des escalier droits à 1 ou 2 volée, des escaliers hélicoïdaux, mais aucun ne proposent une solution correcte à la problématique des escaliers balancés.

Si l'on regarde du coté des logiciels ``Architecture'' (je pense à Archicad, que les autres me renseignent sur les possiblités et limitations de Allplan, Vectorworks, Revit) on trouve aussi des limites aux solutions qu'ils proposent, notamment dès que l'on se retrouve dans des géométries de ``guingois'' comme en rénovation. Je crois que même 3DS sèche un peu dans ce domaine (escaliers balancés à construire dans des cages d'escalier ``tordues''). de toute façon, même si un logiciel arrive à le dessiner correctement, il faudra trouver l'artisan que sache le bâtir ... (recherche artisan sanchant faire un escalier sur voûte sarrasine désespéremment ...)

Nous allons donc explorer les méthodes ``basiques'' pour ``monter'' un escalier.


\subsection{Rectangle \& sous-divisions}
\label{su/escalier:rectangle-sous-divisions}
Cette méthode emploie la sous-division d'un rectangle en autant de ``marches'', puis l'extrusion de ceux-ci. Elle est intuitive, mais laborieuse.
\begin{itemize}
\item {} 
\textbf{R} faire un rectangle représentant l'emprise de l'escalier au sol

\item {} 
groupez-le : \sphinxmenuselection{double-clic \(\rightarrow\) clic-droit \(\rightarrow\) créer groupe} : vous pourrez construire votre escalier ``à l'extérieur'' et déplacer aisémment au sein de votre construction une fois fini.

\item {} 
\textbf{SPC} double-cliquez dessu pour activer l'édition de ce nouveau groupe

\item {} 
sélectionnez un bord (long coté) et divisez-le (\sphinxmenuselection{sélection \(\rightarrow\) clic-droit \(\rightarrow\) diviser}), en un nombre de segments égal au nombre de marches (longueur/giron = )

\item {} 
\textbf{L} dessiner des lignes partant des extrémités de ces segment, en direction de l'autre bord (accroche perpendiculaire svp. : i.e le long d'un axe!) : votre rectangle est divisé en autant de marches.

\end{itemize}

\begin{notice}{note}{Note:}
Sketchup vous indique des points particuliers lorsque vous faites parcourir le pointeur le long d'un segment : extrémité (point vert), milieu (point cyan). En plus de ce code couleur, vous avez des ``info-bulles'' répétant la nature de ces points. Lisez-les!
\end{notice}
\begin{itemize}
\item {} 
\textbf{L} sur une extrémité, dessinez une droite verticale (selon axe bleu) repésentant la hauteur totale à franchir

\item {} 
divisez cette droite en autant de nombre de hauteurs de marche (hauteur totale/hauteur marche = )

\item {} 
\textbf{P} extrudez chaque sous-rectangle, en commançant par la marche la plus haute. Souvenez-vous qu'il doit rester une hauteur à franchir : pour la hauteur d'extrusion, accrochez-vous sur l'avant-dernier point, avant le sommet de la droite.

\item {} 
recommencez pour chaque marche, en allant chercher (avec l'outil ``push-pull'' activé) chaque extémité

\item {} 
\textbf{E} effacez avec l'outil ``gomme'' chaque ligne verticale provenant de l'extrusion des marches

\end{itemize}


\subsection{Profil copié}
\label{su/escalier:profil-copie}
Cette méthode utilise en core l'outil ``push-pull'', mais en une seule fois, en se servant d'un profil représentant les marches. Elle est plus ``élégante''.
\begin{itemize}
\item {} 
\textbf{R} faites un rectangle représentant l'emprise de l'escalier au sol

\item {} 
groupez-le

\item {} 
\textbf{SPC} double-cliquez dessu pour activer l'édition de ce nouveau groupe

\item {} 
\textbf{P} extrudez-le à la hauteur d'étage à franchir

\item {} 
\textbf{L} sur l'un des (long) coté, dessinez le profil (en ``escalier''!) représentant la face supérieure d'une seule contre-marche + marche

\item {} 
\textbf{M + CTRL} faites une copie multiple de ces deux segements : sélectionnez le pobas comme point de départ, et le point haut comme fin pour déterminer la longueur et l'orientation du vecteur de déplacement. Indiquez le nombre de copies (= nombre de marches), par exemple \sphinxcode{14x} (la technique de la copie multiple est dévleoppée ici {\hyperref[su/clonage:deplcmt\string-multipl]{\sphinxcrossref{\DUrole{std,std-ref}{Déplacement multiple/clonage}}}})

\item {} 
\textbf{P} extrudez le volume situé au-dessus du profil. Enjoy

\end{itemize}


\subsection{Composants}
\label{su/escalier:composants}
Cette méthode implique la création d'une marche, la transformation en composant, et la copie de celui-ci. C'est la technique la plus évoluée ... donc la moins évidente.
\begin{itemize}
\item {} 
faites une marche unique (rectangle --\textgreater{} groupe --\textgreater{} extrusion) ayant comme dimension la hauteur et le giron souhaité

\item {} 
transfomez-là en composant : \sphinxmenuselection{clic-droit sur le groupe \(\rightarrow\) créer un composant}

\item {} 
si vous ouvrez la fenêtre ``composant'', profitez-en pour le renommer en ``marche''

\item {} 
\textbf{M + CTRL} faites une copie multiple de ce composant, de l'angle bas vers l'angle haut (= vecteur déplacement), incrémenté du nombre de marche nécéssaires \sphinxcode{12x} par exemple.

\item {} 
\textbf{SPC} éditez n'importe quelle marche, vous verrez que les transformations s'opèrent aussi sur les autres ... c'est la magie des composant.

\item {} 
si vous voulez rendre une marche différente des autres composants : \sphinxmenuselection{clic-droit sur le composant \(\rightarrow\) rendre unique}

\end{itemize}


\section{Création de Scènes}
\label{su/creation-scenes:creation-scenes}\label{su/creation-scenes::doc}\label{su/creation-scenes:creation-de-scenes}
Sketchup nous permet de créer des ``Scènes'' ou ``Pages'', correspondant à des ``vues'' que l'on renregistrera. Le logiciel peut ensuit créer une ``animation'' : en partant de la première scène, il ``interpole / fait bouger'' le volume pour arriver à la deuxième, et ainsi de suite pour les scènes suivantes. On peut ensuite créer un ``film'' (export\textgreater{}animation)

Cela permet une foule de choses (étude d'ensoleillement/parcours des ombres, visualisation façade/façade, etc.).


\subsection{Ajout de scène}
\label{su/creation-scenes:ajout-de-scene}

\subsubsection{Première scène}
\label{su/creation-scenes:premiere-scene}
Créons une scène par le menu \sphinxmenuselection{Affichage \(\rightarrow\) animation \(\rightarrow\) ajouter une scène}

On aperçoit un onglet , sous les barres d'outils supérieures.


\subsubsection{Scènes suivantes}
\label{su/creation-scenes:scenes-suivantes}
L'ajout est possible de 2 façons. La plus riche en fonctionalités est la première :
\begin{enumerate}
\item {} 
gestionnaires de scènes (affichage par \sphinxmenuselection{fenêtre \(\rightarrow\) scènes} ou par clic-droit sur l'onglet de la scène et ``gestionnaire de scène'' )

\end{enumerate}
\begin{figure}[htbp]
\centering

\noindent\sphinxincludegraphics{{gestionnaire-scenes}.png}
\end{figure}

On clique sur le ``+'' pour créer d'autes scènes, et on les renomme de façon plus explicite dans la case ``Nom'' en \sphinxcode{Est}, \sphinxcode{ouest}, etc. (par exemple)
\begin{enumerate}
\item {} 
clic-droit sur l'onglet de la scène --\textgreater{} ajouter

\end{enumerate}

L'inconvénient est qu'on ne peut pas renommer les scènes


\subsection{Adaptation de la vue}
\label{su/creation-scenes:adaptation-de-la-vue}
Imaginons que notre modèle comporte 4 façades (E,O,N,S) et que nous venons de créer 4 scènes portant une appellation éponyme.

Plaçons-nous sur la scène ``Est'' en cliquant sur l'onglet.

Par le jeu des vues prédéfinies de Sketchup, affichons la façade Est du modèle

Faisons un affichage maximal de celle-ci (Zoom --\textgreater{} tout)

\textbf{Mémorisons cet affichage} : clic-droit sur l'onglet de la scène --\textgreater{} actualiser

Faisons de même pour les 3 autres façades.


\section{Export vers le dwg}
\label{su/export-dwg:export-vers-le-dwg}\label{su/export-dwg::doc}\label{su/export-dwg:export-dwg}
Un modèle 3D, c'est bien pour la visualisation, mais ça peut être pratique de l'exporter vers un logiciel de dessin plus abouti tel qu'AutoCAD. On pourra ainsi le dimensionner avec aisance, l'imprimer en pdf ou en papier à l'échelle, etc.

Sachant que l'export consiste à extraire une géométrie 2D d'un modèle tridimensionnel, Il existe 2 méthodes pour aboutir (à peu près) au même résultat :
\begin{figure}[htbp]
\centering

\noindent\sphinxincludegraphics{{ic_bo_vues}.png}
\end{figure}


\subsection{Vues prédéfinies (avant/arrière, cotés)}
\label{su/export-dwg:vues-predefinies-avant-arriere-cotes}\begin{itemize}
\item {} 
\sphinxmenuselection{caméra \(\rightarrow\) projection parrallèle} (aucune déformation)

\item {} 
choisir une vue prédéfinie, par exemple facade avant (correspond par exemple à la facade Ouest du modèle)

\item {} \begin{description}
\item[{\sphinxmenuselection{exporter \(\rightarrow\) graphique 2D ...}}] \leavevmode\begin{itemize}
\item {} 
type d'exportation : AutoCAD dwg

\item {} 
options (par bonheur, Sketchup mémorise les derniers paramètres : on ne recommencera pas pour l'export suivant!) :

\item {} 
échelle réelle 1:1

\item {} 
version : autocad 2007

\item {} 
lignes profilées : aucunes

\item {} 
arêtes prolongées : à décocher

\end{itemize}

\end{description}

\item {} 
enregistrez le fichier (idéalement, dans un sous-dossier ``xrefs'' relatif au dessin autocad regroupant toutes ces vues) en \sphinxcode{\textless{}projet\textgreater{}\_facade-ouest.dwg}. Pour une arborescence complète d'un projet, reportez-vous à cet {\hyperref[init_su+acad/demarrage:arborescence\string-projet]{\sphinxcrossref{\DUrole{std,std-ref}{exemple}}}}

\item {} 
choisissez une autre vue, et recommencez les étapes pour l'export

\end{itemize}

\begin{notice}{note}{Note:}
Il est préferable d'organiser votre dessin avec des scènes, lesquelles contiendront les vues prédéfinies de votre modèle. C'est plus PRO! Reportez-vous à {\hyperref[su/creation\string-scenes:creation\string-scenes]{\sphinxcrossref{\DUrole{std,std-ref}{Création de Scènes}}}} pour voir comment c'est possible.
\end{notice}

Avec la richesse des vues prédefinies enregistrées dans des scènes correspondantes, (=travail préalable) l'export vers AutoCAD est mieux contrôlé.
\begin{figure}[htbp]
\centering

\noindent\sphinxincludegraphics{{ic_coupe}.png}
\end{figure}


\subsection{Coupes}
\label{su/export-dwg:coupes}\label{su/export-dwg:export-dwg-coupes}
Il suffit de faire une coupe en
\begin{itemize}
\item {} 
cliquant sur l'icône ci-dessus

\item {} 
placer-là au bon endroit sur le modèle (vers + 1 m de hauteur pour faire un plan, etc.), (on peut bloquer son orientation en appuyant sur la touche \sphinxcode{maj} )

\item {} 
et de faire \sphinxmenuselection{fichier \(\rightarrow\) exporter \(\rightarrow\) tranche de section} (au format \sphinxcode{*.dwg})

\end{itemize}

Cette méthode à l'avantage d'être rapide, elle montre néanmoins les limites de Sketchup : l'export ne concerne que la ``peau'' extérieure de chaque volume (c'est normal, Sketchup est un modeleur 3D \textbf{surfacique}, les volumes sont ``creux'')

On peut remédier à cet inconvénient en installant le plugin \textbf{fr\_SectionCutFace.rb} que l'on peut trouver sur le site suivant \url{http://www.crai.archi.fr/RubyLibraryDepot/Ruby/FR\_arc\_page.htm} ou, pour plus de facilité, directement \sphinxcode{ici} .

Ce plugin, une fois installé (voir la méthode {\hyperref[su/install\string-plugin\string-su:install\string-plugin\string-su]{\sphinxcrossref{\DUrole{std,std-ref}{Installation de plugins}}}}), est uniquement activé lorsque l'on fait un \sphinxcode{clic-droit} sur la coupe et que l'on choisit l'option \textbf{\texttt{face de section}}. Cette commande ajoute une face entre les arêtes générées par la coupe (nécessite un contour continu pour créer des faces ``cohérentes'')


\section{FAQ Sketchup}
\label{su/faq_su:faq-sketchup}\label{su/faq_su::doc}\label{su/faq_su:faq-su}
C'est la liste de la ``Foire aux questions'' concernant les logiciels de dessin Sketchup et AutoCAD


\subsection{Import}
\label{su/faq_su:import}

\subsection{Export}
\label{su/faq_su:export}\begin{description}
\item[{... exporter une vue en projection orthonormée depuis Sketchup}] \leavevmode
\sphinxmenuselection{caméra \textgreater{} projection parallèle}
\sphinxmenuselection{exporter \textgreater{} graphique 2 D \textgreater{} extension ** *.dwg ** \textgreater{} options
(autocad 2007)}

\end{description}


\section{Glossaire Sketchup}
\label{su/glossaire_su:glossaire-su}\label{su/glossaire_su::doc}\label{su/glossaire_su:glossaire-sketchup}\begin{description}
\item[{Push-pull (tirer-pousser) raccourci\index{Push-pull (tirer-pousser) raccourci|textbf}}] \leavevmode\phantomsection\label{su/glossaire_su:term-push-pull-tirer-pousser-raccourci}
Cette fonction d'\emph{extrusion} a rendu Sketchup célèbre!

\item[{ZCV\index{ZCV|textbf}}] \leavevmode\phantomsection\label{su/glossaire_su:term-zcv}
Zone de Controle de Valeur : on y rentre les valeurs numériques des dimensions. Attention, pour des valeurs non entières, la séparation décimale \emph{,} est réalisée par la \sphinxcode{,} du clavier ``texte'' (le point ne représente rien). La séparation des coordonnées x, y, z est assuré par le \sphinxcode{;} . Pour un rectangle de 12.63 X 6.63 m , on inscrira \sphinxcode{12,63;6,63} .

Attention, ce n'est pas la même chose sur AutoCAD : pour le rectangle précédant, on inscrira les dimensions ainsi : \sphinxcode{12.63,6.63}

\end{description}


\chapter{AutoCAD}
\label{acad/index:index-acad}\label{acad/index::doc}\label{acad/index:autocad}
C'est le gros morceau! Certains aiment, d'autres moins, en tout cas,
c'est à ingérer, puis digérer pour avancer sans lourdeurs !
\begin{quote}
\phantomsection\label{acad/config_acad:config-acad}\end{quote}


\section{Configuration d'AutoCAD}
\label{acad/config_acad:configuration-d-autocad}\label{acad/config_acad::doc}\label{acad/config_acad:config-acad}
La configuration d'AutoCAD est un vaste sujet : nous aborderons les éléments qui nous paraissent essentiels, liberté à vous d'interpréter ces lignes pour personnaliser encore plus finement le logiciel à vos besoins/envies.

Nous pouvons discerner 2 volets : la configuration \emph{graphique} et la configuration \emph{sous le capot}. Est-il nécessaire de préciser que le second volet est de loin le plus important? Nous apercevrons aussi que les paramétrages peuvent être soit exécutés au travers de menus (parfois changeants selon les versions d'AutoCAD), ou soit directement en ligne de commande (bien plus rapide et ... logiquement ... identique quelquesoit les changements de version)

L'objectif du paramétrage est de disposer, lors du démarrage d'AutoCAD d'outils adaptés à un travail productif :
\begin{itemize}
\item {} 
dessin en mètre

\item {} 
styles de texte/cotes de hauteur fixe, quelquesoit l'échelle d'impression (1,5mm, 3mm et 5mm de hauteur)

\item {} 
styles de traits basiques (fin/moyen/fort, de couleur )

\item {} 
bloc dynamiques

\item {} 
palette d'outils personnalisés

\item {} 
liste de calques ()

\item {} 
etc.

\end{itemize}

\begin{notice}{note}{Note:}
\textbf{Pour les impatients}

Tous les réglages ci-dessous sont enregistrés dans un fichier \textbf{gabarit} \sphinxcode{mdl\_teb0910\_metre.dwt}, qui implique aussi une configuration d'imprimante, etc.

Vous pouvez télécharger \emph{directement} ces fichiers (gabarit+config imprimante) : \sphinxcode{ici} . L'archive contient aussi des instructions indiquant ou décompresser ces fichiers. Vous n'aurez plus qu'à indiquer le gabarit à utiliser par défaut au démarrage (comment ? : voir cette {\hyperref[acad/config_acad:chrgt\string-gabarit]{\sphinxcrossref{\DUrole{std,std-ref}{section}}}}) et vous pouvez vous éviter la (fastidieuse, si si) lecture qui suit ...
\end{notice}


\subsection{Aspect graphique}
\label{acad/config_acad:aspect-graphique}

\subsubsection{Accès rapide}
\label{acad/config_acad:acces-rapide}
On commence par le haut ...

La personalisation de cette barre d'outil appporte les fonctionalités des versions ``précédentes'' d'AutoCAD (avant la version 2009). C'est donc un must, car une grande quantité de documents on été édité avant mars 2008 (sortie d'AutoCAD 2009).
\begin{description}
\item[{Si vous cliquez sur la flèche (située à l'extrême droite de la barre d'outil) ``personnaliser la barre d'outils accès rapide'', une liste déroulante apparaîtra, et il faudra cliquer/cocher les éléments suivants :}] \leavevmode\begin{itemize}
\item {} 
copier les propriétés

\item {} 
aperçu du tracé

\item {} 
afficher la barre de menus

\end{itemize}

\end{description}


\subsubsection{Couleurs espace objet/ ligne de commande}
\label{acad/config_acad:couleurs-espace-objet-ligne-de-commande}
En exécutant le menu (visible après son activation dans la personalisation de la barre d'outil d'accès rapide) \sphinxmenuselection{Outils \(\rightarrow\) Options} ou la commande \textbf{\texttt{options}}, on affiche la fenêtre de paramétrage des options d'AutoCAD.

Allez sur l'onglet \textbf{Affichage}, puis cliquez sur le bouton \textbf{Couleurs}. Paramétrez comme suit (cela reste à votre convenance, ici, c'est pour l'example...) :

\noindent\begin{tabulary}{\linewidth}{|L|L|L|}
\hline

espace objet 2D
&
arrière plan uniforme
&
\textbf{Noir}
\\
\hline
ligne de commande
&
arrière plan uniforme
&
\textbf{Noir}
\\
\hline
ligne de commande
&
texte
&
\textbf{Blanc}
\\
\hline\end{tabulary}



\subsubsection{Espace de travail}
\label{acad/config_acad:espace-de-travail}
Choisissez \textbf{Dessin 2D et annotations} dans la liste déroulante (boîte en bas à droite).

Cet ``espace de travail'' regroupe un arrangement spécifique d'icônes et de pré-réglages. On aperçoit notamment le \textbf{ruban}, révolution dans l'interface apportée depuis AutoCAD 2009.

Attention, il faut spécifier que l'on désire enregistrer les modifications : cochez l'option ``enregistrer automatiquement les modifications des paramètres de l'espace de travail'', après avoir cliqué sur le nom ``Dessin 2D et annotations''


\subsubsection{Perte du Ruban : comment le retrouver}
\label{acad/config_acad:perte-du-ruban-comment-le-retrouver}
Faire la commande \textbf{\texttt{cui}} : la fenêtre de personalisation de l'espace de travail s'affiche.

Affichez le panneau droit de cette fenêtre en cliquant sur les doubles-chevrons.

Panneau de gauche : cliquez sur Espaces de travail --\textgreater{} Dessin 2D et annotations

Panneau de droite : cliquez sur Palettes --\textgreater{} Ruban (ou Ribbon), et plus bas mettez la valeur de Apparence --\textgreater{} Afficher à \emph{oui} .
\begin{figure}[htbp]
\centering
\capstart

\noindent\sphinxincludegraphics{{capt_config-acad_affichage-ruban}.png}
\caption{Affichage du ruban dans votre espace de travail}\label{acad/config_acad:id2}\end{figure}

\begin{sphinxShadowBox}
\textbf{Note importante}

\medskip


This section gives a quick summary of what is Mayavi, and should help
you understand where, in this manual, find relevent information to
your use case.
\end{sphinxShadowBox}


\subsubsection{Palette/Groupe de personalisé}
\label{acad/config_acad:palette-groupe-de-personalise}\begin{enumerate}
\item {} 
Création du groupe et des palettes personnalisées

\end{enumerate}

Une palette (pour en voir une : faites \sphinxcode{ctrl+3} pour afficher la palette ``outils'') peut être personalisée, pour ne contenir que les outils dont ont se sert le plus souvent (et que l'on a créé de toute pièce).

Cette personalisation ne peut intervenir qu'après avoir créé les outils de travail personlisés :
* styles de texte
* style de tableau
* blocs
* etc. : tout autre outil ``perso''.

Sur la palette : \sphinxcode{clic-droit} --\textgreater{} nouvelle palette (nom = \textbf{prénom\_txt}). On vient de créer un onglet, apte à recevoir les styles de texte personalisés.

\sphinxcode{Clic-droit} sur bord palette --\textgreater{} personnaliser les palettes

Dans le panneau droit de la fenêtre qui s'ouvre : \sphinxcode{clic-droit} --\textgreater{} nouveau groupe (nom : \textbf{prénom} ) . On vient de créer le panneau (= groupe de palettes) regroupant toutes nos palettes. Attention de bien placer ce ``groupe'' (icône en forme de dossier) à la racine de l'arborecence, et non sous un autre dossier!

Depuis le panneau de gauche, faites ``glisser'' une ou ds palettes dans le nouveau groupe situé dans le panneau droit : la palette \emph{prénom\_txt} par exemple ...

Fermez la fenêtre, et de retour sur la palette ``outils'', faites un \sphinxcode{clic-droit} sur le bord et sélectionez le groupe ``prénom'' (situé vers le milieu-bas de la fenêtre) : il n'y a plus que votre groupe ``prénom'' d'affiché, et il contient la palette ``prénom\_txt''

Un petit \sphinxcode{clic-droit} sur la barre verticale de la palette --\textgreater{} \textgreater{} ancrage à droite. La palette vient se coller sur le bord droit de l'écran. Si elle apparaît sous forme de barre : \sphinxcode{clic-droit} --\textgreater{} icône seulement.
\begin{enumerate}
\item {} 
Ajout d'outils existants dans ces palettes (personalisées ou non ...)

\end{enumerate}

Affichez le ``Design Center'' en faisant \sphinxcode{ctrl+2}. Si votre dessin contient des items à copier, ou un autre dessin, utilisez l'explorateur du panneau de gauche pour afficher le dessin voulu.

Depuis le panneau de droite du ``Design Center'' faites ``glisser'' les items vers le panneau de la palette qui doit le recevoir. C'est comme ça que je fais pour les blocs. Pour le reste, ce copier-coller ``graphique'''' marche plus ou moins bien ...


\sphinxstrong{Voir aussi:}


La configuration ``fine'' d'AutoCAD est contenue dans les fichiers de type \sphinxcode{*.cuix} . On peut le paramétrer directement avec la commande \textbf{\texttt{cui}}. Avis aux amateurs ...




\subsubsection{Barre d'état}
\label{acad/config_acad:barre-d-etat}
On finit par le bas ...

En cliquant sur la petite flèche noire, en bas à droite, ``menu barre d'état de l'application'', on va filtrer l'affichage/masquage de certains outils, jugés non nécessaires dans cet espace de travail (rappel : ce paramétrage est spécifique à l'espace de travail sélectionné, il sera différent du vore certainement ...)
\begin{description}
\item[{La liste suivante indique ce qui doit être coché :}] \leavevmode\begin{itemize}
\item {} 
coordonnées du curseur

\item {} \begin{description}
\item[{basculer l'état :}] \leavevmode\begin{itemize}
\item {} 
accrObj (F3)

\item {} 
reperObj (F11)

\item {} 
Épaisseur de ligne

\end{itemize}

\end{description}

\item {} 
papier / objet

\item {} 
pan

\item {} 
zoom

\item {} 
visibilité de l'annotation

\item {} 
échelle automatique

\end{itemize}

\end{description}


\subsection{Sous le capot}
\label{acad/config_acad:sous-le-capot}

\subsubsection{Unités d'insertion}
\label{acad/config_acad:unites-d-insertion}
L'unité d'insertion contrôle l'importation des blocs \& xref. Il faut que les 3 variables ci-dessous soient \emph{identiques} et soit en \emph{cm} (rentrer la valeur \textbf{5}) soit en \emph{m} (rentrer la valeur \textbf{6}).

Ceci fait, vous n'aurez plus à redimensionner les blocs lors de leur insertion dans votre dessin, si ils ont étés déssinés en m!

Pour les blocs dessinés en mm, ce qui est souvent le cas, il faudra leur appliquer un facteur d'échelle uniforme (x et y) de facteur 1000 pour les avoir de dimension cohérente avec le reste du dessin en mètre.

Certaines personnes vont plus loin : elle definissent l'unité d'insertion globale à 0 (sans unité) par la commande \textbf{\texttt{insunits : 0}}

et spécifient des unités variables pour l'insertion \textbf{\texttt{insunitsdefsource : 4}} (insertion en mm)

Ça commence à devenir compliqué ...

\noindent\begin{tabulary}{\linewidth}{|L|L|}
\hline

\textbf{Variable Système}
&
\textbf{Valeur}
\\
\hline
\textbf{\texttt{insunits}}
&
\textbf{6} (mètre)
\\
\hline
\textbf{\texttt{insunitsdefsource}}
&
\textbf{6}
\\
\hline
\textbf{\texttt{insunitsdeftarget}}
&
\textbf{6}
\\
\hline\end{tabulary}


\begin{notice}{note}{Note:}
Sketchup, bien que configuré pour dessiner en \emph{mètres}, règle l'unité d'insertion en \emph{pouces} lors d'un export en dwg! lorsqu'on fait un export 2d en \sphinxcode{*.dwg} depuis Sketchup, il faut éditer le fichier pour régler les unités d'insertion à 6 (pour un dessin en mètre) ou à 5 (dessin en cm), avec la commande \textbf{\texttt{insunits}}
\end{notice}


\subsubsection{Type de lignes}
\label{acad/config_acad:type-de-lignes}
Exécutez la commande \textbf{\texttt{measurement}}, et réglez ou vérifiez que la valeur à \textbf{1}. Cela charge le fichier \sphinxcode{acadiso.lin} qui correspond au type de ligne ``métrique''


\subsubsection{Unités}
\label{acad/config_acad:unites}
Exécutez la commande \textbf{\texttt{unites}}. (ou par le menu : \sphinxmenuselection{A \(\rightarrow\) utilitaires de dessin \(\rightarrow\) unités} )

Dans la fenêtre qui s'affiche, réglez les paramètres comme suit :

\noindent\begin{tabulary}{\linewidth}{|L|L|L|}
\hline

\textbf{Variable}
&
\textbf{type}
&
\textbf{valeur}
\\
\hline
\textbf{longueur}
&
type : decimal
&
précision : 0,00
\\
\hline
\textbf{angle}
&
type : décimal
&
précision : 0,00
\\
\hline
\textbf{échelle insertion}
&
(correspond à \textbf{\texttt{insunits}})
&
mètre
\\
\hline\end{tabulary}



\subsubsection{Calques}
\label{acad/config_acad:calques}
Il est bon d'avoir une série de calques prête à acueillir toute sorte de dessins, objets insérés, etc. Cette liste évolue au fur et à mesure des besoins, en fonction des projets/intervenants, etc.

Voici une liste minimale de calques :

\noindent\begin{tabulary}{\linewidth}{|L|L|L|L|}
\hline

\textbf{Calque}
&
\textbf{couleur}
&
\textbf{epaisseur ligne}
&
\textbf{impression}
\\
\hline
\textbf{brouillon}
&
magenta
&
default
&
\textbf{non}
\\
\hline
\textbf{trait\_fin}
&
jaune
&
0,25
&
oui
\\
\hline
\textbf{trait-moyen}
&
vert
&
0,50
&
oui
\\
\hline
\textbf{trait\_fort}
&
bleu
&
0,70
&
oui
\\
\hline
\textbf{cotes}
&
vert
&
defaut
&
oui
\\
\hline
\textbf{texte}
&
vert
&
defaut
&
oui
\\
\hline
\textbf{fenetre\_impression}
&
magenta
&
defaut
&
\textbf{non}
\\
\hline
\textbf{cartouche}
&
noir
&
defaut
&
oui
\\
\hline
\textbf{blocs}
&
noir
&
defaut
&
oui
\\
\hline
\textbf{images}
&
noir
&
defaut
&
oui
\\
\hline
\textbf{pdf}
&
noir
&
defaut
&
oui
\\
\hline
\textbf{xrefs}
&
noir
&
defaut
&
oui
\\
\hline
\textbf{objet/niveau ...}
&
...
&
...
&
...
\\
\hline\end{tabulary}


Vous remarquerez que les 2 seuls calques non imprimables sont dans la même couleur magenta


\subsubsection{Style de texte}
\label{acad/config_acad:style-de-texte}
Utilisons une nouveauté introduite dès AutoCAD 2008 : le style de texte \textbf{annotatif}.

Le concept est très simple, et rejoint dans ce sens les autres logiciels de dessin architecturaux : on définit une hauteur de texte pour l'impression (pdf ou papier) et le logiciel fait le reste, pour adapter le texte à cette hauteur, indépendemment de l'échelle.

Un texte annotatif paramétré à 3 mm, sera à 3 mm de haut lorsque imprimé, que cela soit dans une fenêtre à l'échelle 1/50e ou 1/10e!

Affichage de la fenêtre ``Style de texte'' : par la barre de menus \sphinxmenuselection{format \(\rightarrow\) style de texte} ou par la ligne de commande \textbf{\texttt{style}}
\begin{figure}[htbp]
\centering

\noindent\sphinxincludegraphics{{style_txt-1.5mm-anot}.png}
\end{figure}
\begin{enumerate}
\item {} \begin{description}
\item[{Style de texte 1,5 mm de hauteur (cotation)}] \leavevmode\begin{itemize}
\item {} 
clic sur le style de texte \textbf{Annotatif} dans le panneau gauche

\item {} 
choix de la police : \textbf{Tecnic} (cette police ne possède que la fome peite et ``grande'' majuscule. Elle simule l'écriture au Rotring, et a cette hauteur, reste lisible puisque en petite majuscules)

\item {} 
style : standard (il vaut mieux éviter les styles exotiques, c'est pour du dessin technique ...)

\item {} 
taille : \textbf{annotatif}

\item {} 
hauteur texte : \textbf{1.5} (correspond à 1,5 mm en espace papier)

\item {} 
facteur de largeur : 1

\item {} 
angle oblique : 0

\item {} 
enregistrez ce style avec une appellation explicite : \sphinxcode{1.5mm\_txt-anot} par exemple

\item {} 
rendez ce style courant

\end{itemize}

\end{description}

\item {} \begin{description}
\item[{Style de texte de hauteur 3 mm (textes généraux)}] \leavevmode\begin{itemize}
\item {} 
choisissez la police \textbf{Tahoma} (permet la minuscule, mais reste assez étroite), avec un style standard

\item {} 
la hauteur sera réglé à \textbf{3}

\item {} 
le reste est inchangé (ne pas rendre ce style courant!)

\end{itemize}

\end{description}

\item {} \begin{description}
\item[{Style de texte de hauteur 5 mm (titres)}] \leavevmode\begin{itemize}
\item {} 
police \textbf{Arial} en style standard (plus large que Tahoma)

\item {} 
hauteur \textbf{5}

\end{itemize}

\end{description}

\end{enumerate}


\subsubsection{Style de Cotes}
\label{acad/config_acad:style-de-cotes}
On adaptera le style de cote pour qu'il utilise le style de texte annotatif défini précedemment.

Affichage de la fenêtre de configuration : choisissez le plus court! \textbf{\texttt{cotstyle}}
\begin{figure}[htbp]
\centering

\noindent\sphinxincludegraphics{{cotstyle_00}.png}
\end{figure}
\begin{enumerate}
\item {} 
modification d'un style \textbf{Annotatif} existant :

\end{enumerate}
\begin{figure}[htbp]
\centering

\noindent\sphinxincludegraphics{{cotstyle_0}.png}
\end{figure}
\begin{enumerate}
\setcounter{enumi}{1}
\item {} 
modification des lignes (blocage de la longueur des lignes de rappel à 5 mm)

\end{enumerate}
\begin{figure}[htbp]
\centering

\noindent\sphinxincludegraphics{{cotstyle_1}.png}
\end{figure}
\begin{enumerate}
\setcounter{enumi}{2}
\item {} 
modification des symboles et flèches (utilisation du style ``architectural'')

\end{enumerate}
\begin{figure}[htbp]
\centering

\noindent\sphinxincludegraphics{{cotstyle_2}.png}
\end{figure}
\begin{enumerate}
\setcounter{enumi}{3}
\item {} 
modification du texte de la cote (utilisation du style de texte annotatif de hauteur 1.5 mm : c'est petit mais ça passe partout)

\end{enumerate}
\begin{figure}[htbp]
\centering

\noindent\sphinxincludegraphics{{cotstyle_3}.png}
\end{figure}
\begin{enumerate}
\setcounter{enumi}{4}
\item {} 
modification des unités principales 2 zéros pour les longueurs et les angles, et les longueurs inférieures au mètre seront comptées en cm)

\end{enumerate}
\begin{figure}[htbp]
\centering

\noindent\sphinxincludegraphics{{cotstyle_4}.png}
\end{figure}
\begin{enumerate}
\setcounter{enumi}{5}
\item {} 
N'oubliez pas d'enregistrer le style de cote sous un nom explicite, qui rappelle le style de texte par exemple ...

\item {} 
Il ne vous reste plus qu'à créer un deuxième style, employant un texte plus grand (3 mm ?), et à l'enregistrer sous le nom \sphinxcode{3mm\_cot-anot} par exemple.

\end{enumerate}

Avec ce réglage, lorsque vous coterez des éléments d'une longueur inférieure au mètre, la valeur de la cote sera en cm. C'est pratique pour coter les cloisons d'une construction dessinée en m : les épaisseurs apparaîtront en cm, comme sur cette capture d'écran :
\begin{figure}[htbp]
\centering

\noindent\sphinxincludegraphics{{cote_m+cm}.png}
\end{figure}


\subsubsection{Style de tableau}
\label{acad/config_acad:style-de-tableau}
Par le menu \sphinxmenuselection{Format \(\rightarrow\) style de tableau} on affiche la fenêtre de configuration des styles de tableau.

Créons un nouveau style, qui utilisera les styles de texte annotatifs précédents, dans les hauteurs de 5 mm pour le titre, 3 mm pour les en-têtes (=sous titres) et 1.5 mm pour les cellules de données.
\begin{figure}[htbp]
\centering

\noindent\sphinxincludegraphics{{tablo-styles_titre+entete+donnees}.png}
\end{figure}

On peut faire extrêmement ``joli'' pour les tableaux, mais cela dépasse le cadre d'une initiation ...

N'oublions pas d'enregistrer ce nouveau style de tableau sous une appellation différente de standard, et de le rendre courant.


\subsubsection{Style de points}
\label{acad/config_acad:style-de-points}
Choisir la croix inclinée à 45 deg.


\subsubsection{Style de repères multiples}
\label{acad/config_acad:style-de-reperes-multiples}
Créeez un nouveau style faisant appel à du texte annotatif de 3 mm de hauteur.

les variantes à ce nouveau style sont dans
\begin{itemize}
\item {} \begin{description}
\item[{le type de ligne de repère :}] \leavevmode\begin{itemize}
\item {} 
en ligne brisée (il est même possible de spécifier le nombre de segments et l'inclinaison des segments de façon constante)

\item {} 
en \emph{spline} (il est possible de spécifier le nombre de points de construction)

\end{itemize}

\end{description}

\item {} \begin{description}
\item[{le type de contenu :}] \leavevmode\begin{itemize}
\item {} 
texte multiligne

\item {} 
ou bloc (pour faire une étiquette), dont il faut régler l'échelle d'insertion (un facteur compris entre 50 et 100 semble correct (cas d'un dessin ``pensé'' en m, et réglé à 6 (mètres) pour l'insertion de blocs))

\end{itemize}

\end{description}

\end{itemize}

C'est le moment de créer plusieurs styles de repère multiples et de les enregistrer sous une appellation permettant de les différencier!
\begin{figure}[htbp]
\centering

\noindent\sphinxincludegraphics{{styles_reper-multiples}.png}
\end{figure}

La pratique vous fera déterminer lequel devra être courant!


\subsubsection{Imprimantes}
\label{acad/config_acad:imprimantes}
Nous pouvons configurer une imprimante par défaut. Cette configuration reste néammoins dépendante de l'ordinateur qui accueille le dessin.

\begin{notice}{note}{Note:}
** Imprimantes matérielles et imprimantes système **

Vous avez 2 types d'imprimantes dans votre ordinateur : celles qui se voient et les autres. Nous utiliserons le 2ème type en premier (imprimante \sphinxcode{pdf}) et le premier en dernier (imprimante A3+A4 ``HP OfficeJet K8600'') pour l'impression du pdf sur le papier.
\end{notice}

Le travail de création/modification d'une imprimante fait partie de la configuration de la mise en page : {\hyperref[acad/config_acad:mise\string-en\string-page]{\sphinxcrossref{\DUrole{std,std-ref}{voir ci-dessous}}}} pour la configuration de l'imprimante système \sphinxcode{DWG To PDF}.
\begin{description}
\item[{L'impression fait partie intégrante de la stratégie de sauvegarde :}] \leavevmode\begin{itemize}
\item {} 
\emph{si le fichier informatique est perdu il reste l'impression ...} : on peut étendre cet adage \emph{tibétain} et renforcer son application en ...

\item {} 
si l'impression est effectuée vers un format de visualisation, on pourra toujours voir le fichier (sortie \sphinxcode{*.pdf}), sans avoir besoin de l'application d'origine : pratique

\end{itemize}

\end{description}


\subsubsection{Mise en page}
\label{acad/config_acad:mise-en-page}\label{acad/config_acad:id1}
Mise en page \textbf{nommée} : plutôt que de configurer la présentation (choix imprimant, format papier, etc.) de façon \emph{unique}, il vaut mieux en créer une de façon \emph{générique}.
\begin{description}
\item[{Cette méthode de création d'une mise en page permet :}] \leavevmode\begin{itemize}
\item {} 
d'appliquer cette mise en page ``nommée'' à autant de présentations que l'on souhaite

\item {} 
d'importer des mises en pages nommées provenant d'autre dessins.

\end{itemize}

\item[{Affichons la fenêtre de création/modification des mises en page :}] \leavevmode\begin{itemize}
\item {} 
soit par le menu AutoCAD 2010 \textbf{\texttt{A -{-}\textgreater{} imprimer -{-}\textgreater{} Mise en page}} ,

\item {} 
soit par la barre de menus : \textbf{\texttt{Fichier -{-}\textgreater{} gestionnaires des mises en pages}}

\item {} 
soit en activant une présentation (en faisant un \sphinxcode{clic-gauche} sur l'onglet d'une présentation) puis en faisant un \sphinxcode{clic-droit} --\textgreater{} gestionnaire des mises en pages

\end{itemize}

\item[{Nouvelle mise en page ``nommée'' :}] \leavevmode\begin{itemize}
\item {} 
cliquez sur le bouton \emph{nouveau}

\item {} 
nommez cette nouvelle mise en page, par exemple \sphinxcode{A3H\_pdf}

\end{itemize}

\item[{Configuration mise en page :}] \leavevmode\begin{itemize}
\item {} \begin{description}
\item[{choix imprimante \sphinxcode{DWG To PDF}, et cliquez sur ``configuration'' :}] \leavevmode\begin{itemize}
\item {} \begin{description}
\item[{formats de papier :}] \leavevmode\begin{itemize}
\item {} 
filtrage des formats, on déselectionne tout

\item {} 
puis on sélectionne uniquement les formats que l'on souhaite utiliser

\end{itemize}

\end{description}

\item {} \begin{description}
\item[{marges :}] \leavevmode\begin{itemize}
\item {} 
réglons les marges des formats de papier sélectionnés à 10 mm, pour chaque coté

\item {} 
on doit effectuer ces réglages pour chaque format de papier

\end{itemize}

\end{description}

\end{itemize}

\end{description}

\item {} 
enregistrement sous un nouveau nom de fichier \sphinxcode{*.pc3}, par exemple \sphinxcode{DWG\_To\_PDF\_A0+A1+A2+A3+A4\_marges-10mm}, qui permet de savoir immédiatement quel est son usage ...

\end{itemize}

\end{description}


\subsubsection{Présentations}
\label{acad/config_acad:presentations}
Autrement appelé ``Espace Papier'', elles correspondent aux impressions papier.

Il est utile de préconfigurer celle-ci pour disposer de feuilles vierges, aux formats d'impression courants (A4 portrait, A3 paysage, A2 portrait, A1 paysage, A0 et doubleA0).

Ces feuilles contiendront un cartouche (on essaiera d'y inclure des textes ``automatiques'' via des ``champs'', mais on n'utilisera pas la fonction d'attribut de bloc, qui permet d'ajouter du texte selon une suite de questions/réponses, car ce système est lourd à gérer).

Pour les imprimer, il faudra configurer une imprimante, système (\sphinxcode{PDF TO DWG}, excellente sur AutoCAD 2010, pdfcreator, etc.) ou matérielle (traceur Océ, HP A3 grand public ...), et d'autres paramètres, vu à cette {\hyperref[acad/config_acad:mise\string-en\string-page]{\sphinxcrossref{\DUrole{std,std-ref}{section}}}}.
\begin{description}
\item[{Rotation d'une vue dans la présentation :}] \leavevmode\begin{itemize}
\item {} 
déplacez-vous en espace papier, en cliquant sur l'onglet de la présentation où vous désirez faire un rotation (typiquement : les façades)

\item {} 
activez le mode objet, en faisant un \sphinxcode{double-clic} à l'intérieur de la fenêtre d'impression, en en rentrant la commande \textbf{\texttt{eo}}

\item {} 
faites la commande \textbf{\texttt{vuedyn}}

\item {} 
sélectionnez les objets que vous voulez faire touner

\item {} 
entrez \textbf{\texttt{b}} (pour ``Basculer'')

\item {} 
et appliquez une rotation de 90

\end{itemize}

\end{description}

Attention : cette manipulation ne marche que si vous avez ectivé le mode objet en présentation. Attention à la molette de la souris dans ce mode : une simple pression modifie l'échelle du dessin!
\begin{description}
\item[{Masquage d'un calque sur une présentation spécifique (en le laissant affiché sur les autres) :}] \leavevmode\begin{itemize}
\item {} 
déplacez-vous en espace papier, en cliquant sur l'onglet de la présentation ou vous voulez masquer un calque

\item {} 
activez le mode objet, en faisant un \sphinxcode{double-clic} à l'intérieur de la fenêtre d'impression, en en rentrant la commande \textbf{\texttt{eo}}

\item {} 
affichez la palette du gestionnaire des calques

\item {} 
cliquez sur \textbf{gel de la fenêtre}

\end{itemize}

\end{description}
\begin{figure}[htbp]
\centering

\noindent\sphinxincludegraphics{{gel-calque_en-presentation_00}.png}
\end{figure}
\begin{figure}[htbp]
\centering

\noindent\sphinxincludegraphics{{gel-calque_en-presentation_01}.png}
\end{figure}
\begin{figure}[htbp]
\centering

\noindent\sphinxincludegraphics{{gel-calque_en-presentation_02}.png}
\end{figure}
\begin{figure}[htbp]
\centering

\noindent\sphinxincludegraphics{{gel-calque_en-presentation_03}.png}
\end{figure}

Attention : cette manipulation ne marche que si vous avez activé le mode objet en présentation!.

Faites aussi attention à la molette de la souris dans ce mode objet : une simple pression modifie l'échelle du dessin!
\begin{description}
\item[{Pour vérifier l'échelle, lorsque vous avez fini :}] \leavevmode\begin{itemize}
\item {} 
repassez en mode papier (en faisant un \sphinxcode{double-clic} à l'extérieur de la fenêtre ou en tapant la commande \textbf{\texttt{ep}} )

\item {} 
cliquez sur le bord de la fenêtre d'impression (doit être de couleur magenta si vous l'avez placée dans le bon calque ...)

\item {} 
vérifiez l'échelle en regardant ce qui est affiché dans la boîte ``échelles'' (en bas à droite)

\end{itemize}

\end{description}


\subsubsection{Échelles}
\label{acad/config_acad:echelles}
La liste d'échelles  est longue, mais peu utile pour un dessin en mètres ou en cm : il faut la nettoyer de toutes les echelles ``impériales'' et ajouter les échelles pour le dessin architectural (mètre et centimètre)


\subsubsection{Cartouche}
\label{acad/config_acad:cartouche}
Une présentation avec un cartouche tout prêt, c'est bien.


\subsubsection{Fichiers}
\label{acad/config_acad:fichiers}
Affichez la fenêtres ``options'' en tapant la commande éponyme \textbf{\texttt{options}}
\begin{description}
\item[{Type de fichiers (version par défaut) :}] \leavevmode
Qui possède AutoCAD 2010 aujourd'hui? Indiquons lui d'enregistrer les fichiers au format AutoCAD 2004 par défaut. On pourra ainsi travailler avec AutoCAD 2010 au quotidien, et échanger nos fichiers avec les utilisateurs d'AutoCAD de la version 2004 à 2010 ...

\end{description}
\phantomsection\label{acad/config_acad:chrgt-gabarit}\begin{description}
\item[{Chargement du gabarit :}] \leavevmode
Indiquez à AutoCAD de charger le gabarit que vous venez de créer : \sphinxcode{mdl\_teb0910\_metre.dwt} et indiquons dans \sphinxmenuselection{outils \(\rightarrow\) options \(\rightarrow\) fichiers} le fichier modèle à charger au démarrage.

\end{description}


\section{Autocad}
\label{acad/acad:autocad}\label{acad/acad:acad}\label{acad/acad::doc}

\subsection{Import des fichiers générés par Sketchup}
\label{acad/acad:import-des-fichiers-generes-par-sketchup}
Maintenant qeu Sketchup a crée des fichiers *.dwg, on peut les ouvrir avec AutoCAD. Cela fait donc \textbf{7 fichiers} à éditer séparément.

Nous pouvons aussi copier ces fichiers dans un nouveau dessin regroupant tous les exports de Sketchup. Pour ce faire, nous avons 2 choix possibles :
* copier-coller
* \textbf{xref}


\section{FAQ AutoCAD}
\label{acad/faq_acad:faq-acad}\label{acad/faq_acad:faq-autocad}\label{acad/faq_acad::doc}
C'est la liste de la ``Foire aux questions'' concernant les logiciels de dessin Sketchup et AutoCAD


\subsection{Comment faire pour ...}
\label{acad/faq_acad:comment-faire-pour}\begin{description}
\item[{... imprimer à l'échelle?}] \leavevmode
connaitre l'unité de dessin : on a ``pensé'' en mètre, centimètre : il suffit de coter un élément, soit il fait 2,48, soit 248
appliquer un facteur d'échelle:
\begin{quote}

dans le cas du mètre : 1000 (nombre de mm corrrespondant à 1 mètre) divisé par échelle souhaitée (exprimée en nombre entier)
ex :  1000/50 pour obtenir un dessin au 1/50e (1000/50= 20 --\textgreater{} ``facteur d'échelle'')
vous pouvez rentrer ces valeurs en ligne de commande :
\begin{quote}

en espace-papier, double-clic sur la fenêtre : vous passez en espace objet.
ligne de commande : z +  entrée, puis soit 1000/50xp, soit 20xp + entrée
\end{quote}

ou directement dans la barre d'outil ``fenêtre''
\end{quote}

\end{description}


\section{Glossaire AutoCAD}
\label{acad/glossaire_acad::doc}\label{acad/glossaire_acad:glossaire-acad}\label{acad/glossaire_acad:glossaire-autocad}\begin{description}
\item[{dwg\index{dwg|textbf}}] \leavevmode\phantomsection\label{acad/glossaire_acad:term-dwg}
format de fichier par défaut d'AutoCAD.
le format de fichier dwg change tous les 3 ans. Ainsi, avec
Autocad 2007, on peut ouvrir tous les formats dwg des versions
antérieures, mais aussi (fait rare ...) les fichiers acad
2007 + 2008 + 2009. Avec la version AutoCAD 2010, un nouveau
format dwg apparaît. On pourra ouvrir les fichiers dwg
jusqu'à AutoCAD 2013, ainsi de suite ...

\end{description}


\chapter{Photoshop}
\label{psd/index:photoshop}\label{psd/index::doc}\label{psd/index:index-ptshp}

\section{Configuration de Photoshop}
\label{psd/config_psd::doc}\label{psd/config_psd:configuration-de-photoshop}

\subsection{Mémoire vive}
\label{psd/config_psd:memoire-vive}
Edition \textgreater{} Préférences \textgreater{} Mémoire et mémoire cache


\subsection{Disques de travail}
\label{psd/config_psd:disques-de-travail}
Edition \textgreater{} Préférences \textgreater{} Modules externes et disques de travail


\subsection{Réglage de l'écran}
\label{psd/config_psd:reglage-de-l-ecran}
C:Program FilesFichiers communsAdobeCalibration


\subsection{Préférences}
\label{psd/config_psd:preferences}
Edition \textgreater{} Préférences \textgreater{} Général


\subsection{Espace de travail}
\label{psd/config_psd:espace-de-travail}

\section{Détourage d'objets, personnages, etc.}
\label{psd/detourage:detourage-d-objets-personnages-etc}\label{psd/detourage::doc}
Vous pouvez créer votre propre bibliothèque d'objets et de personnages à insérer dans vous perspectives, etc.
\begin{itemize}
\item {} 
commencez par prendre des photos (ou faites des scans) de personnes ou d'objets que vous voulez inclure dans vos compositions.

\item {} 
une fois ce travail réalisé, vous devez extraire les objets du premier plan du fond de l'image pour les utiliser en tant qu'entourage.

\end{itemize}

Extraire des personnes d'une photo : il faut s'arranger pour prendre des photos de personnes en face d'un mur, ou d'un fond net qui permet d'identifier facilement ce qui ressort du personnage ou du fond.


\subsection{Utilisation de l'outil ``extraction''}
\label{psd/detourage:utilisation-de-l-outil-extraction}
Commençons par séparer grossièrement le personnage et le sol en utilisant l'outil \emph{extraire}

Ouvrez l'image, que vous pouvez télécharger directement \sphinxcode{ici} , montrant le sujet (une femme en mouvement) et le fond (l'artère d'une grande ville).

Pour commencer le processus d'extraction, choisissez \sphinxmenuselection{filtre \(\rightarrow\) extraire} ou pressez \sphinxcode{Alt+Ctrl+X}.

Vous aller séparer le premier plan de l'arrière plan en utilisant les outils dans cette boîte de dialogue. Familiarisez-vous avec ces outils.

Vous aller commencer par tracer un contour autour de la personne en utilisant le \sphinxtitleref{Sélecteur de contour}.

Une fois que la détection de bord forme un contour fermé, vous utiliserez l'outil \sphinxtitleref{Remplissage} pour désigner l'objet de premier plan.
\begin{itemize}
\item {} 
Sélectionnez l'outil zoom ou pressez sur \sphinxcode{Z}, et cliquez quelques fois pour grossir la zone correspondant à la femme en plaçant la loupe sur la zone désirée : nous commencerons par les pieds et les jambes.

\item {} 
Vous pouvez déplacer l'image obtenue avec l'outil main, en pressant \sphinxcode{H} ou \sphinxcode{barre espace}

\item {} 
Cliquez sur l'outil \sphinxtitleref{Sélecteur de contour} ou pressez \sphinxcode{B} pour le sélectionner.

\item {} 
Cochez \emph{Sélection optimisée} dans les options d'outils sur la droite. Utilisez \emph{Sélection optimisée} pour permettre à Photoshop de sélectionner une taille de brosse appropriée quand vous tracez au-dessus de \emph{bords biens définis}.

\end{itemize}

Image
\begin{itemize}
\item {} 
Tracez précautionneusement les bords des pieds et des jambes de la femme. Une surbrillance en vert apparaîtra sur l'écran lorsque vous tracerez. La propriété d'ajustement automatique de la taille de brosse permet une bonne couverture du bord. Idéalement, vous devriez obtenir une couverture de la moitié du fond et de la moitié de l'avant-plan sur les bords que vous tracez.

\end{itemize}

Image
\begin{itemize}
\item {} 
Occasionnellement, dans les zones de fort contraste, ou lorsque les bords sont flous, la \emph{Sélection optimisée} suit le mauvais chemin. Si vous faite une erreur, utiliser l'outil gomme (``eraser'' en anglais) pour corriger le tracé : Pressez \sphinxcode{E} et enlevez tout segment inadéquat.

\item {} 
Pressez \sphinxcode{B} et continuez de révéler les bords bien définis de la jambe et de la chaussure.

\item {} 
Continuez jusqu'à ce que tous les contours soit réalisés dans cette zone. Déplacez l'image en pressant sur \sphinxcode{H} ou sur \sphinxcode{barre espace}, puis recommencer le travail de sélection de contour en pressant \sphinxcode{B}

\item {} 
Si la zone à sélectionner comporte des trous, zoomez en pressant \sphinxcode{Z} et cliquez, éventuellement, diminuez la taille de la brosse, puis pressez \sphinxcode{B} pour faire un contour fin.

\item {} 
Pour revenir au zoom précédant, (zoom négatif) pressez sur \sphinxcode{Z}, et appuyer sur la touche \sphinxtitleref{Alt} avant de cliquer sur la zone (c'est l'équivalent de la combinaison \sphinxcode{Alt+Z}).

\item {} 
Déplacez l'image en appuyant sur la barre d'espace et continuez à sélectionner le contour. Lorsque celui-ci est flou, dé-cochez l'option Sélection optimisée.

\item {} 
Lorsque vous arrivez sur les cheveux, décochez la sélection optimisée de la brosse et tracez le contour de la tête en faisant bien attention de couvrir la moitié des cheveux et la moitié du fond. Lorsque vous avez fini avec cette zone à bord flous, re-sélectionnez \emph{sélection optimisée}

\end{itemize}

Image
\begin{itemize}
\item {} 
Finissez le contour complètement, et zoomez pour voir le personnage en entier ; pressez \sphinxcode{Z}, puis \sphinxcode{Alt} et cliquer sur la femme, plusieurs fois si nécessaire, en gardant en tête que le point de clic est le centre du ``zoomage''.

\end{itemize}

Image
\begin{itemize}
\item {} 
Utilisez l'outil \sphinxcode{Remplir} ou pressez \sphinxcode{G} : le personnage doit être recouvert de bleu. Attention : si l'image entière est bleue, c'est que le contour n'est pas fermé!

\end{itemize}

Image
\begin{itemize}
\item {} 
Cliquez sur \sphinxcode{aperçu} pour accéder aux outils de correction du contour. On voit de nombreux défauts, plus ou moins importants, en fonction de la qualité du tracé.

\end{itemize}

Image
\begin{itemize}
\item {} 
Faites un zoom sur le pied droit : pressez \sphinxcode{Z} et tracez une fenêtre autour de ce pied.

\item {} 
Cliquez sur outil de nettoyage ou pressez \sphinxcode{C} : cet outil enlève l'opacité (ou efface!) du premier plan, mais si on appuie sur \sphinxcode{Alt}, il en rajoute (la touche Alt appelle l'opposé d'une commande), c.a.d qu'il rajoute des portions de l'image sur celle qui vient d'être détourée.

\end{itemize}

Image
\begin{itemize}
\item {} 
Alternativement, et ceci sur tout le contour, vous allez \emph{rajouter du fond} (pressez \sphinxcode{C}, puis \sphinxcode{Alt} et \sphinxcode{cliquez}) et effacer plus finement celui-ci (vous avez déjà pressé sur C, il ne vous reste plus qu'a cliquer) : veillez à effectuer le bon niveau de zoom et adapter la taille de la brosse en conséquence (fort grossissement =\textgreater{} petite brosse)

\item {} 
Si vous remarquez des bords flous, activez l'outil \sphinxcode{Correction de bord} ou pressez \sphinxcode{T}, et tracez le long des bords à rendre plus net. Attention, n'utilisez pas cet outil sur des contours naturellement flous, comme les cheveux!

\item {} 
Lorsque vous êtes satisfait du résultat, cliquez sur \sphinxcode{OK}. La boîte de dialogue d'extraction se ferme, en laissant apparaître uniquement le personnage sélectionné.

\end{itemize}


\subsection{Correction des défauts}
\label{psd/detourage:correction-des-defauts}
Vous avez terminé la première étape d'extraction. L'image obtenue a besoin d'être finement nettoyée pour prétendre au fait d'être réutilisée comme personnage à insérer dans vos compositions.

L'unique calque, contenant l'image détourée grossièrement, a été automatiquement converti en calque de fond.
\begin{itemize}
\item {} 
Double-cliquez sur l'icône dans la fenêtre des calques et renommez celui-ci en \sphinxtitleref{femme qui marche}.

\item {} 
Créez un nouveau calque en cliquant sur le bouton situé en bas et renommez le \sphinxtitleref{fond}.

\item {} 
Faites glisser ce calque \sphinxtitleref{en-dessous} du calque \sphinxtitleref{femme qui marche}.

\item {} 
Pressez \sphinxcode{D} pour mettre les couleurs par défaut (noir au premier plan, blanc en arrière-plan)

\item {} 
Cliquez sur l'icône du calque fond et pressez su \sphinxcode{Alt+Retour arrière} pour remplir le calque courant avec la couleur de premier plan. Le fond devient noir, et on aperçoit des \sphinxtitleref{défauts}.

\end{itemize}

Image
\begin{itemize}
\item {} 
Cliquez sur le calque \sphinxtitleref{femme qui marche}, et sélectionnez l'outil \sphinxmenuselection{Gomme} en cliquant sur son icône à gauche, ou en pressant \sphinxcode{E}.

\end{itemize}

(im 26 27).

Choisissez une brosse douce de diamètre adapté en fonction du zoom, et effacez les points couleurs qui apparaissent sur le fond noir, avec précaution!

Si vous avez été trop loin, vous pouvez afficher le panneau d'historique, et supprimer les dernières opérations.

Pour annuler la dernière opération seulement, pressez \sphinxcode{Ctrl+Z}.

Vous pouvez aussi utiliser la brosse d'historique, et peindre sur les zones effacées involontairement.
\begin{itemize}
\item {} 
Vérifions le travail de nettoyage sur un fond blanc :
\begin{itemize}
\item {} 
Cliquez sur le calque \sphinxcode{fond}.

\item {} 
Pressez \sphinxcode{D} pour établir les couleurs par défaut.

\item {} 
Cliquez sur \sphinxcode{Ctrl+Retour arrière} pour remplir le calque avec la couleur d'arrière-plan.

\item {} 
Pressez \sphinxcode{E} pour sélectionner la gomme, cliquez sur le calque \emph{femme qui marche} et effacez soigneusement les bords mal finis.

\end{itemize}

\item {} 
Une fois le travail de nettoyage manuel fini, nous allons encore améliorer la qualité de l'image en utilisant les filtres de Photoshop :
\begin{itemize}
\item {} 
Sélectionnez le menu \sphinxmenuselection{Filtres \(\rightarrow\) Renforcement \(\rightarrow\) Netteté optimisée}.

\item {} 
Faites glisser la règle à 100\%, choisissez un rayon de 2 pixels.

\item {} 
Cliquez sur \sphinxcode{OK}.

\end{itemize}

\end{itemize}


\subsection{Sauvegarde de l'extraction dans un canal alpha}
\label{psd/detourage:sauvegarde-de-l-extraction-dans-un-canal-alpha}
Les images peuvent posséder un quatrième canal, le canal \emph{alpha}, en plus des trois canaux Rouge, Vert et Bleu.

Ce canal ne modifie pas les couleurs de l'image et sert dans la plupart des cas à gérer la transparence de l'image par exemple pour permettre de voir ce qu'il y a derrière l'image.

Les formats \sphinxcode{*.gif}, \sphinxcode{*.png}, \sphinxcode{*.tiff}, \sphinxcode{*.tga}, peuvent supporter un canal alpha. Dans les applications 3D, l'emploi du canal alpha sur les textures permet par exemple de modifier leur réflection ou leur transparence. Les canaux sont dénommés \emph{couches} dans Photoshop.

Nous avons extrait l'image et renforcé ses contours. Il nous reste à créer une couche alpha, qui est une couche de ton gris montrant ou l'image est opaque, translucide et ou transparente. On utilise l'outil opérations pour convertir la transparence en une couche alpha.
\begin{itemize}
\item {} 
Sélectionnez le menu \sphinxmenuselection{Image \(\rightarrow\) opérations} et réglez le comme sur la capture d'écran :
\begin{itemize}
\item {} 
dans le groupe Source1, sélectionnez le calque \sphinxcode{femme qui marche} et juste en dessous, choisissez la couche \sphinxcode{transparence}

\item {} 
dans le groupe Source2, sélectionnez le même calque, et laisser le rouge comme couche.

\item {} 
Sélectionnez un mode fusion \sphinxcode{Normal}

\item {} 
cliquez sur \sphinxcode{OK}.

\end{itemize}

\item {} 
Dans la palette des couches (sélectionnez l'onglet à coté de l'onglet des calques) de la , il y a une nouvelle couche dénommée \sphinxcode{Alpha 1}. cliquez sur la couche RVB pour faire réapparaître l'image normale.

\item {} 
Aplatissez l'image : faites apparaître le panneau des calques. Faites un clic droit dessus et cliquez sur \sphinxcode{aplatir l'image} (im 32). N'oubliez pas de remplir le calque de fond avec une couleur blanche! (clic sur le calque \sphinxcode{fond}, \sphinxcode{D} pour les couleurs par défaut, puis \sphinxcode{Ctrl+Retour arrière} pour remplir le calque avec la couleur d'arrière-plan)

\end{itemize}

Pour rogner automatiquement les pixels autour de la bordure de l'image :
\begin{itemize}
\item {} 
sélectionnez le menu \sphinxmenuselection{Image \(\rightarrow\) Rognage}.

\item {} 
Cliquez sur le bouton \sphinxcode{pixel supérieur gauche} laissez le reste inchangé et cliquez sur \sphinxcode{OK}.

\item {} 
La dernière étape consiste à enregistrer l'image dans un format de fichier qui intègre la couche alpha. Choisissez le format \sphinxcode{*.tiff} et vérifiez que couche alpha est cochée dans les options. Dans les options, choisissez LZW dans le mode de compression.

\item {} 
Vous pouvez choisir un format alternatif comme le \sphinxcode{*.png}, si vous désirez ``alléger'' cette image

\end{itemize}


\subsection{Réutilisez votre travail}
\label{psd/detourage:reutilisez-votre-travail}
Ce que vous venez de réaliser représente une somme considérable de travail ... pour un seul personnage! Imaginez la quantité de travail nécéssaire pour une foule!

Il est bon de se créer une \emph{bibliothèque} de ces personnages/objets détourés \emph{avec canal alpha} et de l'organiser calirement pour une réutilisation aisée : idéalement, placez ce travail dans un dossier réseau, avec des droit de lecture pour tous et d'écriture/lecture pour l'administrateur. Il y aura donc une seule personne capable d'ajouter d'éditer la bibliothèque, c'est plus sûr!


\subsection{Achetez celui des autres}
\label{psd/detourage:achetez-celui-des-autres}
Vous vous doutez bien que des petits malins y ont pensé avant vous!

Si votre patron est prêt à investir, n'hésitez pas à lui faire la comparaison entre coût d'achat produit externe / temps de travail interne en lui faisant visiter des sites web comme :
\begin{itemize}
\item {} \begin{description}
\item[{Images libres de droits (gratuites?)}] \leavevmode\begin{itemize}
\item {} 
\url{http://www.all-free-photos.com/fr/main-fr.php}

\item {} 
\url{http://pdphoto.org/}

\item {} 
\url{http://www.photo-libre.fr/index.html}

\item {} 
\url{http://gimp-savvy.com/PHOTO-ARCHIVE/}

\item {} 
\url{http://www.flickr.com/}

\item {} 
\url{http://browse.deviantart.com/collections/}

\item {} 
\url{http://www.sxc.hu/}

\item {} 
\url{http://www.freefoto.com/index.jsp}

\item {} 
\url{http://www.imagearc.com/imagearc.html}

\item {} 
etc.

\end{itemize}

\end{description}

\item {} \begin{description}
\item[{payantes (couche alpha?)}] \leavevmode\begin{itemize}
\item {} 
\url{http://fr.fotolia.com/}

\item {} 
\url{http://www.phovoir-images.com/}

\item {} 
\url{http://www.epictura.fr/}

\item {} 
etc.

\end{itemize}

\end{description}

\end{itemize}
\begin{quote}
\phantomsection\label{psd/prepa-image_su+psd:prepa-image-su-psd}\end{quote}


\section{Préparation de la perspective}
\label{psd/prepa-image_su+psd:prepa-image-su-psd}\label{psd/prepa-image_su+psd:preparation-de-la-perspective}\label{psd/prepa-image_su+psd::doc}

\subsection{Sketchup}
\label{psd/prepa-image_su+psd:sketchup}

\subsubsection{Importation du modèle}
\label{psd/prepa-image_su+psd:importation-du-modele}
\noindent\sphinxincludegraphics[width=0.700\linewidth]{{mdl01-maison-ecoconst-vast_ombre+couleur}.png}
\begin{itemize}
\item {} 
télécharger un modèle de maison type sur ``l'entrepôt 3D'' de google \emph{3D Warehouse}, on prendra un modèle moderne, dans l'esprit ``écoconstruction''. Cliquez sur le lien et enregistrez le fichier sur votre disque dur : \url{http://sketchup.google.com/3dwarehouse/details?mid=77f4a9f5ee868f9e972ec726e7a08cca\&prevstart=0}

\item {} 
vous pouvez utiliser tout autre ``matériau''. Allez voir la section {\hyperref[init_su+acad/demarrage:demarrage\string-init\string-su\string-acad]{\sphinxcrossref{\DUrole{std,std-ref}{Démarrage}}}} du projet ``porkeno'' pour voir si vous avez effectué quelquechose dans ce domaine ...

\item {} 
on ouvre le dessin dans sketchup, et on le dispose de 3/4 arrière, pour simuler une perspective.

\end{itemize}

Ce placement est important car dans le résultat final, on tentera de faire apparaître une extrémité en premier plan en couleur+texture, tandis que l'arrière-plan du modèle apparaîtra en blanc+arêtes et flou en prime. (voir le résultat final) .

Une fois l'orientation choisie, on ne bougera plus le modèle!

L'idéal est de créer une scène.


\subsubsection{Création des scènes}
\label{psd/prepa-image_su+psd:creation-des-scenes}
Par le menu \sphinxmenuselection{Fichier \textgreater{} Exporter \textgreater{} Graphique 2D}, on va faire \textbf{3} images.

Ces images seront exportées à partir de ``scènes'' qui correspondent à la même perspective, mais avec des styles différents.
\begin{enumerate}
\item {} 
\textbf{Perspective couleur}

\end{enumerate}

Cette scène sert base de travail, on y peaufine  le positionnement du modèle (3/4 arrière, en fonction de linsertion ultérieure, donc du terrain ...
\begin{itemize}
\item {} 
Réglons l'affichage :
\begin{itemize}
\item {} 
Affichage \textgreater{} Style d'arêtes \textgreater{} décochez tous les types d'arêtes

\item {} 
Affichage \textgreater{} Style de faces \textgreater{} Ombré avec textures

\end{itemize}

\end{itemize}
\begin{enumerate}
\setcounter{enumi}{1}
\item {} 
\textbf{Perspective style ``croquis'' noir et blanc}

\end{enumerate}

Export d'une image sur fond blanc, avec les arêtes soulignées. Cette image sera utilisée pour afficher les arêtes en superposition de l'image en couleur (avant-plan) et pour afficher uniquement les arêtes en arrière plan, dans Photoshop :
\begin{itemize}
\item {} 
Affichage \textgreater{} Style d'arêtes \textgreater{} afficher les arêtes \& arêtes profilées \& arêtes prolongées

\item {} 
Affichage \textgreater{} Style de faces \textgreater{} lignes cachées

\item {} 
fenêtre \textgreater{} Style \textgreater{} sélectionnez le style : Arêtes de croquis \textgreater{} Croquis au marqueur

\item {} 
éditez le style, en réglant le niveau de détails au maximum

\end{itemize}
\begin{enumerate}
\setcounter{enumi}{2}
\item {} 
\textbf{Perspective style ``croquis'' + brouillard}

\end{enumerate}

Export d'une image sur fond noir, avec un flou dans l'arrière-plan créé par un ``brouillard''. L'objectif est d'avoir un arrière plan très sombre et un avant plan très clair. Cette image sera utilisée pour le remplissage du masque de fusion, dans Photoshop :

À partir de l'affichage précédent (Style \textgreater{} Croquis au marqueur = fond blanc + trait noirs forts)
* afficher la fenêtre brouillard : Fenêtres \textgreater{} Brouillard
* réglages des options de brouillard :
\begin{itemize}
\item {} 
cochez ``Afficher le brouillard''

\item {} 
décochez ``Utiliser la couleur d'arrière plan''

\item {} 
cliquez dans le carré blanc pour ouvrir le sélecteur de couleurs

\item {} 
faites glisser le curseur vers le bas pour obtenir du noir

\item {} 
validez par OK : le fond devient noir

\item {} 
avec les curseurs de la barre centrale (il se déplace à la souris, ou – plus finement – en cliquant dessus puis en utilisant les flèches directionnelles gauche et droite) :

\item {} 
faites glisser le curseur droit vers la gauche : le brouillard quote\{ avance\}(obscurcissement sur l'arrière)

\item {} 
faites glisser le curseur gauche vers la droite : l'éclairement se renforce sur l'avant

\end{itemize}


\subsubsection{Exportation des images}
\label{psd/prepa-image_su+psd:exportation-des-images}\begin{enumerate}
\item {} 
Pour la première scène créée,
\begin{itemize}
\item {} 
réglons l'export des images :
\begin{itemize}
\item {} 
Fichier \textgreater{} Exporter \textgreater{} Graphique 2D

\item {} 
Type d'exportation \textgreater{} portable network graphics (\sphinxcode{*.png})

\item {} 
Options \textgreater{} Largeur = 1200, hauteur =\textasciitilde{} (automatiquement fonction de la largeur), Anticrénelage

\end{itemize}

\item {} 
Cliquez sur Exporter
\begin{itemize}
\item {} 
en personne bien organisée, choisissez un lieu sûr parce que évident pour stocker vos image : le dossier \sphinxtitleref{dessin/images} de votre projet!

\item {} 
une appelation explicite est mère de clarté, quelquechose comme \sphinxcode{porkeno\_perspective\_couleur.png} me semble correct

\end{itemize}

\end{itemize}

\item {} 
Pour la 2ème scène
\begin{itemize}
\item {} 
négligez les options d'exportations, elles restent configurées comme pour la fois précédente

\item {} 
exporter et enregistrez en donnant un nom légèrement différent, comme \sphinxcode{porkeno\_perspective\_croquis.png}

\end{itemize}

\item {} 
Pour la 3ème scène : \sphinxcode{porkeno\_perspective\_brouillard.png}

\end{enumerate}


\subsection{Photoshop}
\label{psd/prepa-image_su+psd:photoshop}

\subsubsection{Ouverture des images}
\label{psd/prepa-image_su+psd:ouverture-des-images}
Ouvrez les 3 images précédemment exportées :
\begin{itemize}
\item {} 
lancez Photoshop

\item {} 
\textbf{\texttt{ctrl+o}} et sélectionnez l'ensemble des images à ouvrir, soit :

\end{itemize}

\noindent\sphinxincludegraphics[width=0.700\linewidth]{{capt_ng_sketchup_su+psp_rendu-facades_04}.png}


\subsubsection{Travail sur l'image de couleur}
\label{psd/prepa-image_su+psd:travail-sur-l-image-de-couleur}
\noindent\sphinxincludegraphics[width=0.700\linewidth]{{capt_ng_sketchup_su+psp_rendu-facades_05}.png}

Renforcement des arêtes de l'image de couleur :
* placez-vous sur l'image à fond blanc et arêtes vues :
\begin{itemize}
\item {} 
sélectionner l'ensemble \sphinxcode{ctrl+a}

\item {} 
copiez \sphinxcode{ctrl+c}

\end{itemize}
\begin{itemize}
\item {} 
placez-vous sur l'image en couleur :
\begin{itemize}
\item {} 
collez l'image précédente \sphinxcode{ctrl+c} : un nouveau calque apparaît, au-dessus du calque contenant l'image en couleur

\item {} 
double-cliquez sur l'icône du calque (ou faites un \sphinxcode{clic-droit} --\textgreater{} option de fusion) pour ouvrir le dialogue de fusion (avec le calque sous-jacent)

\item {} 
mode de fusion \textgreater{} produit, \& laissez les autres options intouchées. Le mode ``produit'' permet d'afficher uniquement les ombres du calque supérieur (les zones blanches disparaissent). Vous remarquerez que l'image en couleur du calque inférieur voit ses arêtes ``renforcées''.

\end{itemize}

\end{itemize}

\noindent\sphinxincludegraphics[width=0.700\linewidth]{{capt_ng_sketchup_su+psp_rendu-facades_05}.png}

Fondu entre les 2 calques :
* placez-vous sur l'image au fond noir :
\begin{itemize}
\item {} 
sélectionner l'ensemble : \{bf ctrl+a\}

\item {} 
copiez : \{bf ctrl+c\}

\end{itemize}
\begin{itemize}
\item {} 
placez-vous sur l'image en couleur :
\begin{itemize}
\item {} 
désactiver la visibilité de tous les claques, en cliquant sur les icônes représentant des yeux, sur la gauche des icônes de calque

\item {} 
créez un masque de fusion en cliquant sur l'icône en bas du panneau des calques, ou par le menu Calques \textgreater{} masque de fusion \textgreater{} tout faire apparaître

\item {} 
appuyez sur la touche alt et cliquez sur l'icône du masque de fusion : la zone d'image se remplit de blanc : on vient d'activer le masque, donc toutes les modifications que l'on fait à présent se font sur le masque de fusion.

\item {} 
collez l'image à fond noir dans le masque de fusion : \{bf ctrl+v\} sur la zone d'image.

\item {} 
ré-affichez les calques, et annulez la sélection en cours : \{bf ctrl+d\}

\end{itemize}

\end{itemize}


\subsubsection{Effet de flou}
\label{psd/prepa-image_su+psd:effet-de-flou}
\noindent\sphinxincludegraphics{{capt_ng_sketchup_su+psp_rendu-facades_10}.png}
\begin{itemize}
\item {} 
Placez-vous sur l'image en couleur

\item {} 
sélectionnez le calque contenant le masque de fusion

\item {} 
Filtre \textgreater{} Atténuation \textgreater{} Flou de l'objectif

\item {} 
Source \textgreater{} masque de fusion

\item {} 
Flou de la distance focale \textgreater{} 255

\item {} 
Rayon \textgreater{} 10

\item {} 
laissez le reste inchangé

\item {} 
et admirez le résultat !

\end{itemize}

\noindent\sphinxincludegraphics[width=0.700\linewidth]{{mdl01-maison-ecoconst-vast-final}.png}


\section{Insertion dans le site}
\label{psd/insertion::doc}\label{psd/insertion:insertion-dans-le-site}
Si vous arrivez ici, c'est pour insérer une image de perspective de bâtiment sur un paysage, pour réaliser ce que l'on appelle chez les archis une ``insertion dans le site''.

Ce document est indispensable car il fait partie du volet ``paysager'' du dossier Permis de Construire.


\subsection{Préparation de l'image}
\label{psd/insertion:preparation-de-l-image}\begin{description}
\item[{Vous disposez d'une image de perspective de votre projet de construction réalisée au moyen de :}] \leavevmode\begin{itemize}
\item {} 
un croquis manuel, que vous avez numérisé avec un scanner

\item {} 
une image réalisée par un logiciel de rendu (Artlantis, V-Ray, Mental Ray, Maxwell, Piranesi, Kerkythea, etc.)

\item {} 
une perspective réalisée directement dans un logiciel de dessin (Sketchup, AutoCAD, Archicad, Allplan, Vectorworks, etc.) au moyen d'une capture d'écran, etc.

\end{itemize}

\end{description}

Cette image à insérer sur un paysage peut être améliorée au préalable, etc. Pour les curieux, voir la section {\hyperref[psd/prepa\string-image_su+psd:prepa\string-image\string-su\string-psd]{\sphinxcrossref{\DUrole{std,std-ref}{Préparation de la perspective}}}} pour découvrir \textbf{une} façon de faire cela en associant Sketchup et Photshop.


\subsection{Photomontage (ajout d'une image sur une autre)}
\label{psd/insertion:photomontage-ajout-d-une-image-sur-une-autre}

\subsubsection{Préparation}
\label{psd/insertion:preparation}\begin{itemize}
\item {} \begin{description}
\item[{depuis Photoshop, ouvrir :}] \leavevmode\begin{itemize}
\item {} 
image de fond

\item {} 
image contenant l'objet à insérer

\end{itemize}

\end{description}

\item {} \begin{description}
\item[{sur l'image à insérer :}] \leavevmode\begin{itemize}
\item {} 
sélectionner grossièrement (en réalisant un contour fermé avec une marge autour de l'objet) au lasso \textbf{\texttt{L}}

\item {} 
copier \sphinxcode{CTRL+C}

\end{itemize}

\end{description}

\item {} \begin{description}
\item[{se déplacer sur l'image du fond, en faisant \sphinxcode{CTRL+TAB} :}] \leavevmode\begin{itemize}
\item {} 
coller l'image de premier-plan sur celle du fond, en faisant \sphinxcode{CTRL+V} : un nouveau calque apparaît au-dessus de celui du fond

\end{itemize}

\end{description}

\end{itemize}

\begin{notice}{note}{Note:}
les techniques de sélection sont nombreuses : baguette magique, plage de couleurs, lasso, plume, etc.

Ces techniques varient en fonction de l'objet à ``capturer'' : plus vous avez de détails, plus elle sera sophistiquée. Retenez le fait que l'on peut :
* \textbf{ajouter} à une sélection existante grâce à la touche \sphinxcode{MAJ} (avec l'outil initial de sélection, ou avec le lasso)
* \textbf{soustraire} à la sélection initiale avec la touche \sphinxcode{ALT}

Par exemple, si vous voulez extraire la perspective du projet ``Porkeno'' réalisé sur Sketchup, la technique qui semble la plus appropriée semble être la ``baguette magique'' ou la ``sélection de couleurs'', suivie de l'inversion de sélection
\end{notice}


\subsubsection{Déplacement \& transformation du collage}
\label{psd/insertion:deplacement-transformation-du-collage}
Positionnement correct, rotation, déformation, mise à léchelle, etc. :
\begin{itemize}
\item {} 
sur l'image de fond, qui vient de recevoit la perspective en premier plan :

\item {} 
activez le nouveau calque de l'objet en cliquant sur son calque (palette des calques)

\item {} 
vous pouvez déplacer cet objet avec l'outil déplacement \textbf{\texttt{V}}

\item {} 
en ajuster la taille avec \sphinxcode{CTRL+T} (transformation manuelle) et validation

\item {} 
le déformer \sphinxmenuselection{édition \(\rightarrow\) transformation \(\rightarrow\) déformation} et validation

\item {} 
en modifier la perspective \sphinxmenuselection{édition \(\rightarrow\) transformation \(\rightarrow\) perspective} et validation

\end{itemize}


\subsubsection{Nettoyage de l'image rapportée}
\label{psd/insertion:nettoyage-de-l-image-rapportee}
Il se peut que les bords de l'image rapportée ne soit pas bien ``découpés'' par la sélection préalable.

Nous allons appliquer une technique qui permet de re-découper l'image sans la détruire, et de façon révevrsible : le travail sur masque de fusion.
\begin{itemize}
\item {} 
Ajoutez un masque de fusion en cliquant sur l'icône appropriée, en bas de la palette des calques, ou par le menu \sphinxmenuselection{calques \(\rightarrow\) ajouter un masque de fusion \(\rightarrow\) tout faire apparaître}

\item {} 
mettre les couleurs par défaut : avant-plan en noir, arrière-plan en blanc, en cliquant sur l'icône de la petite flèche, située à proximité de l'icône des couleurs d'avant plan/arrière plan, ou en appuyant sur \textbf{\texttt{D}}. Vous pourrez intervertir ces couleurs en appuyant sur la lettre \textbf{\texttt{X}}.

\item {} 
cliquez sur l'icône du masque de fusion, dans la palette des calques, pour activer celui-ci : son icône doit être ``doublement'' encadrée

\item {} 
prendre une brosse aux contour flous \textbf{\texttt{B}}, choisir une taille appropriée \sphinxcode{clic-droit} et ``peignez'' la zone à enlever autour de l'objet.

\end{itemize}

Le fait de peindre en \textbf{noir} sur le masque de fusion fait \textbf{disparaître} le contour de l'objet.

Le fait de peindre en \textbf{blanc} sur le masque de fusion fait \textbf{apparaître} le contour de l'objet.

Vous pouvez donc corriger vos erreurs, en intervertissant les couleurs de la brosse, en appuyant sur la lettre \textbf{\texttt{X}} lorsque vous peignez les contours.

Vous pouvez avoir besoin de \emph{zoomer} pour afficher plus de détails : appuyez sur les touches \sphinxcode{CTRL + +} (à l'inverse, pour ``dé-zoomer'', faites:kbd:\sphinxtitleref{CTRL + -} )

Pour déplacer l'image, sans perdre l'outil en cours, appuyez sur la \sphinxcode{BARRE ESPACE} et relâcher-là lorsque vous estimer que le déplacement est fini.

Le diamètre de a brosse peut être changé

Ce nettoyage est beaucoup moins agressif que la gomme \textbf{\texttt{E}} car il ne fait pas disparaître réellement le contour, il le masque. Il suffit de désactiver le masque de fusion pour voir réapparaître le contour autour de l'objet.


\subsubsection{Harmonisation de l'image rapportée avec le fond}
\label{psd/insertion:harmonisation-de-l-image-rapportee-avec-le-fond}\begin{itemize}
\item {} 
Une fois le nettoyage de l'objet terminé

\item {} 
Activez le calque de l'image rapportée (qui contient aussi le masque de fusion)

\item {} 
ajoutez un réglage teinte/saturation

\item {} 
se positionner sur ce nouveau calque, et \sphinxcode{clic-droit} ajouter un masque d'écrêtage : le réglage ne sera effectué que sur le calque/objet situé en dessous du calque de réglage.

\item {} 
déplacez les curseurs du réglage \sphinxincludegraphics[width=0.300\linewidth]{{tuto-insertion_palette_calques}.png}

\end{itemize}


\section{Préparer une jaquette de CD}
\label{psd/jaquette_CD:preparer-une-jaquette-de-cd}\label{psd/jaquette_CD::doc}
texte entièrement sorti de \url{http://www.octopussy-world.comdossiers/jaquette.php} . Je remercie Vivien GROS pour la clarté de ses explications.


\subsection{Pourquoi?}
\label{psd/jaquette_CD:pourquoi}
Imaginez votre projet enfin terminé : les plans, coupes et façades, mais aussi les plans de vente, les perspectives et autres rendu, etc.

C'est le moment de l'immortaliser sur un CD que vous distriburez au monde entier ... heu à votre secretaire ... pour archive ...


\subsection{La face}
\label{psd/jaquette_CD:la-face}
Une jaquette a des dimensions trés précises.

Pour la Face ou la Front créez donc un document de 12,6cm par 12,6cm avec une résolution de 300ppi (pixels par pouce).

\begin{notice}{note}{Note:}
Pourquoi 12,6cm alors que lorsque je mesure la Front d'un CD existant je trouve 12cm ?
Eh bien tout simplement pour créer un fond perdu. Un fond perdu sert de ``marge d'erreur''. Si vous imprimiez votre document sans fond perdu, c'est à dire aux dimensions exactes (12cm x 12cm), au moment de la découpe il y aurait le risque de voir apparaitre une fine bande blanche sur les bords, résultant d'un certain manque de précision. L'ajustement de la découpe ne se fait pas au dixième de mm près, il faut donc prévoir un petit débordement de l'image : le fond perdu. Communément le fond perdu est de 3mm de chaque côté. Il peut être de plus mais jamais de moins
\end{notice}

Maintenant il faut placer des repères afin de définir la zone de travail effective, celle qui se verra vraiment, le carré de 12x12cm.

Affichez les règles : \sphinxcode{Ctrl+R} ou \sphinxmenuselection{Affichage \(\rightarrow\) Afficher les règles} si ce n'est déjà fait.

Faites un \sphinxcode{ClicDroit} sur une règle et choisissez dans le menu contextuel qui apparait \sphinxtitleref{cm}. Vous avez vos côtes en cm.

Pour créer un repère vertical, il suffit de placer votre curseur sur la règle verticale, d'enfoncer le clic et sans le relacher de déplacer le curseur sur le document (le repère est créé). En vous servant des règles placez des repères à 3mm de chaque bord.

Vous obtenez ceci :

\noindent\sphinxincludegraphics[width=0.700\linewidth]{{faire_jaquette_cd_01}.png}


\subsection{L'arrière}
\label{psd/jaquette_CD:l-arriere}
Maintenant il ne vous reste plus qu'à faire pareil avec le Dos ou le Back. Dimensions du Dos : 15,6 cm par 11,8 cm. Largeur des tranches du Dos : 6 mm.

Ce qui devrait vous donner ceci :

\noindent\sphinxincludegraphics[width=0.700\linewidth]{{faire_jaquette_cd_02}.png}


\subsection{Le rond du CD}
\label{psd/jaquette_CD:le-rond-du-cd}
Si vous deviez aussi vous occuper du CD en lui-même voici ses côtes :
\begin{itemize}
\item {} 
Diamètre : 11,8cm

\item {} 
Diamètre du trou central : 3,4cm

\end{itemize}

Il existe une fine bande d`1 mm non-imprimable située à 3,6cm du bord.

Maintenant il est tout à fait possible que votre imprimeur vous demande de ne pas vous occuper du milieu et de fournir une image avec le centre plein. Toute fois je vous conseille vivement de garder ces cotes là en mémoire histoire de ne pas vous retrouver ensuite avec un texte ou un logo coupé !

\begin{notice}{note}{Note:}\begin{enumerate}
\item {} 
Il n'y a pas de fond perdu pour le CD

\item {} 
Il n'est pas necessaire de fournir un document en 300dpi pour le CD. Du fait de la surface du CD, l'encre bave légerement, donc tous les petits détails risquent d'être un peu ``flou''.

\end{enumerate}

Je vous conseille une résolution de 150ppi.
\end{notice}

Vous devriez donc obtenir ceci :

\noindent\sphinxincludegraphics[width=0.700\linewidth]{{faire_jaquette_cd_03}.png}


\subsection{La facilité}
\label{psd/jaquette_CD:la-facilite}
Comme l'auteur est quelqu'un de très très sympa (si, si), il vous propose en libre téléchargement ses gabarits (au format \sphinxcode{*.psd}). Vous pourrez donc comparer votre travail avec celui-ci : \sphinxcode{gabarit\_jaquette-CD}

Maintenant il ne vous reste plus qu'à débrider votre créativité pour faire de jolies jaquettes!


\section{Scan et assemblage d'images}
\label{psd/scan+assemblage_psd+gimp+autopano::doc}\label{psd/scan+assemblage_psd+gimp+autopano:scan-et-assemblage-d-images}\label{psd/scan+assemblage_psd+gimp+autopano:scan-assemblage-images-psd-gimp}

\subsection{Scan : considérations générales}
\label{psd/scan+assemblage_psd+gimp+autopano:scan-considerations-generales}
Les fois où l'on a besoin de scanner des images sont nombreuses :
\begin{itemize}
\item {} \begin{description}
\item[{extension d'une construction existante :}] \leavevmode\begin{itemize}
\item {} 
numérisation du plan papier existant

\item {} 
import dans AutoCAD

\item {} 
mise à l'échelle (en utilisant l'option ``référence'')

\item {} 
vectorisation de celui-ci, ou

\item {} 
reprise complète du dessin, avec l'image en fond de plan

\end{itemize}

\end{description}

\item {} \begin{description}
\item[{ajout d'un plan cadastral :}] \leavevmode\begin{itemize}
\item {} 
numérisation

\item {} 
import dans AutoCAD

\item {} 
mise à l'échelle (en utilisant l'option ``référence'')

\end{itemize}

\end{description}

\item {} \begin{description}
\item[{numérisation de plan papier :}] \leavevmode\begin{itemize}
\item {} 
numérisation de plan A0 (scanner à rouleau)

\item {} 
vectorisation dans la foulée

\item {} 
destruction du plan papier pour gagner en espace de stockage ... et diminuer les risques incendie, dégâts des eaux, au regard du délai de conservation trentenaire des documents.

\end{itemize}

\end{description}

\item {} 
ajout d'un logo, etc.

\end{itemize}

Les appareils de numérisation de documents papier sont des \emph{scanners}. Ils sont équipés d'une caméra qui prend une ``photo'' du document, en fonction d'une zone que l'on aura déterminé à l'avance.

\begin{notice}{note}{Note:}\begin{description}
\item[{Résolution de scannage :}] \leavevmode
attention à la taille des fichiers! Si on cherche à numériser un document avec une résolution trop élévée, la taille de l'image risque d'être disproportionnée par rapport à son utilisation. Une résolution de 300 dpi semble être correcte dans la plupart des cas

\item[{Taille et poids :}] \leavevmode
essayez de travailler avec des images ``pesant'' 3 à 4 Mo maximum. Dans le cas d'un assemblage de plusieurs d'entre-elles, on arrive souvent à une image finale de plus de 100 MO !

\item[{Mémoire vive disponible :}] \leavevmode
Ordinateur 32 bit : 4 GO semblent un minimum. (Essayez de charger une image (scannée bien-sûr!) de \textasciitilde{}150 MO dans AutoCAD 32 bit, sur un ordinateur peu véloce, avec ``juste'' 2 GO de RAM : c'est impossible!)
Pour les autres, en 64 bit, on peut rapidement grimper à 8 GO pour plus de confort.

\end{description}
\end{notice}
\begin{description}
\item[{Logiciels :}] \leavevmode\begin{itemize}
\item {} \begin{description}
\item[{numérisation :}] \leavevmode\begin{itemize}
\item {} 
\href{http://www.adobe.com/fr/products/photoshop/photoshop/}{Photoshop} bien sûr, mais attention : gourmand en RAM. Si vous tenez à travailler sur un ordinateur ``de salon'', essayez

\item {} 
\href{http://www.gimp.org/}{Gimp} : on y arrive aussi, et gratuitement en prime.

\end{itemize}

\end{description}

\item {} \begin{description}
\item[{assemblage :}] \leavevmode\begin{itemize}
\item {} 
les logiciels précedent permettent de ``bricoler'' un assemblage/panorama, soit manuellement, soit à l'aide de script/plugins. Si le projet contient 2400 images, il vaut mieux essayez des logiciels dédiés à cet usage, comme

\item {} 
\href{http://www.autopano.net/fr}{autopano} : léger, pas très cher (100 \texteuro{}) et très performant. (pour les panoramas comportant une netteté époustouflante, ainsi qu'un nombre d'images incroyable, visitez les sites comme \url{http://www.paris-20-gigapixels.com/fr/}, \url{http://www.yosemite-17-gigapixels.com/}, \url{http://www.harlem-13-gigapixels.com/})

\item {} 
autodesk \emph{stitcher} : interface ``design'' mais résultats quelquefois décevants.

\item {} 
hugin (interface graphique à autopano) : gratuit --\textgreater{} rend de bons services, mais patience+ingéniosité nécessaire!

\end{itemize}

\end{description}

\end{itemize}

\end{description}


\subsection{Scan : Photoshop}
\label{psd/scan+assemblage_psd+gimp+autopano:scan-psd}

\subsubsection{Scan}
\label{psd/scan+assemblage_psd+gimp+autopano:scan}\begin{itemize}
\item {} 
déposer le document à scanner sur la vitre d'exposition légèrement en travers, pour forcer le détramage de l'image

\item {} 
\sphinxmenuselection{fichier \(\rightarrow\) importation \(\rightarrow\) epsonGT-1000} . le nom de votre logiciel/pilote ``twain'' changera en fonction de votre scanner...ici c'est un scanner A3 ``EPSON GT10000+''

\item {} 
effectuez une prévisualisation, et réglez la fenêtre de scannage (zone en pointillée) à la taille / portion de document que vous désirez numériser.

\item {} 
réglez les options de scan à des valeurs minimales :
\begin{itemize}
\item {} 
résolution : 300 dpi (attention : un scan en A3 d'un document assez riche en détails, le tout sous une résolution de 1200 dpi, va ``peser'' facilement 350/450 MO !)

\item {} 
si document en couleur : 24 bit couleur

\item {} 
si document au trait : noir et blanc

\end{itemize}

\item {} 
scannez (= numérisez)

\item {} 
fermez le logiciel de scannage

\item {} 
depuis Photoshop, enregistrez l'image  en format ``natif'' par défaut : le \sphinxcode{*.psd} (\sphinxcode{scan\_maison-turlier\_001.psd}, \sphinxcode{scan..2.psd}, etc.)

\end{itemize}

Si ces scans ont un ``poids'' trop important (3 à 4 MO semblent un maximum) :
\begin{itemize}
\item {} 
fichier \textgreater{} eregistrer pour le web et les périphériques

\item {} 
choix paramètre prédéfini : jpeg supérieur

\end{itemize}


\subsubsection{Post-traitement des images}
\label{psd/scan+assemblage_psd+gimp+autopano:post-traitement-des-images}\begin{itemize}
\item {} \begin{description}
\item[{redressement image :}] \leavevmode\begin{itemize}
\item {} 
outil \textgreater{} règle (accessible sous la pipette) : tracez une droite partant du coin gauche supérieur vers le coin droit supérieur

\item {} 
image \textgreater{} rotation de l'image \textgreater{} paramétrée : la valeur de rotation est déjà calculée, grâce au traçage de la règle (si l'image apparaît à l'envers : indiquez 180°)

\item {} 
OK

\end{itemize}

\end{description}

\item {} \begin{description}
\item[{Découpe des bords inutiles :}] \leavevmode\begin{itemize}
\item {} 
outil \textgreater{} recadrage, ou C

\item {} 
tracez une zone

\item {} 
ajustez-là finement

\item {} 
validez par OK

\end{itemize}

\end{description}

\item {} \begin{description}
\item[{Couleurs (tons clairs vers blanc et tons foncés vers noir) :}] \leavevmode\begin{itemize}
\item {} 
outil \textgreater{} pipette (doit être dessous l'outil règle)

\item {} 
maj + pipette et :
\begin{itemize}
\item {} 
cliquez sur 2 points distants les plus blancs de l'image

\item {} 
cliquez sur 2 points distants les plus noirs de l'image

\end{itemize}

\item {} 
affichez la palette information : \sphinxcode{F8}

\item {} 
affichage des valeurs en ``niveau de gris'' : vers le bas de la palette, cliquez sur l'icone de la pipette, en dessous des chifffres, et choisisez ``niveau de gris'' dans le menu déroulant

\item {} 
\sphinxmenuselection{Image \(\rightarrow\)réglage \(\rightarrow\)niveaux}

\item {} 
faites bouger le curseur des tons noirs (à gauche) légèrement vers la droite, jusqu'à que les points noirs présents dans la palette d'information soient à 100\%

\item {} 
faites bouger le curseur des blancs légèrement vers la gauche, jusquà que les points blancs de la palette soient à 0\%

\end{itemize}

\end{description}

\item {} \begin{description}
\item[{Renforcement netteté :}] \leavevmode\begin{itemize}
\item {} 
filtre \textgreater{} renforcement \textgreater{} netteté optimisée

\item {} 
grain : 200\% maxi

\item {} 
rayon : 1 pixel

\end{itemize}

\end{description}

\end{itemize}


\subsection{Scan : Gimp}
\label{psd/scan+assemblage_psd+gimp+autopano:scan-gimp}

\subsection{Assemblage : Photoshop}
\label{psd/scan+assemblage_psd+gimp+autopano:assemblage-photoshop}
utilisation du script ``photomerge'' de création de panoramas
\begin{description}
\item[{\sphinxmenuselection{Fichier \(\rightarrow\) automatisation \(\rightarrow\) photomerge}}] \leavevmode
choix (c'est pour assembler +sieurs scans en une seule image) : collage
sélectionner les fichiers à coller (les 2 images scannées précédentes)
validez (le traitement est automatique)

\end{description}

Avantage : intégré à toshop, facile
Inconvénient : on a aucun contrôle sur le processus, et celui-ci peut donner des résultats abberrants.

Type de format de fichier pour l'enregistrement de l'image assemblée


\subsubsection{Gimp}
\label{psd/scan+assemblage_psd+gimp+autopano:id1}
Ouverture de toutes les images :
* lancez gimp
* avec la commande \sphinxcode{ctrl + o}, ouvrez toutes les images à assembler


\subsection{Assemblage : Autopano Pro}
\label{psd/scan+assemblage_psd+gimp+autopano:assemblage-autopano-pro}
\href{http://www.autopano.net/fr}{Autopano Pro/Giga} est un logiciel dédié à la réalisation de panoramas.

Parmi l'éventail des logiciels offrant cette fonctionnlaité, c'est l'un des plus efficace ; c'est la version payante d'un logiciel libre/gratuit en ligne de commande, \href{http://autopano.kolor.com/}{autopano} , dont l'interface graphique est \href{http://hugin.sourceforge.net/}{Hugin} .

Vous trouverez une documentation très riche sur le \href{http://www.autopano.net/wiki-fr/action/view/Accueil}{wiki} d'autopano


\subsubsection{Installation}
\label{psd/scan+assemblage_psd+gimp+autopano:installation}\begin{enumerate}
\item {} 
Téléchargez la version d'essai en vous rendant sur le site éponyme

\item {} 
Configurez le logiciel pour qu'il utilise au mieux les capacités de votre ordinateur :

\end{enumerate}
\begin{itemize}
\item {} 
préférences :
\begin{itemize}
\item {} 
\% de RAM à utiliser et priorité du processus

\item {} 
nombre de coeurs/processeurs

\item {} 
quantité de RAM

\item {} 
emplacement du répertoire temporaire

\end{itemize}

\item {} 
détection :
\begin{itemize}
\item {} 
qualité : standard ou haute (+ lente)

\item {} 
points de contrôle : 50 minimum

\end{itemize}

\item {} 
optimisation :
\begin{itemize}
\item {} 
algorithme  robuste

\item {} 
correction distorsion des lentilles

\item {} 
distorsion avancée

\end{itemize}

\end{itemize}


\subsubsection{Premier panoramique}
\label{psd/scan+assemblage_psd+gimp+autopano:premier-panoramique}\begin{enumerate}
\item {} 
téléchargez le groupe d'images suivantes \href{http://autopano.kolor.com/sample\_pictures.zip}{sur ce site} (tutoriel autopano, version gratuite) et décompressez les images dans un dossier

\item {} 
lancez Autopano et indiquez l'emplacement de ce dossier

\end{enumerate}

3. Détection :
Lancez la détection, si le logiciel ne l'a pas déjà fait : cela consiste à déterminer si un panorama est possible avec le lot d'images analysées. Vous pouvez encore configurer à cette étape la détection de groupe d'images (basée sur l'écart temporel de prise de vue), activer ou non un mode automatique, etc.
le logiciel détermine un ou plusieurs panoramas en fonction des images présentes :

\sphinxincludegraphics{{autopano_01}.png}

4. Pré-Assemblage :
Le logiciel effectue automatiquement un préassemblage, et il est possible d'affiner ce travail en ré-éditant les préférences/détection du logiciel, notamment si le pré-assemblage ne donne pas satisfaction (raccords imparfaits, etc.)

5. Édition du préassemblage :
Cliquez sur le bouton ``éditer le panorama'' \sphinxincludegraphics{{autopano_02}.png}

Le nombre d'options est important, et nécessite un peu de réflexion :
\begin{itemize}
\item {} \begin{description}
\item[{modes de projection:}] \leavevmode\begin{itemize}
\item {} 
\emph{sphérique (par défaut)} : convient pour des photos panoramiques de plus de 110° de champ horizontal et jusqu'à 360°, quelque soit l'endroit où l'on a placé la ligne d'horizon. C'est donc parfait si l'on a orienté l'appareil photo vers le haut ou la bas pour simuler un objectif à décentrement. Ce mode a tendance a tasser un peu le haut de la photo.

\item {} 
\emph{cylindrique} : convient pour des photos de 110° ou plus mais à la condition qu'il y est peu de décentrement auquel cas le haut (ou le bas) de la photo sera très, trop, étiré. C'est très disgracieux. Quand la ligne d'horizon est proche du milieu de la photo c'est un rendu agréable

\item {} 
\emph{rectilinéaire} : ne convient guère aux photos dont l'angle de champ dépasse les 100° à l'horizontal car comme cette géométrie est orthoscopique, les bords de l'image sont de plus en plus étirés, comme avec un vrai 14 mm

\end{itemize}

\end{description}

\item {} 
vue par image

\item {} 
mise à niveau/redressement : on peut redresser les perspectives

\item {} 
harmonisation des niveaux : ont peut égaliser les tons de différentes images en en prenant une comme repère (ancre)

\end{itemize}

Vous pouvez redresser l'image, ajouter des ligne verticales/horizontales pour ``aider'' le logiciel, retailler l'image, etc.

6. Rendu :
Cliquez sur le bouton ``effectuez'' le rendu pour réaliser l'assemblage final
\begin{itemize}
\item {} \begin{description}
\item[{Taille de sortie :}] \leavevmode\begin{itemize}
\item {} 
restons à 100\% de la taille originelle

\end{itemize}

\end{description}

\item {} \begin{description}
\item[{Algorithme :}] \leavevmode\begin{itemize}
\item {} 
bicubique la plupart du temps

\item {} 
bilinéaire le restant ...(rendu rapide)

\end{itemize}

\end{description}

\item {} \begin{description}
\item[{Mélangeur :}] \leavevmode\begin{itemize}
\item {} 
Smartblend

\end{itemize}

\end{description}

\item {} \begin{description}
\item[{Format :}] \leavevmode\begin{itemize}
\item {} 
si c'est pour réaliser un ``fond de plan'' à insérer dans un logiciel de dessin comme AutoCAD, on a intérêt à minimiser la taille de l'image, donc sa qualité : \textbf{jpg}

\item {} 
si c'est pour imprimer ce panorama sur une feuille double A0, poussons les courseurs à fond ... \textbf{psd/psb}

\end{itemize}

\end{description}

\end{itemize}


\subsubsection{À faire}
\label{psd/scan+assemblage_psd+gimp+autopano:a-faire}
Allez, au travail!

Contexte ``professionnel'' : vous venez de scanner en 18 passes (format A3 avec recouvrement entre les scans) un plan papier A0 datant de 1960.

Il vous faut maintenant assembler ces 18 images, que vous pouvez télécharger \href{http://www.canopee.org/fichiers/teb-d/images/panoramas/scans\_plan-papier-A0\_maison-Turlier/}{sur le site canopee}

Vous y verrez un sous-dossier portant le même nom, avec le terme ``\_phatch'' à la fin : c'est le nom d'un logiciel de traitement par lot d'images : j'ai pu réduire leur taille de 3Mo à 300 KO en 1 minute seulement! Ce logiciel \href{http://photobatch.stani.be/}{phatch} fonctionne aussi bien sous windows que linux. Il est très pratique pour réaliser un traitement identique (ici : réduction de la taille des images) sur un ensembles d'images (il est gratuit bien sûr).

Nota : le résultat de ce travail doit être importé dans AutoCAD, mis à l'échelle et entièrement re-dessiné.

Pour votre confort ... je vous laisse admire/utiliser le résultat à \href{http://www.canopee.org/fichiers/teb-d/images/panoramas/scans\_plan-papier-A0\_maison-Turlier/}{l'adresse web précédente}.


\section{FAQ Photoshop}
\label{psd/faq_psd:faq-photoshop}\label{psd/faq_psd::doc}

\section{Glossaire Photoshop}
\label{psd/glossaire_psd::doc}\label{psd/glossaire_psd:glossaire-photoshop}

\chapter{FTP//Web//Mail}
\label{ftpwebmail/index:index-ftpwebmail}\label{ftpwebmail/index::doc}\label{ftpwebmail/index:ftp-web-mail}

\section{Création d'un compte ``gratuit'' chez Free.fr}
\label{ftpwebmail/compte-free:creation-d-un-compte-gratuit-chez-free-fr}\label{ftpwebmail/compte-free::doc}

\section{Services FREE}
\label{ftpwebmail/services-free:services-free}\label{ftpwebmail/services-free::doc}
Le fait de posséder un compte chez ce FAI (payant ou gratuit, c'est là l'avantage) permet d'accéder à une foule de services bien pratiques.


\subsection{Fax}
\label{ftpwebmail/services-free:fax}
Si votre document à faxer est sous format \sphinxcode{*.pdf}, vous pouvez envoyer ce pdf sur le fax de votre destinataire.

Il existe 2 possibilités :
\begin{enumerate}
\item {} \begin{description}
\item[{Connection à votre \href{http://www.free.fr/adsl/pages/accueil/plus-de-20-exclusivites/service-de-fax.html}{espace web de gestion} :}] \leavevmode\begin{itemize}
\item {} 
Connectez-vous sur votre interface de gestion (Mon compte) à l'aide de votre numéro de téléphone identifiant et du mot de passe associé.

\item {} 
A la rubrique Téléphone, cliquez sur Envoyer un fax.

\item {} 
Renseignez les champs de l'interface. Choisissez ou non de masquer le numéro d'émission, puis saisissez le numéro de votre destinataire. Entrez le code de sécurité affiché à l'image.

\end{itemize}

Cliquez sur Parcourir pour accéder au fichier PDF que vous souhaitez envoyer. Terminez par un clic sur Envoyer.
* Un écran de confirmation d'envoi s'affiche.

\end{description}

\end{enumerate}

Une fois le fax transmis, un eMail de confirmation est envoyé à l'adresse eMail de contact de votre compte Free Haut Débit.

\begin{notice}{note}{Note:}
Le numéro d'émission est votre numéro de téléphone Freebox dont le quatrième chiffre aura été modifié (quatrième chiffre + \textbf{5}).

Exemple : Si votre numéro de téléphone Freebox est : 095**1**456789, votre numéro de fax est 095**6**456789.

Ce numéro fonctionne en émission et en réception.
\end{notice}
\begin{enumerate}
\setcounter{enumi}{1}
\item {} 
utiliser un des nombreux logiciels, souvent gratuits, qui permettent de le faire

\end{enumerate}

Ce qu'il y a de bien, c'est que vous recevrez un mail de confirmation, lors de la réussite de l'envoi.

Ce service permet aussi de s'affranchir des limites d'envoi nationales souvent appliquées aux lignes de fax.
\begin{quote}
\phantomsection\label{ftpwebmail/glossaire_ftpwebmail:glossaire-ftpwebmail}\end{quote}


\section{Glossaire FTP//Web//Mail}
\label{ftpwebmail/glossaire_ftpwebmail::doc}\label{ftpwebmail/glossaire_ftpwebmail:glossaire-ftpwebmail}\label{ftpwebmail/glossaire_ftpwebmail:glossaire-ftp-web-mail}\begin{description}
\item[{FTP\index{FTP|textbf}}] \leavevmode\phantomsection\label{ftpwebmail/glossaire_ftpwebmail:term-ftp}
FileTransfertProtocol : un type particulier de transfert de fichiers.

\end{description}


\chapter{Bureautique}
\label{burotic/index::doc}\label{burotic/index:bureautique}\begin{quote}
\phantomsection\label{burotic/glossaire_burotic:glossaire-burotic}\end{quote}


\section{Glossaire Bureautique}
\label{burotic/glossaire_burotic:glossaire-burotic}\label{burotic/glossaire_burotic::doc}\label{burotic/glossaire_burotic:glossaire-bureautique}\begin{description}
\item[{pdf\index{pdf|textbf}}] \leavevmode\phantomsection\label{burotic/glossaire_burotic:term-pdf}
Portable Document Format : format de fichier de \textbf{visualisation} : permet d'avoir une représentation fidèle de ce qui peut être imprimé. Utilisé partout et par tous.

C'est aussi un mode de \textbf{sauvegarde}, car les imrimantes \emph{virtuelles} qui génèrent ces type de document le font de plus en plus en mode \emph{vectoriel}

\end{description}

Pour toute erreur, commentaire positif, etc., écrivez moi sur \sphinxcode{o point turlier at gmail point com}

Document en version 1.0 indice 1.0.1 alpha du 11-10-2016



\renewcommand{\indexname}{Index}
\printindex
\end{document}
